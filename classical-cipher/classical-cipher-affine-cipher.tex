\sectionthree{Affine cipher}
\begin{python0}
from solutions import *; clear()
\end{python0}

Suppose we use $\Z/26$ instead of \verb!'a'! to \verb!'z'! again.
Recall that the shift cipher is
\[
  E(k, x) = (x + k) \pmod{26}
\]
and
\[
    D(k, x) = (x - k) \pmod{26}
\]

The benefit of translating our encryption/decryption
\lq\lq shift up" and \lq\lq shift down" into
mathematical operations in $\Z/26$, is that
we can now generalize and use different mathematical formulas!

Here's the affine cipher.
The encryption algorithm for the affine cipher looks like this:
\[
E((a,b), x) = (ax + b) \pmod{26}
\]
Note that the key is not one single number -- the key
is $(a, b)$ mod 26 which is made up of two numbers.
Why is that important?

Because this means that there are more key values!
Which means that Eve has to try more key!!!!
Get it??

But what is the decryption function?
Of course we know that
\[
D((a,b), E((a,b), x)) = x \pmod{26}
\]
This means that
\[
D((a,b), ax + b) = x \pmod{26}
\]
Suppose I make a guess ...
When I look at the shift cipher, I notice that the
decryption function is similar in form to the encryption function.
Maybe the decryption function for the affine cipher
is similar in form to the encryption function???

Let's try
\[
  D((a,b), x) = cx + d \pmod{26}
\]
In that case, from 
\[
D((a,b), ax + b) \equiv x \pmod{26}
\]
we would get
\[
c(ax + b)) + d \equiv x \pmod{26}
\]
which gives us
\[
cax + cb + d \equiv x \pmod{26}
\]
Now what?
Well maybe
\[
cax \equiv x \pmod{26}
\]
and
\[
cb + d \equiv 0 \pmod{26}
\]

The first condition
\[
cax \equiv x \pmod{26}
\]
is achieved if we have
\[
ca \equiv 1 \pmod{26}
\]

Can this be done?


\begin{ex} 
  \label{ex:semigroup-associativity-0}
  \tinysidebar{\debug{exercises/{exercise-12/question.tex}}}
\mbox{}
  \begin{myenum}
  \item
    Solve
    \[
      5x^2 + y^2 = 3
    \]
    % mod 5, squares = 0^2=0, 1^2=1, 2^2=4, 3^2=4, 4^2=1
    (HINT: You don't really need number theory for this one. Why?
    But if you want to, imitate the solution for the previous
    problem.)
  \item
    Solve
    \[
      11y^2 - 5x^2 = 3
    \]
    % mod 5, squares = 0,1,4
    (HINT: This is just a slight change from the
    previous problem. \textit{But} now you need number theory. Try mod 4.
    If it does not work, try mod 5. Repeat.)
    % 3x^2 + y^2 = 3
    % {0,3} + {0,1} = 3
    % 0, 1, 3, 0 = 3
    % So x = 1, y = 0 (4)
    %
    % y^2=3 (5)
  \item
    Solve
    \[
      y^2 - 5x^2 = 2
    \]
    % mod 5, squares = 0^2=0, 1^2=1, 2^2=4, 3^2=4, 4^2=1    

  \item
    What about this one:
    \[
      x^2 - 5y^2 = 1
    \]    
  \end{myenum}
  {\scriptsize
[\textsc{Aside.}
Integer solutions to $x^2 - dy^2 = 1$ has been studied since at least 400BC.
This equation appear the Cattle Problem of Archimedes:

\begin{itemize}
  \item[]
If thou art diligent and wise, O stranger, compute the number of
cattle of the Sun, who once upon a time grazed on the fields of the
Thrinacian isle of Sicily, divided into four herds of different colours,
one milk white, another a glossy black, the third yellow and the last
dappled. In each herd were bulls, mighty in number according to these
proportions: Understand, stranger, that the white bulls were equal to
a half and a third of the black together with the whole of the yellow,
while the black were equal to the fourth part of the dappled and
a fifth, together with, once more, the whole of the yellow. Observe
further that the remaining bulls, the dappled, were equal to a sixth
part of the white and a seventh, together with all the yellow. These
were the proportions of the cows: The white were precisely equal to the
third part and a fourth of the whole herd of the black; while the black
were equal to the fourth part once more of the dappled and with it a
fifth part, when all, including the bulls went to pasture together. Now
the dappled in four parts8 were equal in number to a fifth part and a
sixth of the yellow herd. Finally the yellow were in number equal to
a sixth part and a seventh of the white herd. If thou canst accurately
tell, O stranger, the number of cattle of the Sun, giving separately the
number of well-fed bulls and again the number of females according
to each colour, thou wouldst not be called unskilled or ignorant of
numbers, but not yet shall thou be numbered among the wise. But
come, understand also all these conditions regarding the cows of the
Sun. When the white bulls mingled their number with the black, they
stood firm, equal in depth and breadth, and the plains of Thrinacia,
stretching far in all ways, were filled with their multitude. Again,
when the yellow and the dappled bulls were gathered into one herd
they stood in such a manner that their number, beginning from one,
grew slowly greater till it completed a triangular figure, there being
no bulls of other colours in their midst nor none of them lacking.
If thou art able, O stranger, to find out all these things and gather
them together in your mind, giving all the relations, thou shalt depart
crowned with glory and knowing that thou hast been adjudged perfect
in this species of wisdom.
\end{itemize}

If $W,X,Y,Z$ represents the number of white, black, yellow,
dappled bulls, you will get 
a systems of 7 linear equations, the first two being 
\begin{align*}
  W &= (1/2 + 1/3)X + Z \\
  X &= (1/4 + 1/5)Y + Z
\end{align*}
together with some constraints such as $W + X$ must be a square.
After some manipulations, it can be shown that the equation to solve looks like
\[
  x^2 - 410286423278424 y^2 = 1
\]
What was Archimedes thinking? You are find information on the Archimedes Cattle Problem on the web.]
}


  \solutionlink{sol:semigroup-associativity-0}
  \qed
\end{ex} 
\begin{python0}
from solutions import *
add(label="ex:semigroup-associativity-0",
    srcfilename='exercises/semigroup-associativity-0/answer.tex') 
\end{python0}



Given an integer $a$, if $c$ satisfies
\[
  ca \equiv 1 \pmod{26}
\]
we say that $c$ is a
\defone{multiplicative inverse}
of $a$ mod 26.
We usually write $c$ as $a^{-1} \mod 26$.
Note that $a^{-1} \mod 26$ is NOT a fraction!!!
It's a whole number.
If $a$ has a multiplicative inverse mod 26, we say that
$a$ is
\defone{invertible}
mod 26.

Recall from a previous section, you already
have a table of $a^{-1} \mod 26$ for all $a$'s in mod 26.


\begin{ex} 
  \label{ex:semigroup-associativity-0}
  \tinysidebar{\debug{exercises/{exercise-12/question.tex}}}
\mbox{}
  \begin{myenum}
  \item
    Solve
    \[
      5x^2 + y^2 = 3
    \]
    % mod 5, squares = 0^2=0, 1^2=1, 2^2=4, 3^2=4, 4^2=1
    (HINT: You don't really need number theory for this one. Why?
    But if you want to, imitate the solution for the previous
    problem.)
  \item
    Solve
    \[
      11y^2 - 5x^2 = 3
    \]
    % mod 5, squares = 0,1,4
    (HINT: This is just a slight change from the
    previous problem. \textit{But} now you need number theory. Try mod 4.
    If it does not work, try mod 5. Repeat.)
    % 3x^2 + y^2 = 3
    % {0,3} + {0,1} = 3
    % 0, 1, 3, 0 = 3
    % So x = 1, y = 0 (4)
    %
    % y^2=3 (5)
  \item
    Solve
    \[
      y^2 - 5x^2 = 2
    \]
    % mod 5, squares = 0^2=0, 1^2=1, 2^2=4, 3^2=4, 4^2=1    

  \item
    What about this one:
    \[
      x^2 - 5y^2 = 1
    \]    
  \end{myenum}
  {\scriptsize
[\textsc{Aside.}
Integer solutions to $x^2 - dy^2 = 1$ has been studied since at least 400BC.
This equation appear the Cattle Problem of Archimedes:

\begin{itemize}
  \item[]
If thou art diligent and wise, O stranger, compute the number of
cattle of the Sun, who once upon a time grazed on the fields of the
Thrinacian isle of Sicily, divided into four herds of different colours,
one milk white, another a glossy black, the third yellow and the last
dappled. In each herd were bulls, mighty in number according to these
proportions: Understand, stranger, that the white bulls were equal to
a half and a third of the black together with the whole of the yellow,
while the black were equal to the fourth part of the dappled and
a fifth, together with, once more, the whole of the yellow. Observe
further that the remaining bulls, the dappled, were equal to a sixth
part of the white and a seventh, together with all the yellow. These
were the proportions of the cows: The white were precisely equal to the
third part and a fourth of the whole herd of the black; while the black
were equal to the fourth part once more of the dappled and with it a
fifth part, when all, including the bulls went to pasture together. Now
the dappled in four parts8 were equal in number to a fifth part and a
sixth of the yellow herd. Finally the yellow were in number equal to
a sixth part and a seventh of the white herd. If thou canst accurately
tell, O stranger, the number of cattle of the Sun, giving separately the
number of well-fed bulls and again the number of females according
to each colour, thou wouldst not be called unskilled or ignorant of
numbers, but not yet shall thou be numbered among the wise. But
come, understand also all these conditions regarding the cows of the
Sun. When the white bulls mingled their number with the black, they
stood firm, equal in depth and breadth, and the plains of Thrinacia,
stretching far in all ways, were filled with their multitude. Again,
when the yellow and the dappled bulls were gathered into one herd
they stood in such a manner that their number, beginning from one,
grew slowly greater till it completed a triangular figure, there being
no bulls of other colours in their midst nor none of them lacking.
If thou art able, O stranger, to find out all these things and gather
them together in your mind, giving all the relations, thou shalt depart
crowned with glory and knowing that thou hast been adjudged perfect
in this species of wisdom.
\end{itemize}

If $W,X,Y,Z$ represents the number of white, black, yellow,
dappled bulls, you will get 
a systems of 7 linear equations, the first two being 
\begin{align*}
  W &= (1/2 + 1/3)X + Z \\
  X &= (1/4 + 1/5)Y + Z
\end{align*}
together with some constraints such as $W + X$ must be a square.
After some manipulations, it can be shown that the equation to solve looks like
\[
  x^2 - 410286423278424 y^2 = 1
\]
What was Archimedes thinking? You are find information on the Archimedes Cattle Problem on the web.]
}


  \solutionlink{sol:semigroup-associativity-0}
  \qed
\end{ex} 
\begin{python0}
from solutions import *
add(label="ex:semigroup-associativity-0",
    srcfilename='exercises/semigroup-associativity-0/answer.tex') 
\end{python0}



Therefore to
satisfy
\[
ca \equiv 1 \pmod{26}
\]
we can't just pick any $a$.
We have to pick an $a$ with a multiplicative inverse mod 26.

After we have chosen a good $a$, what do we do?
We then have
\[
  D((a,b), x) = cx + d \pmod{26}
\]
where $c$ is the multiplicative inverse of $a$ mod 26.
But what about $d$???

Remember we still have the condition
\[
cb + d \equiv 0 \pmod{26}
\]
Writing $a^{-1} \mod 26$ for $c$, we get
\[
a^{-1}b + d \equiv 0 \pmod{26}
\]
we get
\[
d \equiv -a^{-1}b \pmod{26}
\]

Therefore we have the following:
The affine cipher is
\[
E((a,b), x) = (ax + b) \pmod{26}
\]
where $a$ is invertible mod 26 and
\begin{align*}
  D((a,b), x)
  \equiv a^{-1}x - a^{-1}b\pmod{26} \\
  \equiv a^{-1}(x - b)\pmod{26} \\
\end{align*}


\begin{ex} 
  \label{ex:semigroup-associativity-0}
  \tinysidebar{\debug{exercises/{exercise-12/question.tex}}}
\mbox{}
  \begin{myenum}
  \item
    Solve
    \[
      5x^2 + y^2 = 3
    \]
    % mod 5, squares = 0^2=0, 1^2=1, 2^2=4, 3^2=4, 4^2=1
    (HINT: You don't really need number theory for this one. Why?
    But if you want to, imitate the solution for the previous
    problem.)
  \item
    Solve
    \[
      11y^2 - 5x^2 = 3
    \]
    % mod 5, squares = 0,1,4
    (HINT: This is just a slight change from the
    previous problem. \textit{But} now you need number theory. Try mod 4.
    If it does not work, try mod 5. Repeat.)
    % 3x^2 + y^2 = 3
    % {0,3} + {0,1} = 3
    % 0, 1, 3, 0 = 3
    % So x = 1, y = 0 (4)
    %
    % y^2=3 (5)
  \item
    Solve
    \[
      y^2 - 5x^2 = 2
    \]
    % mod 5, squares = 0^2=0, 1^2=1, 2^2=4, 3^2=4, 4^2=1    

  \item
    What about this one:
    \[
      x^2 - 5y^2 = 1
    \]    
  \end{myenum}
  {\scriptsize
[\textsc{Aside.}
Integer solutions to $x^2 - dy^2 = 1$ has been studied since at least 400BC.
This equation appear the Cattle Problem of Archimedes:

\begin{itemize}
  \item[]
If thou art diligent and wise, O stranger, compute the number of
cattle of the Sun, who once upon a time grazed on the fields of the
Thrinacian isle of Sicily, divided into four herds of different colours,
one milk white, another a glossy black, the third yellow and the last
dappled. In each herd were bulls, mighty in number according to these
proportions: Understand, stranger, that the white bulls were equal to
a half and a third of the black together with the whole of the yellow,
while the black were equal to the fourth part of the dappled and
a fifth, together with, once more, the whole of the yellow. Observe
further that the remaining bulls, the dappled, were equal to a sixth
part of the white and a seventh, together with all the yellow. These
were the proportions of the cows: The white were precisely equal to the
third part and a fourth of the whole herd of the black; while the black
were equal to the fourth part once more of the dappled and with it a
fifth part, when all, including the bulls went to pasture together. Now
the dappled in four parts8 were equal in number to a fifth part and a
sixth of the yellow herd. Finally the yellow were in number equal to
a sixth part and a seventh of the white herd. If thou canst accurately
tell, O stranger, the number of cattle of the Sun, giving separately the
number of well-fed bulls and again the number of females according
to each colour, thou wouldst not be called unskilled or ignorant of
numbers, but not yet shall thou be numbered among the wise. But
come, understand also all these conditions regarding the cows of the
Sun. When the white bulls mingled their number with the black, they
stood firm, equal in depth and breadth, and the plains of Thrinacia,
stretching far in all ways, were filled with their multitude. Again,
when the yellow and the dappled bulls were gathered into one herd
they stood in such a manner that their number, beginning from one,
grew slowly greater till it completed a triangular figure, there being
no bulls of other colours in their midst nor none of them lacking.
If thou art able, O stranger, to find out all these things and gather
them together in your mind, giving all the relations, thou shalt depart
crowned with glory and knowing that thou hast been adjudged perfect
in this species of wisdom.
\end{itemize}

If $W,X,Y,Z$ represents the number of white, black, yellow,
dappled bulls, you will get 
a systems of 7 linear equations, the first two being 
\begin{align*}
  W &= (1/2 + 1/3)X + Z \\
  X &= (1/4 + 1/5)Y + Z
\end{align*}
together with some constraints such as $W + X$ must be a square.
After some manipulations, it can be shown that the equation to solve looks like
\[
  x^2 - 410286423278424 y^2 = 1
\]
What was Archimedes thinking? You are find information on the Archimedes Cattle Problem on the web.]
}


  \solutionlink{sol:semigroup-associativity-0}
  \qed
\end{ex} 
\begin{python0}
from solutions import *
add(label="ex:semigroup-associativity-0",
    srcfilename='exercises/semigroup-associativity-0/answer.tex') 
\end{python0}



The above however uses a lot of \lq\lq iffy'' math.
For instance we used the fact
\[
  a(b + c) \equiv ab + ac \pmod{26}
\]
(where?)
We also use the fact that if 
\[
  a + b \equiv 0 \pmod{26}
\]
then
\[
  a \equiv -b \pmod{26}
\]
We seem to be treated math in mod 26 like math in $\Z$
and $\R$!!!
Is that justifiable?
It turns out that the above algebra is actually correct.
I'll have to come back to that later otherwise people will think
we are rambling nonsense and making things up.


\begin{ex} 
  \label{ex:semigroup-associativity-0}
  \tinysidebar{\debug{exercises/{exercise-12/question.tex}}}
\mbox{}
  \begin{myenum}
  \item
    Solve
    \[
      5x^2 + y^2 = 3
    \]
    % mod 5, squares = 0^2=0, 1^2=1, 2^2=4, 3^2=4, 4^2=1
    (HINT: You don't really need number theory for this one. Why?
    But if you want to, imitate the solution for the previous
    problem.)
  \item
    Solve
    \[
      11y^2 - 5x^2 = 3
    \]
    % mod 5, squares = 0,1,4
    (HINT: This is just a slight change from the
    previous problem. \textit{But} now you need number theory. Try mod 4.
    If it does not work, try mod 5. Repeat.)
    % 3x^2 + y^2 = 3
    % {0,3} + {0,1} = 3
    % 0, 1, 3, 0 = 3
    % So x = 1, y = 0 (4)
    %
    % y^2=3 (5)
  \item
    Solve
    \[
      y^2 - 5x^2 = 2
    \]
    % mod 5, squares = 0^2=0, 1^2=1, 2^2=4, 3^2=4, 4^2=1    

  \item
    What about this one:
    \[
      x^2 - 5y^2 = 1
    \]    
  \end{myenum}
  {\scriptsize
[\textsc{Aside.}
Integer solutions to $x^2 - dy^2 = 1$ has been studied since at least 400BC.
This equation appear the Cattle Problem of Archimedes:

\begin{itemize}
  \item[]
If thou art diligent and wise, O stranger, compute the number of
cattle of the Sun, who once upon a time grazed on the fields of the
Thrinacian isle of Sicily, divided into four herds of different colours,
one milk white, another a glossy black, the third yellow and the last
dappled. In each herd were bulls, mighty in number according to these
proportions: Understand, stranger, that the white bulls were equal to
a half and a third of the black together with the whole of the yellow,
while the black were equal to the fourth part of the dappled and
a fifth, together with, once more, the whole of the yellow. Observe
further that the remaining bulls, the dappled, were equal to a sixth
part of the white and a seventh, together with all the yellow. These
were the proportions of the cows: The white were precisely equal to the
third part and a fourth of the whole herd of the black; while the black
were equal to the fourth part once more of the dappled and with it a
fifth part, when all, including the bulls went to pasture together. Now
the dappled in four parts8 were equal in number to a fifth part and a
sixth of the yellow herd. Finally the yellow were in number equal to
a sixth part and a seventh of the white herd. If thou canst accurately
tell, O stranger, the number of cattle of the Sun, giving separately the
number of well-fed bulls and again the number of females according
to each colour, thou wouldst not be called unskilled or ignorant of
numbers, but not yet shall thou be numbered among the wise. But
come, understand also all these conditions regarding the cows of the
Sun. When the white bulls mingled their number with the black, they
stood firm, equal in depth and breadth, and the plains of Thrinacia,
stretching far in all ways, were filled with their multitude. Again,
when the yellow and the dappled bulls were gathered into one herd
they stood in such a manner that their number, beginning from one,
grew slowly greater till it completed a triangular figure, there being
no bulls of other colours in their midst nor none of them lacking.
If thou art able, O stranger, to find out all these things and gather
them together in your mind, giving all the relations, thou shalt depart
crowned with glory and knowing that thou hast been adjudged perfect
in this species of wisdom.
\end{itemize}

If $W,X,Y,Z$ represents the number of white, black, yellow,
dappled bulls, you will get 
a systems of 7 linear equations, the first two being 
\begin{align*}
  W &= (1/2 + 1/3)X + Z \\
  X &= (1/4 + 1/5)Y + Z
\end{align*}
together with some constraints such as $W + X$ must be a square.
After some manipulations, it can be shown that the equation to solve looks like
\[
  x^2 - 410286423278424 y^2 = 1
\]
What was Archimedes thinking? You are find information on the Archimedes Cattle Problem on the web.]
}


  \solutionlink{sol:semigroup-associativity-0}
  \qed
\end{ex} 
\begin{python0}
from solutions import *
add(label="ex:semigroup-associativity-0",
    srcfilename='exercises/semigroup-associativity-0/answer.tex') 
\end{python0}


Now let's look at attacking the affine cipher.

Recall the encryption and decryption of the affine cipher looks
like
\[
 E_{a,b}(x) \equiv (ax + b) \,\,\,(\operatorname{mod} 26),
 \,\,\,\,\,\,\,\,\,\,
 D_{a,b}(x) \equiv a^{-1}(x - b) \,\,\,(\operatorname{mod} 26)
\]
Note that the key is $(a,b)$. 
Note also that $a$ must be invertible mod 26.

Therefore (by the multiplication principle in discrete mathematics),
the total numbers of keys is
\[
\phi(26) \cdot 26 = 312
\]
This is not that big, but it's definitely bigger than the number of keys
for the shift cipher (which is 26).
This means that to carry out a brute force attack on an affine
cipher, assume the attacker has the cipher,
he/she must try 312 possible keys.


\begin{ex} 
  \label{ex:semigroup-associativity-0}
  \tinysidebar{\debug{exercises/{exercise-12/question.tex}}}
\mbox{}
  \begin{myenum}
  \item
    Solve
    \[
      5x^2 + y^2 = 3
    \]
    % mod 5, squares = 0^2=0, 1^2=1, 2^2=4, 3^2=4, 4^2=1
    (HINT: You don't really need number theory for this one. Why?
    But if you want to, imitate the solution for the previous
    problem.)
  \item
    Solve
    \[
      11y^2 - 5x^2 = 3
    \]
    % mod 5, squares = 0,1,4
    (HINT: This is just a slight change from the
    previous problem. \textit{But} now you need number theory. Try mod 4.
    If it does not work, try mod 5. Repeat.)
    % 3x^2 + y^2 = 3
    % {0,3} + {0,1} = 3
    % 0, 1, 3, 0 = 3
    % So x = 1, y = 0 (4)
    %
    % y^2=3 (5)
  \item
    Solve
    \[
      y^2 - 5x^2 = 2
    \]
    % mod 5, squares = 0^2=0, 1^2=1, 2^2=4, 3^2=4, 4^2=1    

  \item
    What about this one:
    \[
      x^2 - 5y^2 = 1
    \]    
  \end{myenum}
  {\scriptsize
[\textsc{Aside.}
Integer solutions to $x^2 - dy^2 = 1$ has been studied since at least 400BC.
This equation appear the Cattle Problem of Archimedes:

\begin{itemize}
  \item[]
If thou art diligent and wise, O stranger, compute the number of
cattle of the Sun, who once upon a time grazed on the fields of the
Thrinacian isle of Sicily, divided into four herds of different colours,
one milk white, another a glossy black, the third yellow and the last
dappled. In each herd were bulls, mighty in number according to these
proportions: Understand, stranger, that the white bulls were equal to
a half and a third of the black together with the whole of the yellow,
while the black were equal to the fourth part of the dappled and
a fifth, together with, once more, the whole of the yellow. Observe
further that the remaining bulls, the dappled, were equal to a sixth
part of the white and a seventh, together with all the yellow. These
were the proportions of the cows: The white were precisely equal to the
third part and a fourth of the whole herd of the black; while the black
were equal to the fourth part once more of the dappled and with it a
fifth part, when all, including the bulls went to pasture together. Now
the dappled in four parts8 were equal in number to a fifth part and a
sixth of the yellow herd. Finally the yellow were in number equal to
a sixth part and a seventh of the white herd. If thou canst accurately
tell, O stranger, the number of cattle of the Sun, giving separately the
number of well-fed bulls and again the number of females according
to each colour, thou wouldst not be called unskilled or ignorant of
numbers, but not yet shall thou be numbered among the wise. But
come, understand also all these conditions regarding the cows of the
Sun. When the white bulls mingled their number with the black, they
stood firm, equal in depth and breadth, and the plains of Thrinacia,
stretching far in all ways, were filled with their multitude. Again,
when the yellow and the dappled bulls were gathered into one herd
they stood in such a manner that their number, beginning from one,
grew slowly greater till it completed a triangular figure, there being
no bulls of other colours in their midst nor none of them lacking.
If thou art able, O stranger, to find out all these things and gather
them together in your mind, giving all the relations, thou shalt depart
crowned with glory and knowing that thou hast been adjudged perfect
in this species of wisdom.
\end{itemize}

If $W,X,Y,Z$ represents the number of white, black, yellow,
dappled bulls, you will get 
a systems of 7 linear equations, the first two being 
\begin{align*}
  W &= (1/2 + 1/3)X + Z \\
  X &= (1/4 + 1/5)Y + Z
\end{align*}
together with some constraints such as $W + X$ must be a square.
After some manipulations, it can be shown that the equation to solve looks like
\[
  x^2 - 410286423278424 y^2 = 1
\]
What was Archimedes thinking? You are find information on the Archimedes Cattle Problem on the web.]
}


  \solutionlink{sol:semigroup-associativity-0}
  \qed
\end{ex} 
\begin{python0}
from solutions import *
add(label="ex:semigroup-associativity-0",
    srcfilename='exercises/semigroup-associativity-0/answer.tex') 
\end{python0}


Again you can do a brute force search for $a,b$. After all there are
not that many possibilities for $a$ and $b$. But we can do better if
we use letter (1--gram) frequencies again. Again suppose you have
computed the frequencies of the letters of the ciphertext and say
that \texttt{g} is the most common letter. So you assume \texttt{e}
is encrypted as \texttt{g}. This is the same as saying $4$ is
encrypted as $6$, i.e.,
\[
 E_{a,b}(4) = 6
\]
right? Now using the formula for $E_{a,b}$ we get
\[
 4a + b = 6
 \]
 To be accurate the equation should be
\[
 4a + b \equiv 6 \pmod{26}
 \]
 
Now suppose the second most common letter in the ciphertext is
\texttt{y}. So you assume that \texttt{t} is encrypted as
\texttt{y}. This means
\[
 20a + b \equiv 24 \pmod{26}
\]
Right? Yes, no? Think about it. So you can solve for $a$ and $b$
from the linear equations
\begin{alignat*}
 4a + b  &\equiv   6 &&\pmod{26}\\
 20a + b &\equiv 24  &&\pmod{26}
\end{alignat*}


\begin{ex} 
  \label{ex:semigroup-associativity-0}
  \tinysidebar{\debug{exercises/{exercise-12/question.tex}}}
\mbox{}
  \begin{myenum}
  \item
    Solve
    \[
      5x^2 + y^2 = 3
    \]
    % mod 5, squares = 0^2=0, 1^2=1, 2^2=4, 3^2=4, 4^2=1
    (HINT: You don't really need number theory for this one. Why?
    But if you want to, imitate the solution for the previous
    problem.)
  \item
    Solve
    \[
      11y^2 - 5x^2 = 3
    \]
    % mod 5, squares = 0,1,4
    (HINT: This is just a slight change from the
    previous problem. \textit{But} now you need number theory. Try mod 4.
    If it does not work, try mod 5. Repeat.)
    % 3x^2 + y^2 = 3
    % {0,3} + {0,1} = 3
    % 0, 1, 3, 0 = 3
    % So x = 1, y = 0 (4)
    %
    % y^2=3 (5)
  \item
    Solve
    \[
      y^2 - 5x^2 = 2
    \]
    % mod 5, squares = 0^2=0, 1^2=1, 2^2=4, 3^2=4, 4^2=1    

  \item
    What about this one:
    \[
      x^2 - 5y^2 = 1
    \]    
  \end{myenum}
  {\scriptsize
[\textsc{Aside.}
Integer solutions to $x^2 - dy^2 = 1$ has been studied since at least 400BC.
This equation appear the Cattle Problem of Archimedes:

\begin{itemize}
  \item[]
If thou art diligent and wise, O stranger, compute the number of
cattle of the Sun, who once upon a time grazed on the fields of the
Thrinacian isle of Sicily, divided into four herds of different colours,
one milk white, another a glossy black, the third yellow and the last
dappled. In each herd were bulls, mighty in number according to these
proportions: Understand, stranger, that the white bulls were equal to
a half and a third of the black together with the whole of the yellow,
while the black were equal to the fourth part of the dappled and
a fifth, together with, once more, the whole of the yellow. Observe
further that the remaining bulls, the dappled, were equal to a sixth
part of the white and a seventh, together with all the yellow. These
were the proportions of the cows: The white were precisely equal to the
third part and a fourth of the whole herd of the black; while the black
were equal to the fourth part once more of the dappled and with it a
fifth part, when all, including the bulls went to pasture together. Now
the dappled in four parts8 were equal in number to a fifth part and a
sixth of the yellow herd. Finally the yellow were in number equal to
a sixth part and a seventh of the white herd. If thou canst accurately
tell, O stranger, the number of cattle of the Sun, giving separately the
number of well-fed bulls and again the number of females according
to each colour, thou wouldst not be called unskilled or ignorant of
numbers, but not yet shall thou be numbered among the wise. But
come, understand also all these conditions regarding the cows of the
Sun. When the white bulls mingled their number with the black, they
stood firm, equal in depth and breadth, and the plains of Thrinacia,
stretching far in all ways, were filled with their multitude. Again,
when the yellow and the dappled bulls were gathered into one herd
they stood in such a manner that their number, beginning from one,
grew slowly greater till it completed a triangular figure, there being
no bulls of other colours in their midst nor none of them lacking.
If thou art able, O stranger, to find out all these things and gather
them together in your mind, giving all the relations, thou shalt depart
crowned with glory and knowing that thou hast been adjudged perfect
in this species of wisdom.
\end{itemize}

If $W,X,Y,Z$ represents the number of white, black, yellow,
dappled bulls, you will get 
a systems of 7 linear equations, the first two being 
\begin{align*}
  W &= (1/2 + 1/3)X + Z \\
  X &= (1/4 + 1/5)Y + Z
\end{align*}
together with some constraints such as $W + X$ must be a square.
After some manipulations, it can be shown that the equation to solve looks like
\[
  x^2 - 410286423278424 y^2 = 1
\]
What was Archimedes thinking? You are find information on the Archimedes Cattle Problem on the web.]
}


  \solutionlink{sol:semigroup-associativity-0}
  \qed
\end{ex} 
\begin{python0}
from solutions import *
add(label="ex:semigroup-associativity-0",
    srcfilename='exercises/semigroup-associativity-0/answer.tex') 
\end{python0}


\begin{ex} 
  \label{ex:semigroup-associativity-0}
  \tinysidebar{\debug{exercises/{exercise-12/question.tex}}}
\mbox{}
  \begin{myenum}
  \item
    Solve
    \[
      5x^2 + y^2 = 3
    \]
    % mod 5, squares = 0^2=0, 1^2=1, 2^2=4, 3^2=4, 4^2=1
    (HINT: You don't really need number theory for this one. Why?
    But if you want to, imitate the solution for the previous
    problem.)
  \item
    Solve
    \[
      11y^2 - 5x^2 = 3
    \]
    % mod 5, squares = 0,1,4
    (HINT: This is just a slight change from the
    previous problem. \textit{But} now you need number theory. Try mod 4.
    If it does not work, try mod 5. Repeat.)
    % 3x^2 + y^2 = 3
    % {0,3} + {0,1} = 3
    % 0, 1, 3, 0 = 3
    % So x = 1, y = 0 (4)
    %
    % y^2=3 (5)
  \item
    Solve
    \[
      y^2 - 5x^2 = 2
    \]
    % mod 5, squares = 0^2=0, 1^2=1, 2^2=4, 3^2=4, 4^2=1    

  \item
    What about this one:
    \[
      x^2 - 5y^2 = 1
    \]    
  \end{myenum}
  {\scriptsize
[\textsc{Aside.}
Integer solutions to $x^2 - dy^2 = 1$ has been studied since at least 400BC.
This equation appear the Cattle Problem of Archimedes:

\begin{itemize}
  \item[]
If thou art diligent and wise, O stranger, compute the number of
cattle of the Sun, who once upon a time grazed on the fields of the
Thrinacian isle of Sicily, divided into four herds of different colours,
one milk white, another a glossy black, the third yellow and the last
dappled. In each herd were bulls, mighty in number according to these
proportions: Understand, stranger, that the white bulls were equal to
a half and a third of the black together with the whole of the yellow,
while the black were equal to the fourth part of the dappled and
a fifth, together with, once more, the whole of the yellow. Observe
further that the remaining bulls, the dappled, were equal to a sixth
part of the white and a seventh, together with all the yellow. These
were the proportions of the cows: The white were precisely equal to the
third part and a fourth of the whole herd of the black; while the black
were equal to the fourth part once more of the dappled and with it a
fifth part, when all, including the bulls went to pasture together. Now
the dappled in four parts8 were equal in number to a fifth part and a
sixth of the yellow herd. Finally the yellow were in number equal to
a sixth part and a seventh of the white herd. If thou canst accurately
tell, O stranger, the number of cattle of the Sun, giving separately the
number of well-fed bulls and again the number of females according
to each colour, thou wouldst not be called unskilled or ignorant of
numbers, but not yet shall thou be numbered among the wise. But
come, understand also all these conditions regarding the cows of the
Sun. When the white bulls mingled their number with the black, they
stood firm, equal in depth and breadth, and the plains of Thrinacia,
stretching far in all ways, were filled with their multitude. Again,
when the yellow and the dappled bulls were gathered into one herd
they stood in such a manner that their number, beginning from one,
grew slowly greater till it completed a triangular figure, there being
no bulls of other colours in their midst nor none of them lacking.
If thou art able, O stranger, to find out all these things and gather
them together in your mind, giving all the relations, thou shalt depart
crowned with glory and knowing that thou hast been adjudged perfect
in this species of wisdom.
\end{itemize}

If $W,X,Y,Z$ represents the number of white, black, yellow,
dappled bulls, you will get 
a systems of 7 linear equations, the first two being 
\begin{align*}
  W &= (1/2 + 1/3)X + Z \\
  X &= (1/4 + 1/5)Y + Z
\end{align*}
together with some constraints such as $W + X$ must be a square.
After some manipulations, it can be shown that the equation to solve looks like
\[
  x^2 - 410286423278424 y^2 = 1
\]
What was Archimedes thinking? You are find information on the Archimedes Cattle Problem on the web.]
}


  \solutionlink{sol:semigroup-associativity-0}
  \qed
\end{ex} 
\begin{python0}
from solutions import *
add(label="ex:semigroup-associativity-0",
    srcfilename='exercises/semigroup-associativity-0/answer.tex') 
\end{python0}


\begin{ex} 
  \label{ex:semigroup-associativity-0}
  \tinysidebar{\debug{exercises/{exercise-12/question.tex}}}
\mbox{}
  \begin{myenum}
  \item
    Solve
    \[
      5x^2 + y^2 = 3
    \]
    % mod 5, squares = 0^2=0, 1^2=1, 2^2=4, 3^2=4, 4^2=1
    (HINT: You don't really need number theory for this one. Why?
    But if you want to, imitate the solution for the previous
    problem.)
  \item
    Solve
    \[
      11y^2 - 5x^2 = 3
    \]
    % mod 5, squares = 0,1,4
    (HINT: This is just a slight change from the
    previous problem. \textit{But} now you need number theory. Try mod 4.
    If it does not work, try mod 5. Repeat.)
    % 3x^2 + y^2 = 3
    % {0,3} + {0,1} = 3
    % 0, 1, 3, 0 = 3
    % So x = 1, y = 0 (4)
    %
    % y^2=3 (5)
  \item
    Solve
    \[
      y^2 - 5x^2 = 2
    \]
    % mod 5, squares = 0^2=0, 1^2=1, 2^2=4, 3^2=4, 4^2=1    

  \item
    What about this one:
    \[
      x^2 - 5y^2 = 1
    \]    
  \end{myenum}
  {\scriptsize
[\textsc{Aside.}
Integer solutions to $x^2 - dy^2 = 1$ has been studied since at least 400BC.
This equation appear the Cattle Problem of Archimedes:

\begin{itemize}
  \item[]
If thou art diligent and wise, O stranger, compute the number of
cattle of the Sun, who once upon a time grazed on the fields of the
Thrinacian isle of Sicily, divided into four herds of different colours,
one milk white, another a glossy black, the third yellow and the last
dappled. In each herd were bulls, mighty in number according to these
proportions: Understand, stranger, that the white bulls were equal to
a half and a third of the black together with the whole of the yellow,
while the black were equal to the fourth part of the dappled and
a fifth, together with, once more, the whole of the yellow. Observe
further that the remaining bulls, the dappled, were equal to a sixth
part of the white and a seventh, together with all the yellow. These
were the proportions of the cows: The white were precisely equal to the
third part and a fourth of the whole herd of the black; while the black
were equal to the fourth part once more of the dappled and with it a
fifth part, when all, including the bulls went to pasture together. Now
the dappled in four parts8 were equal in number to a fifth part and a
sixth of the yellow herd. Finally the yellow were in number equal to
a sixth part and a seventh of the white herd. If thou canst accurately
tell, O stranger, the number of cattle of the Sun, giving separately the
number of well-fed bulls and again the number of females according
to each colour, thou wouldst not be called unskilled or ignorant of
numbers, but not yet shall thou be numbered among the wise. But
come, understand also all these conditions regarding the cows of the
Sun. When the white bulls mingled their number with the black, they
stood firm, equal in depth and breadth, and the plains of Thrinacia,
stretching far in all ways, were filled with their multitude. Again,
when the yellow and the dappled bulls were gathered into one herd
they stood in such a manner that their number, beginning from one,
grew slowly greater till it completed a triangular figure, there being
no bulls of other colours in their midst nor none of them lacking.
If thou art able, O stranger, to find out all these things and gather
them together in your mind, giving all the relations, thou shalt depart
crowned with glory and knowing that thou hast been adjudged perfect
in this species of wisdom.
\end{itemize}

If $W,X,Y,Z$ represents the number of white, black, yellow,
dappled bulls, you will get 
a systems of 7 linear equations, the first two being 
\begin{align*}
  W &= (1/2 + 1/3)X + Z \\
  X &= (1/4 + 1/5)Y + Z
\end{align*}
together with some constraints such as $W + X$ must be a square.
After some manipulations, it can be shown that the equation to solve looks like
\[
  x^2 - 410286423278424 y^2 = 1
\]
What was Archimedes thinking? You are find information on the Archimedes Cattle Problem on the web.]
}


  \solutionlink{sol:semigroup-associativity-0}
  \qed
\end{ex} 
\begin{python0}
from solutions import *
add(label="ex:semigroup-associativity-0",
    srcfilename='exercises/semigroup-associativity-0/answer.tex') 
\end{python0}

