\sectionthree{Hill cipher}
\begin{python0}
from solutions import *; clear()
\end{python0}

\defone{Hill's cipher}
was announced in 1929.
This cipher works with a block of $n$ characters at a time.
For both encryption and decryption, you 
treat each block of $n$ characters (or mod 26 integers) as a column
vector
and we multiply it with an $n$--by--$n$ matrix.
Of course everything is done in mod 26.
Here are more details.

Recall how to multiple a $n$--by--$n$ matrix with a column vector of
size $n$.
I'll do this for $n = 2$.
\[
\begin{pmatrix}
  a & b \\
  c & d 
\end{pmatrix}
\begin{pmatrix}
  x \\
  y 
\end{pmatrix}
=
\begin{pmatrix}
  ax + by \\
  cx + dy 
\end{pmatrix}
\]
(If you have not seen this before take the computer graphics class CISS380.
You can't call yourself a serious computer scienctist if you can't do matrix
multiplication.
Most top CS programs actually has a linear algebra requirement.) 
That's matrix multiplication where all numbers are in $\R$.
The (multiplicative)
\defone{inverse matrix}
of
\[
\begin{pmatrix}
  a & b \\
  c & d 
\end{pmatrix}
\]
is
given by
\[
\begin{pmatrix}
  a & b \\
  c & d 
\end{pmatrix}^{-1}
=
\frac{1}{ad - bc}
\begin{pmatrix}
  d & -b \\
  -c & a 
\end{pmatrix}
\]
The expression $ad - bc$ is called the
\defone{determinant}
of
$\begin{pmatrix}
  a & b \\
  c & d 
\end{pmatrix}$ and is written
\[
\det
\begin{pmatrix}
  a & b \\
  c & d 
\end{pmatrix}
= ad - bc
\]
If a matrix has an inverse, we say that the matrix is
\defone{invertible}.

Not only can you multiply a 2-by-2 matrix
with a 2-by-1, you can actually also multiply
a 2-by-2 matrix
with a 2-by-2.
Here's how you do it:
\[
\begin{pmatrix}
  a & b \\
  c & d 
\end{pmatrix}
\begin{pmatrix}
  A & C \\
  B & D 
\end{pmatrix}
=
\begin{pmatrix}
  aA + bB &   aC + bD \\
  cA + dB &   cC + dD \\
\end{pmatrix}
\]



\begin{ex} 
  \label{ex:semigroup-associativity-0}
  \tinysidebar{\debug{exercises/{exercise-12/question.tex}}}
\mbox{}
  \begin{myenum}
  \item
    Solve
    \[
      5x^2 + y^2 = 3
    \]
    % mod 5, squares = 0^2=0, 1^2=1, 2^2=4, 3^2=4, 4^2=1
    (HINT: You don't really need number theory for this one. Why?
    But if you want to, imitate the solution for the previous
    problem.)
  \item
    Solve
    \[
      11y^2 - 5x^2 = 3
    \]
    % mod 5, squares = 0,1,4
    (HINT: This is just a slight change from the
    previous problem. \textit{But} now you need number theory. Try mod 4.
    If it does not work, try mod 5. Repeat.)
    % 3x^2 + y^2 = 3
    % {0,3} + {0,1} = 3
    % 0, 1, 3, 0 = 3
    % So x = 1, y = 0 (4)
    %
    % y^2=3 (5)
  \item
    Solve
    \[
      y^2 - 5x^2 = 2
    \]
    % mod 5, squares = 0^2=0, 1^2=1, 2^2=4, 3^2=4, 4^2=1    

  \item
    What about this one:
    \[
      x^2 - 5y^2 = 1
    \]    
  \end{myenum}
  {\scriptsize
[\textsc{Aside.}
Integer solutions to $x^2 - dy^2 = 1$ has been studied since at least 400BC.
This equation appear the Cattle Problem of Archimedes:

\begin{itemize}
  \item[]
If thou art diligent and wise, O stranger, compute the number of
cattle of the Sun, who once upon a time grazed on the fields of the
Thrinacian isle of Sicily, divided into four herds of different colours,
one milk white, another a glossy black, the third yellow and the last
dappled. In each herd were bulls, mighty in number according to these
proportions: Understand, stranger, that the white bulls were equal to
a half and a third of the black together with the whole of the yellow,
while the black were equal to the fourth part of the dappled and
a fifth, together with, once more, the whole of the yellow. Observe
further that the remaining bulls, the dappled, were equal to a sixth
part of the white and a seventh, together with all the yellow. These
were the proportions of the cows: The white were precisely equal to the
third part and a fourth of the whole herd of the black; while the black
were equal to the fourth part once more of the dappled and with it a
fifth part, when all, including the bulls went to pasture together. Now
the dappled in four parts8 were equal in number to a fifth part and a
sixth of the yellow herd. Finally the yellow were in number equal to
a sixth part and a seventh of the white herd. If thou canst accurately
tell, O stranger, the number of cattle of the Sun, giving separately the
number of well-fed bulls and again the number of females according
to each colour, thou wouldst not be called unskilled or ignorant of
numbers, but not yet shall thou be numbered among the wise. But
come, understand also all these conditions regarding the cows of the
Sun. When the white bulls mingled their number with the black, they
stood firm, equal in depth and breadth, and the plains of Thrinacia,
stretching far in all ways, were filled with their multitude. Again,
when the yellow and the dappled bulls were gathered into one herd
they stood in such a manner that their number, beginning from one,
grew slowly greater till it completed a triangular figure, there being
no bulls of other colours in their midst nor none of them lacking.
If thou art able, O stranger, to find out all these things and gather
them together in your mind, giving all the relations, thou shalt depart
crowned with glory and knowing that thou hast been adjudged perfect
in this species of wisdom.
\end{itemize}

If $W,X,Y,Z$ represents the number of white, black, yellow,
dappled bulls, you will get 
a systems of 7 linear equations, the first two being 
\begin{align*}
  W &= (1/2 + 1/3)X + Z \\
  X &= (1/4 + 1/5)Y + Z
\end{align*}
together with some constraints such as $W + X$ must be a square.
After some manipulations, it can be shown that the equation to solve looks like
\[
  x^2 - 410286423278424 y^2 = 1
\]
What was Archimedes thinking? You are find information on the Archimedes Cattle Problem on the web.]
}


  \solutionlink{sol:semigroup-associativity-0}
  \qed
\end{ex} 
\begin{python0}
from solutions import *
add(label="ex:semigroup-associativity-0",
    srcfilename='exercises/semigroup-associativity-0/answer.tex') 
\end{python0}


Define the following:
  \[
  I =
  \begin{pmatrix}
    1 & 0 \\
    0 & 1 
  \end{pmatrix}
  \]
  This is called the
  (2-by-2)
  \defone{identity matrix}.



\begin{ex} 
  \label{ex:semigroup-associativity-0}
  \tinysidebar{\debug{exercises/{exercise-12/question.tex}}}
\mbox{}
  \begin{myenum}
  \item
    Solve
    \[
      5x^2 + y^2 = 3
    \]
    % mod 5, squares = 0^2=0, 1^2=1, 2^2=4, 3^2=4, 4^2=1
    (HINT: You don't really need number theory for this one. Why?
    But if you want to, imitate the solution for the previous
    problem.)
  \item
    Solve
    \[
      11y^2 - 5x^2 = 3
    \]
    % mod 5, squares = 0,1,4
    (HINT: This is just a slight change from the
    previous problem. \textit{But} now you need number theory. Try mod 4.
    If it does not work, try mod 5. Repeat.)
    % 3x^2 + y^2 = 3
    % {0,3} + {0,1} = 3
    % 0, 1, 3, 0 = 3
    % So x = 1, y = 0 (4)
    %
    % y^2=3 (5)
  \item
    Solve
    \[
      y^2 - 5x^2 = 2
    \]
    % mod 5, squares = 0^2=0, 1^2=1, 2^2=4, 3^2=4, 4^2=1    

  \item
    What about this one:
    \[
      x^2 - 5y^2 = 1
    \]    
  \end{myenum}
  {\scriptsize
[\textsc{Aside.}
Integer solutions to $x^2 - dy^2 = 1$ has been studied since at least 400BC.
This equation appear the Cattle Problem of Archimedes:

\begin{itemize}
  \item[]
If thou art diligent and wise, O stranger, compute the number of
cattle of the Sun, who once upon a time grazed on the fields of the
Thrinacian isle of Sicily, divided into four herds of different colours,
one milk white, another a glossy black, the third yellow and the last
dappled. In each herd were bulls, mighty in number according to these
proportions: Understand, stranger, that the white bulls were equal to
a half and a third of the black together with the whole of the yellow,
while the black were equal to the fourth part of the dappled and
a fifth, together with, once more, the whole of the yellow. Observe
further that the remaining bulls, the dappled, were equal to a sixth
part of the white and a seventh, together with all the yellow. These
were the proportions of the cows: The white were precisely equal to the
third part and a fourth of the whole herd of the black; while the black
were equal to the fourth part once more of the dappled and with it a
fifth part, when all, including the bulls went to pasture together. Now
the dappled in four parts8 were equal in number to a fifth part and a
sixth of the yellow herd. Finally the yellow were in number equal to
a sixth part and a seventh of the white herd. If thou canst accurately
tell, O stranger, the number of cattle of the Sun, giving separately the
number of well-fed bulls and again the number of females according
to each colour, thou wouldst not be called unskilled or ignorant of
numbers, but not yet shall thou be numbered among the wise. But
come, understand also all these conditions regarding the cows of the
Sun. When the white bulls mingled their number with the black, they
stood firm, equal in depth and breadth, and the plains of Thrinacia,
stretching far in all ways, were filled with their multitude. Again,
when the yellow and the dappled bulls were gathered into one herd
they stood in such a manner that their number, beginning from one,
grew slowly greater till it completed a triangular figure, there being
no bulls of other colours in their midst nor none of them lacking.
If thou art able, O stranger, to find out all these things and gather
them together in your mind, giving all the relations, thou shalt depart
crowned with glory and knowing that thou hast been adjudged perfect
in this species of wisdom.
\end{itemize}

If $W,X,Y,Z$ represents the number of white, black, yellow,
dappled bulls, you will get 
a systems of 7 linear equations, the first two being 
\begin{align*}
  W &= (1/2 + 1/3)X + Z \\
  X &= (1/4 + 1/5)Y + Z
\end{align*}
together with some constraints such as $W + X$ must be a square.
After some manipulations, it can be shown that the equation to solve looks like
\[
  x^2 - 410286423278424 y^2 = 1
\]
What was Archimedes thinking? You are find information on the Archimedes Cattle Problem on the web.]
}


  \solutionlink{sol:semigroup-associativity-0}
  \qed
\end{ex} 
\begin{python0}
from solutions import *
add(label="ex:semigroup-associativity-0",
    srcfilename='exercises/semigroup-associativity-0/answer.tex') 
\end{python0}


\begin{ex} 
  \label{ex:semigroup-associativity-0}
  \tinysidebar{\debug{exercises/{exercise-12/question.tex}}}
\mbox{}
  \begin{myenum}
  \item
    Solve
    \[
      5x^2 + y^2 = 3
    \]
    % mod 5, squares = 0^2=0, 1^2=1, 2^2=4, 3^2=4, 4^2=1
    (HINT: You don't really need number theory for this one. Why?
    But if you want to, imitate the solution for the previous
    problem.)
  \item
    Solve
    \[
      11y^2 - 5x^2 = 3
    \]
    % mod 5, squares = 0,1,4
    (HINT: This is just a slight change from the
    previous problem. \textit{But} now you need number theory. Try mod 4.
    If it does not work, try mod 5. Repeat.)
    % 3x^2 + y^2 = 3
    % {0,3} + {0,1} = 3
    % 0, 1, 3, 0 = 3
    % So x = 1, y = 0 (4)
    %
    % y^2=3 (5)
  \item
    Solve
    \[
      y^2 - 5x^2 = 2
    \]
    % mod 5, squares = 0^2=0, 1^2=1, 2^2=4, 3^2=4, 4^2=1    

  \item
    What about this one:
    \[
      x^2 - 5y^2 = 1
    \]    
  \end{myenum}
  {\scriptsize
[\textsc{Aside.}
Integer solutions to $x^2 - dy^2 = 1$ has been studied since at least 400BC.
This equation appear the Cattle Problem of Archimedes:

\begin{itemize}
  \item[]
If thou art diligent and wise, O stranger, compute the number of
cattle of the Sun, who once upon a time grazed on the fields of the
Thrinacian isle of Sicily, divided into four herds of different colours,
one milk white, another a glossy black, the third yellow and the last
dappled. In each herd were bulls, mighty in number according to these
proportions: Understand, stranger, that the white bulls were equal to
a half and a third of the black together with the whole of the yellow,
while the black were equal to the fourth part of the dappled and
a fifth, together with, once more, the whole of the yellow. Observe
further that the remaining bulls, the dappled, were equal to a sixth
part of the white and a seventh, together with all the yellow. These
were the proportions of the cows: The white were precisely equal to the
third part and a fourth of the whole herd of the black; while the black
were equal to the fourth part once more of the dappled and with it a
fifth part, when all, including the bulls went to pasture together. Now
the dappled in four parts8 were equal in number to a fifth part and a
sixth of the yellow herd. Finally the yellow were in number equal to
a sixth part and a seventh of the white herd. If thou canst accurately
tell, O stranger, the number of cattle of the Sun, giving separately the
number of well-fed bulls and again the number of females according
to each colour, thou wouldst not be called unskilled or ignorant of
numbers, but not yet shall thou be numbered among the wise. But
come, understand also all these conditions regarding the cows of the
Sun. When the white bulls mingled their number with the black, they
stood firm, equal in depth and breadth, and the plains of Thrinacia,
stretching far in all ways, were filled with their multitude. Again,
when the yellow and the dappled bulls were gathered into one herd
they stood in such a manner that their number, beginning from one,
grew slowly greater till it completed a triangular figure, there being
no bulls of other colours in their midst nor none of them lacking.
If thou art able, O stranger, to find out all these things and gather
them together in your mind, giving all the relations, thou shalt depart
crowned with glory and knowing that thou hast been adjudged perfect
in this species of wisdom.
\end{itemize}

If $W,X,Y,Z$ represents the number of white, black, yellow,
dappled bulls, you will get 
a systems of 7 linear equations, the first two being 
\begin{align*}
  W &= (1/2 + 1/3)X + Z \\
  X &= (1/4 + 1/5)Y + Z
\end{align*}
together with some constraints such as $W + X$ must be a square.
After some manipulations, it can be shown that the equation to solve looks like
\[
  x^2 - 410286423278424 y^2 = 1
\]
What was Archimedes thinking? You are find information on the Archimedes Cattle Problem on the web.]
}


  \solutionlink{sol:semigroup-associativity-0}
  \qed
\end{ex} 
\begin{python0}
from solutions import *
add(label="ex:semigroup-associativity-0",
    srcfilename='exercises/semigroup-associativity-0/answer.tex') 
\end{python0}


The set of $n$--by--$n$ matrices with $\R$ values is usually denoted by
$\operatorname{M}_{n}(\R)$.
The set of invertible matrices is usually denoted by
$\operatorname{GL}_{n}(\R)$. This is called the
\defone{general linear group} of $n$-by-$n$ matrices over $\R$.

It turns out that everything works just as fine if the numbers are
in mod 26.
You just need to be careful with the determinant of the inverse
computation.
Recall that
\[
\begin{pmatrix}
  a & b \\
  c & d 
\end{pmatrix}^{-1}
=
\frac{1}{ad - bc}
\begin{pmatrix}
  d & -b \\
  -c & a 
\end{pmatrix}
\]
for values in $\R$.
This is the same for values in $\Z/26$ as long as you think of
\[
\frac{1}{ad - bc}
\]
as the multiplicative inverse of $ad - bd$ in $\Z/26$.


\begin{ex} 
  \label{ex:semigroup-associativity-0}
  \tinysidebar{\debug{exercises/{exercise-12/question.tex}}}
\mbox{}
  \begin{myenum}
  \item
    Solve
    \[
      5x^2 + y^2 = 3
    \]
    % mod 5, squares = 0^2=0, 1^2=1, 2^2=4, 3^2=4, 4^2=1
    (HINT: You don't really need number theory for this one. Why?
    But if you want to, imitate the solution for the previous
    problem.)
  \item
    Solve
    \[
      11y^2 - 5x^2 = 3
    \]
    % mod 5, squares = 0,1,4
    (HINT: This is just a slight change from the
    previous problem. \textit{But} now you need number theory. Try mod 4.
    If it does not work, try mod 5. Repeat.)
    % 3x^2 + y^2 = 3
    % {0,3} + {0,1} = 3
    % 0, 1, 3, 0 = 3
    % So x = 1, y = 0 (4)
    %
    % y^2=3 (5)
  \item
    Solve
    \[
      y^2 - 5x^2 = 2
    \]
    % mod 5, squares = 0^2=0, 1^2=1, 2^2=4, 3^2=4, 4^2=1    

  \item
    What about this one:
    \[
      x^2 - 5y^2 = 1
    \]    
  \end{myenum}
  {\scriptsize
[\textsc{Aside.}
Integer solutions to $x^2 - dy^2 = 1$ has been studied since at least 400BC.
This equation appear the Cattle Problem of Archimedes:

\begin{itemize}
  \item[]
If thou art diligent and wise, O stranger, compute the number of
cattle of the Sun, who once upon a time grazed on the fields of the
Thrinacian isle of Sicily, divided into four herds of different colours,
one milk white, another a glossy black, the third yellow and the last
dappled. In each herd were bulls, mighty in number according to these
proportions: Understand, stranger, that the white bulls were equal to
a half and a third of the black together with the whole of the yellow,
while the black were equal to the fourth part of the dappled and
a fifth, together with, once more, the whole of the yellow. Observe
further that the remaining bulls, the dappled, were equal to a sixth
part of the white and a seventh, together with all the yellow. These
were the proportions of the cows: The white were precisely equal to the
third part and a fourth of the whole herd of the black; while the black
were equal to the fourth part once more of the dappled and with it a
fifth part, when all, including the bulls went to pasture together. Now
the dappled in four parts8 were equal in number to a fifth part and a
sixth of the yellow herd. Finally the yellow were in number equal to
a sixth part and a seventh of the white herd. If thou canst accurately
tell, O stranger, the number of cattle of the Sun, giving separately the
number of well-fed bulls and again the number of females according
to each colour, thou wouldst not be called unskilled or ignorant of
numbers, but not yet shall thou be numbered among the wise. But
come, understand also all these conditions regarding the cows of the
Sun. When the white bulls mingled their number with the black, they
stood firm, equal in depth and breadth, and the plains of Thrinacia,
stretching far in all ways, were filled with their multitude. Again,
when the yellow and the dappled bulls were gathered into one herd
they stood in such a manner that their number, beginning from one,
grew slowly greater till it completed a triangular figure, there being
no bulls of other colours in their midst nor none of them lacking.
If thou art able, O stranger, to find out all these things and gather
them together in your mind, giving all the relations, thou shalt depart
crowned with glory and knowing that thou hast been adjudged perfect
in this species of wisdom.
\end{itemize}

If $W,X,Y,Z$ represents the number of white, black, yellow,
dappled bulls, you will get 
a systems of 7 linear equations, the first two being 
\begin{align*}
  W &= (1/2 + 1/3)X + Z \\
  X &= (1/4 + 1/5)Y + Z
\end{align*}
together with some constraints such as $W + X$ must be a square.
After some manipulations, it can be shown that the equation to solve looks like
\[
  x^2 - 410286423278424 y^2 = 1
\]
What was Archimedes thinking? You are find information on the Archimedes Cattle Problem on the web.]
}


  \solutionlink{sol:semigroup-associativity-0}
  \qed
\end{ex} 
\begin{python0}
from solutions import *
add(label="ex:semigroup-associativity-0",
    srcfilename='exercises/semigroup-associativity-0/answer.tex') 
\end{python0}


\begin{ex} 
  \label{ex:semigroup-associativity-0}
  \tinysidebar{\debug{exercises/{exercise-12/question.tex}}}
\mbox{}
  \begin{myenum}
  \item
    Solve
    \[
      5x^2 + y^2 = 3
    \]
    % mod 5, squares = 0^2=0, 1^2=1, 2^2=4, 3^2=4, 4^2=1
    (HINT: You don't really need number theory for this one. Why?
    But if you want to, imitate the solution for the previous
    problem.)
  \item
    Solve
    \[
      11y^2 - 5x^2 = 3
    \]
    % mod 5, squares = 0,1,4
    (HINT: This is just a slight change from the
    previous problem. \textit{But} now you need number theory. Try mod 4.
    If it does not work, try mod 5. Repeat.)
    % 3x^2 + y^2 = 3
    % {0,3} + {0,1} = 3
    % 0, 1, 3, 0 = 3
    % So x = 1, y = 0 (4)
    %
    % y^2=3 (5)
  \item
    Solve
    \[
      y^2 - 5x^2 = 2
    \]
    % mod 5, squares = 0^2=0, 1^2=1, 2^2=4, 3^2=4, 4^2=1    

  \item
    What about this one:
    \[
      x^2 - 5y^2 = 1
    \]    
  \end{myenum}
  {\scriptsize
[\textsc{Aside.}
Integer solutions to $x^2 - dy^2 = 1$ has been studied since at least 400BC.
This equation appear the Cattle Problem of Archimedes:

\begin{itemize}
  \item[]
If thou art diligent and wise, O stranger, compute the number of
cattle of the Sun, who once upon a time grazed on the fields of the
Thrinacian isle of Sicily, divided into four herds of different colours,
one milk white, another a glossy black, the third yellow and the last
dappled. In each herd were bulls, mighty in number according to these
proportions: Understand, stranger, that the white bulls were equal to
a half and a third of the black together with the whole of the yellow,
while the black were equal to the fourth part of the dappled and
a fifth, together with, once more, the whole of the yellow. Observe
further that the remaining bulls, the dappled, were equal to a sixth
part of the white and a seventh, together with all the yellow. These
were the proportions of the cows: The white were precisely equal to the
third part and a fourth of the whole herd of the black; while the black
were equal to the fourth part once more of the dappled and with it a
fifth part, when all, including the bulls went to pasture together. Now
the dappled in four parts8 were equal in number to a fifth part and a
sixth of the yellow herd. Finally the yellow were in number equal to
a sixth part and a seventh of the white herd. If thou canst accurately
tell, O stranger, the number of cattle of the Sun, giving separately the
number of well-fed bulls and again the number of females according
to each colour, thou wouldst not be called unskilled or ignorant of
numbers, but not yet shall thou be numbered among the wise. But
come, understand also all these conditions regarding the cows of the
Sun. When the white bulls mingled their number with the black, they
stood firm, equal in depth and breadth, and the plains of Thrinacia,
stretching far in all ways, were filled with their multitude. Again,
when the yellow and the dappled bulls were gathered into one herd
they stood in such a manner that their number, beginning from one,
grew slowly greater till it completed a triangular figure, there being
no bulls of other colours in their midst nor none of them lacking.
If thou art able, O stranger, to find out all these things and gather
them together in your mind, giving all the relations, thou shalt depart
crowned with glory and knowing that thou hast been adjudged perfect
in this species of wisdom.
\end{itemize}

If $W,X,Y,Z$ represents the number of white, black, yellow,
dappled bulls, you will get 
a systems of 7 linear equations, the first two being 
\begin{align*}
  W &= (1/2 + 1/3)X + Z \\
  X &= (1/4 + 1/5)Y + Z
\end{align*}
together with some constraints such as $W + X$ must be a square.
After some manipulations, it can be shown that the equation to solve looks like
\[
  x^2 - 410286423278424 y^2 = 1
\]
What was Archimedes thinking? You are find information on the Archimedes Cattle Problem on the web.]
}


  \solutionlink{sol:semigroup-associativity-0}
  \qed
\end{ex} 
\begin{python0}
from solutions import *
add(label="ex:semigroup-associativity-0",
    srcfilename='exercises/semigroup-associativity-0/answer.tex') 
\end{python0}


This should not be surprising:
The set of $n$--by--$n$ matrices with mod 26 values is usually denoted by
$\operatorname{M}_{n}(\Z/26)$.
The set of invertible matrices is usually denoted by
$\operatorname{GL}_{n}(\Z/26)$.


\begin{ex} 
  \label{ex:semigroup-associativity-0}
  \tinysidebar{\debug{exercises/{exercise-12/question.tex}}}
\mbox{}
  \begin{myenum}
  \item
    Solve
    \[
      5x^2 + y^2 = 3
    \]
    % mod 5, squares = 0^2=0, 1^2=1, 2^2=4, 3^2=4, 4^2=1
    (HINT: You don't really need number theory for this one. Why?
    But if you want to, imitate the solution for the previous
    problem.)
  \item
    Solve
    \[
      11y^2 - 5x^2 = 3
    \]
    % mod 5, squares = 0,1,4
    (HINT: This is just a slight change from the
    previous problem. \textit{But} now you need number theory. Try mod 4.
    If it does not work, try mod 5. Repeat.)
    % 3x^2 + y^2 = 3
    % {0,3} + {0,1} = 3
    % 0, 1, 3, 0 = 3
    % So x = 1, y = 0 (4)
    %
    % y^2=3 (5)
  \item
    Solve
    \[
      y^2 - 5x^2 = 2
    \]
    % mod 5, squares = 0^2=0, 1^2=1, 2^2=4, 3^2=4, 4^2=1    

  \item
    What about this one:
    \[
      x^2 - 5y^2 = 1
    \]    
  \end{myenum}
  {\scriptsize
[\textsc{Aside.}
Integer solutions to $x^2 - dy^2 = 1$ has been studied since at least 400BC.
This equation appear the Cattle Problem of Archimedes:

\begin{itemize}
  \item[]
If thou art diligent and wise, O stranger, compute the number of
cattle of the Sun, who once upon a time grazed on the fields of the
Thrinacian isle of Sicily, divided into four herds of different colours,
one milk white, another a glossy black, the third yellow and the last
dappled. In each herd were bulls, mighty in number according to these
proportions: Understand, stranger, that the white bulls were equal to
a half and a third of the black together with the whole of the yellow,
while the black were equal to the fourth part of the dappled and
a fifth, together with, once more, the whole of the yellow. Observe
further that the remaining bulls, the dappled, were equal to a sixth
part of the white and a seventh, together with all the yellow. These
were the proportions of the cows: The white were precisely equal to the
third part and a fourth of the whole herd of the black; while the black
were equal to the fourth part once more of the dappled and with it a
fifth part, when all, including the bulls went to pasture together. Now
the dappled in four parts8 were equal in number to a fifth part and a
sixth of the yellow herd. Finally the yellow were in number equal to
a sixth part and a seventh of the white herd. If thou canst accurately
tell, O stranger, the number of cattle of the Sun, giving separately the
number of well-fed bulls and again the number of females according
to each colour, thou wouldst not be called unskilled or ignorant of
numbers, but not yet shall thou be numbered among the wise. But
come, understand also all these conditions regarding the cows of the
Sun. When the white bulls mingled their number with the black, they
stood firm, equal in depth and breadth, and the plains of Thrinacia,
stretching far in all ways, were filled with their multitude. Again,
when the yellow and the dappled bulls were gathered into one herd
they stood in such a manner that their number, beginning from one,
grew slowly greater till it completed a triangular figure, there being
no bulls of other colours in their midst nor none of them lacking.
If thou art able, O stranger, to find out all these things and gather
them together in your mind, giving all the relations, thou shalt depart
crowned with glory and knowing that thou hast been adjudged perfect
in this species of wisdom.
\end{itemize}

If $W,X,Y,Z$ represents the number of white, black, yellow,
dappled bulls, you will get 
a systems of 7 linear equations, the first two being 
\begin{align*}
  W &= (1/2 + 1/3)X + Z \\
  X &= (1/4 + 1/5)Y + Z
\end{align*}
together with some constraints such as $W + X$ must be a square.
After some manipulations, it can be shown that the equation to solve looks like
\[
  x^2 - 410286423278424 y^2 = 1
\]
What was Archimedes thinking? You are find information on the Archimedes Cattle Problem on the web.]
}


  \solutionlink{sol:semigroup-associativity-0}
  \qed
\end{ex} 
\begin{python0}
from solutions import *
add(label="ex:semigroup-associativity-0",
    srcfilename='exercises/semigroup-associativity-0/answer.tex') 
\end{python0}


Note that in the above, since the block size is 2, you would need to
encrypt a string with a string length that is a multiple of $2$.
If it's not, you can (for instance) pad with some character.

The important thing to observe is that if
\[
  E(M, (x, y)) = (x', y')
  \]
  ($M$ is a matrix key)
note that $x$ takes part in the value $x'$ \textit{and} $y'$.
Same thing for $y'$.
This means that if I change one character in my plaintext, \textit{two}
character in the ciphertext is changed.
Informally this means that the statistical behavior of $x$
is spread out across multiple characters.
Note only that $x'$ also depends on several values of the key (the matrix).
This makes it much hard to analyze the key one part at a time when
studying ciphertexts.
This makes it a lot harder to break Hill's cipher, especially when
the matrice size is larger, say the matrix is 32-by-32.

The above two important observations
were later formalized and studied by Claude Shannon in 1945.
\begin{enumerate}[nosep]
\item The fact that a character of the ciphertext
depends on several parts of the key is the
concept of
\defone{confusion}
and
\item the fact that one character in the plaintext
influences multiple characters in the ciphertext
is called the concept of
\defone{diffusion}
\end{enumerate}
More on that in a much later chapter.

\begin{ex} 
  \label{ex:semigroup-associativity-0}
  \tinysidebar{\debug{exercises/{exercise-12/question.tex}}}
\mbox{}
  \begin{myenum}
  \item
    Solve
    \[
      5x^2 + y^2 = 3
    \]
    % mod 5, squares = 0^2=0, 1^2=1, 2^2=4, 3^2=4, 4^2=1
    (HINT: You don't really need number theory for this one. Why?
    But if you want to, imitate the solution for the previous
    problem.)
  \item
    Solve
    \[
      11y^2 - 5x^2 = 3
    \]
    % mod 5, squares = 0,1,4
    (HINT: This is just a slight change from the
    previous problem. \textit{But} now you need number theory. Try mod 4.
    If it does not work, try mod 5. Repeat.)
    % 3x^2 + y^2 = 3
    % {0,3} + {0,1} = 3
    % 0, 1, 3, 0 = 3
    % So x = 1, y = 0 (4)
    %
    % y^2=3 (5)
  \item
    Solve
    \[
      y^2 - 5x^2 = 2
    \]
    % mod 5, squares = 0^2=0, 1^2=1, 2^2=4, 3^2=4, 4^2=1    

  \item
    What about this one:
    \[
      x^2 - 5y^2 = 1
    \]    
  \end{myenum}
  {\scriptsize
[\textsc{Aside.}
Integer solutions to $x^2 - dy^2 = 1$ has been studied since at least 400BC.
This equation appear the Cattle Problem of Archimedes:

\begin{itemize}
  \item[]
If thou art diligent and wise, O stranger, compute the number of
cattle of the Sun, who once upon a time grazed on the fields of the
Thrinacian isle of Sicily, divided into four herds of different colours,
one milk white, another a glossy black, the third yellow and the last
dappled. In each herd were bulls, mighty in number according to these
proportions: Understand, stranger, that the white bulls were equal to
a half and a third of the black together with the whole of the yellow,
while the black were equal to the fourth part of the dappled and
a fifth, together with, once more, the whole of the yellow. Observe
further that the remaining bulls, the dappled, were equal to a sixth
part of the white and a seventh, together with all the yellow. These
were the proportions of the cows: The white were precisely equal to the
third part and a fourth of the whole herd of the black; while the black
were equal to the fourth part once more of the dappled and with it a
fifth part, when all, including the bulls went to pasture together. Now
the dappled in four parts8 were equal in number to a fifth part and a
sixth of the yellow herd. Finally the yellow were in number equal to
a sixth part and a seventh of the white herd. If thou canst accurately
tell, O stranger, the number of cattle of the Sun, giving separately the
number of well-fed bulls and again the number of females according
to each colour, thou wouldst not be called unskilled or ignorant of
numbers, but not yet shall thou be numbered among the wise. But
come, understand also all these conditions regarding the cows of the
Sun. When the white bulls mingled their number with the black, they
stood firm, equal in depth and breadth, and the plains of Thrinacia,
stretching far in all ways, were filled with their multitude. Again,
when the yellow and the dappled bulls were gathered into one herd
they stood in such a manner that their number, beginning from one,
grew slowly greater till it completed a triangular figure, there being
no bulls of other colours in their midst nor none of them lacking.
If thou art able, O stranger, to find out all these things and gather
them together in your mind, giving all the relations, thou shalt depart
crowned with glory and knowing that thou hast been adjudged perfect
in this species of wisdom.
\end{itemize}

If $W,X,Y,Z$ represents the number of white, black, yellow,
dappled bulls, you will get 
a systems of 7 linear equations, the first two being 
\begin{align*}
  W &= (1/2 + 1/3)X + Z \\
  X &= (1/4 + 1/5)Y + Z
\end{align*}
together with some constraints such as $W + X$ must be a square.
After some manipulations, it can be shown that the equation to solve looks like
\[
  x^2 - 410286423278424 y^2 = 1
\]
What was Archimedes thinking? You are find information on the Archimedes Cattle Problem on the web.]
}


  \solutionlink{sol:semigroup-associativity-0}
  \qed
\end{ex} 
\begin{python0}
from solutions import *
add(label="ex:semigroup-associativity-0",
    srcfilename='exercises/semigroup-associativity-0/answer.tex') 
\end{python0}


\begin{ex} 
  \label{ex:semigroup-associativity-0}
  \tinysidebar{\debug{exercises/{exercise-12/question.tex}}}
\mbox{}
  \begin{myenum}
  \item
    Solve
    \[
      5x^2 + y^2 = 3
    \]
    % mod 5, squares = 0^2=0, 1^2=1, 2^2=4, 3^2=4, 4^2=1
    (HINT: You don't really need number theory for this one. Why?
    But if you want to, imitate the solution for the previous
    problem.)
  \item
    Solve
    \[
      11y^2 - 5x^2 = 3
    \]
    % mod 5, squares = 0,1,4
    (HINT: This is just a slight change from the
    previous problem. \textit{But} now you need number theory. Try mod 4.
    If it does not work, try mod 5. Repeat.)
    % 3x^2 + y^2 = 3
    % {0,3} + {0,1} = 3
    % 0, 1, 3, 0 = 3
    % So x = 1, y = 0 (4)
    %
    % y^2=3 (5)
  \item
    Solve
    \[
      y^2 - 5x^2 = 2
    \]
    % mod 5, squares = 0^2=0, 1^2=1, 2^2=4, 3^2=4, 4^2=1    

  \item
    What about this one:
    \[
      x^2 - 5y^2 = 1
    \]    
  \end{myenum}
  {\scriptsize
[\textsc{Aside.}
Integer solutions to $x^2 - dy^2 = 1$ has been studied since at least 400BC.
This equation appear the Cattle Problem of Archimedes:

\begin{itemize}
  \item[]
If thou art diligent and wise, O stranger, compute the number of
cattle of the Sun, who once upon a time grazed on the fields of the
Thrinacian isle of Sicily, divided into four herds of different colours,
one milk white, another a glossy black, the third yellow and the last
dappled. In each herd were bulls, mighty in number according to these
proportions: Understand, stranger, that the white bulls were equal to
a half and a third of the black together with the whole of the yellow,
while the black were equal to the fourth part of the dappled and
a fifth, together with, once more, the whole of the yellow. Observe
further that the remaining bulls, the dappled, were equal to a sixth
part of the white and a seventh, together with all the yellow. These
were the proportions of the cows: The white were precisely equal to the
third part and a fourth of the whole herd of the black; while the black
were equal to the fourth part once more of the dappled and with it a
fifth part, when all, including the bulls went to pasture together. Now
the dappled in four parts8 were equal in number to a fifth part and a
sixth of the yellow herd. Finally the yellow were in number equal to
a sixth part and a seventh of the white herd. If thou canst accurately
tell, O stranger, the number of cattle of the Sun, giving separately the
number of well-fed bulls and again the number of females according
to each colour, thou wouldst not be called unskilled or ignorant of
numbers, but not yet shall thou be numbered among the wise. But
come, understand also all these conditions regarding the cows of the
Sun. When the white bulls mingled their number with the black, they
stood firm, equal in depth and breadth, and the plains of Thrinacia,
stretching far in all ways, were filled with their multitude. Again,
when the yellow and the dappled bulls were gathered into one herd
they stood in such a manner that their number, beginning from one,
grew slowly greater till it completed a triangular figure, there being
no bulls of other colours in their midst nor none of them lacking.
If thou art able, O stranger, to find out all these things and gather
them together in your mind, giving all the relations, thou shalt depart
crowned with glory and knowing that thou hast been adjudged perfect
in this species of wisdom.
\end{itemize}

If $W,X,Y,Z$ represents the number of white, black, yellow,
dappled bulls, you will get 
a systems of 7 linear equations, the first two being 
\begin{align*}
  W &= (1/2 + 1/3)X + Z \\
  X &= (1/4 + 1/5)Y + Z
\end{align*}
together with some constraints such as $W + X$ must be a square.
After some manipulations, it can be shown that the equation to solve looks like
\[
  x^2 - 410286423278424 y^2 = 1
\]
What was Archimedes thinking? You are find information on the Archimedes Cattle Problem on the web.]
}


  \solutionlink{sol:semigroup-associativity-0}
  \qed
\end{ex} 
\begin{python0}
from solutions import *
add(label="ex:semigroup-associativity-0",
    srcfilename='exercises/semigroup-associativity-0/answer.tex') 
\end{python0}


\begin{ex} 
  \label{ex:semigroup-associativity-0}
  \tinysidebar{\debug{exercises/{exercise-12/question.tex}}}
\mbox{}
  \begin{myenum}
  \item
    Solve
    \[
      5x^2 + y^2 = 3
    \]
    % mod 5, squares = 0^2=0, 1^2=1, 2^2=4, 3^2=4, 4^2=1
    (HINT: You don't really need number theory for this one. Why?
    But if you want to, imitate the solution for the previous
    problem.)
  \item
    Solve
    \[
      11y^2 - 5x^2 = 3
    \]
    % mod 5, squares = 0,1,4
    (HINT: This is just a slight change from the
    previous problem. \textit{But} now you need number theory. Try mod 4.
    If it does not work, try mod 5. Repeat.)
    % 3x^2 + y^2 = 3
    % {0,3} + {0,1} = 3
    % 0, 1, 3, 0 = 3
    % So x = 1, y = 0 (4)
    %
    % y^2=3 (5)
  \item
    Solve
    \[
      y^2 - 5x^2 = 2
    \]
    % mod 5, squares = 0^2=0, 1^2=1, 2^2=4, 3^2=4, 4^2=1    

  \item
    What about this one:
    \[
      x^2 - 5y^2 = 1
    \]    
  \end{myenum}
  {\scriptsize
[\textsc{Aside.}
Integer solutions to $x^2 - dy^2 = 1$ has been studied since at least 400BC.
This equation appear the Cattle Problem of Archimedes:

\begin{itemize}
  \item[]
If thou art diligent and wise, O stranger, compute the number of
cattle of the Sun, who once upon a time grazed on the fields of the
Thrinacian isle of Sicily, divided into four herds of different colours,
one milk white, another a glossy black, the third yellow and the last
dappled. In each herd were bulls, mighty in number according to these
proportions: Understand, stranger, that the white bulls were equal to
a half and a third of the black together with the whole of the yellow,
while the black were equal to the fourth part of the dappled and
a fifth, together with, once more, the whole of the yellow. Observe
further that the remaining bulls, the dappled, were equal to a sixth
part of the white and a seventh, together with all the yellow. These
were the proportions of the cows: The white were precisely equal to the
third part and a fourth of the whole herd of the black; while the black
were equal to the fourth part once more of the dappled and with it a
fifth part, when all, including the bulls went to pasture together. Now
the dappled in four parts8 were equal in number to a fifth part and a
sixth of the yellow herd. Finally the yellow were in number equal to
a sixth part and a seventh of the white herd. If thou canst accurately
tell, O stranger, the number of cattle of the Sun, giving separately the
number of well-fed bulls and again the number of females according
to each colour, thou wouldst not be called unskilled or ignorant of
numbers, but not yet shall thou be numbered among the wise. But
come, understand also all these conditions regarding the cows of the
Sun. When the white bulls mingled their number with the black, they
stood firm, equal in depth and breadth, and the plains of Thrinacia,
stretching far in all ways, were filled with their multitude. Again,
when the yellow and the dappled bulls were gathered into one herd
they stood in such a manner that their number, beginning from one,
grew slowly greater till it completed a triangular figure, there being
no bulls of other colours in their midst nor none of them lacking.
If thou art able, O stranger, to find out all these things and gather
them together in your mind, giving all the relations, thou shalt depart
crowned with glory and knowing that thou hast been adjudged perfect
in this species of wisdom.
\end{itemize}

If $W,X,Y,Z$ represents the number of white, black, yellow,
dappled bulls, you will get 
a systems of 7 linear equations, the first two being 
\begin{align*}
  W &= (1/2 + 1/3)X + Z \\
  X &= (1/4 + 1/5)Y + Z
\end{align*}
together with some constraints such as $W + X$ must be a square.
After some manipulations, it can be shown that the equation to solve looks like
\[
  x^2 - 410286423278424 y^2 = 1
\]
What was Archimedes thinking? You are find information on the Archimedes Cattle Problem on the web.]
}


  \solutionlink{sol:semigroup-associativity-0}
  \qed
\end{ex} 
\begin{python0}
from solutions import *
add(label="ex:semigroup-associativity-0",
    srcfilename='exercises/semigroup-associativity-0/answer.tex') 
\end{python0}


\begin{ex} 
  \label{ex:semigroup-associativity-0}
  \tinysidebar{\debug{exercises/{exercise-12/question.tex}}}
\mbox{}
  \begin{myenum}
  \item
    Solve
    \[
      5x^2 + y^2 = 3
    \]
    % mod 5, squares = 0^2=0, 1^2=1, 2^2=4, 3^2=4, 4^2=1
    (HINT: You don't really need number theory for this one. Why?
    But if you want to, imitate the solution for the previous
    problem.)
  \item
    Solve
    \[
      11y^2 - 5x^2 = 3
    \]
    % mod 5, squares = 0,1,4
    (HINT: This is just a slight change from the
    previous problem. \textit{But} now you need number theory. Try mod 4.
    If it does not work, try mod 5. Repeat.)
    % 3x^2 + y^2 = 3
    % {0,3} + {0,1} = 3
    % 0, 1, 3, 0 = 3
    % So x = 1, y = 0 (4)
    %
    % y^2=3 (5)
  \item
    Solve
    \[
      y^2 - 5x^2 = 2
    \]
    % mod 5, squares = 0^2=0, 1^2=1, 2^2=4, 3^2=4, 4^2=1    

  \item
    What about this one:
    \[
      x^2 - 5y^2 = 1
    \]    
  \end{myenum}
  {\scriptsize
[\textsc{Aside.}
Integer solutions to $x^2 - dy^2 = 1$ has been studied since at least 400BC.
This equation appear the Cattle Problem of Archimedes:

\begin{itemize}
  \item[]
If thou art diligent and wise, O stranger, compute the number of
cattle of the Sun, who once upon a time grazed on the fields of the
Thrinacian isle of Sicily, divided into four herds of different colours,
one milk white, another a glossy black, the third yellow and the last
dappled. In each herd were bulls, mighty in number according to these
proportions: Understand, stranger, that the white bulls were equal to
a half and a third of the black together with the whole of the yellow,
while the black were equal to the fourth part of the dappled and
a fifth, together with, once more, the whole of the yellow. Observe
further that the remaining bulls, the dappled, were equal to a sixth
part of the white and a seventh, together with all the yellow. These
were the proportions of the cows: The white were precisely equal to the
third part and a fourth of the whole herd of the black; while the black
were equal to the fourth part once more of the dappled and with it a
fifth part, when all, including the bulls went to pasture together. Now
the dappled in four parts8 were equal in number to a fifth part and a
sixth of the yellow herd. Finally the yellow were in number equal to
a sixth part and a seventh of the white herd. If thou canst accurately
tell, O stranger, the number of cattle of the Sun, giving separately the
number of well-fed bulls and again the number of females according
to each colour, thou wouldst not be called unskilled or ignorant of
numbers, but not yet shall thou be numbered among the wise. But
come, understand also all these conditions regarding the cows of the
Sun. When the white bulls mingled their number with the black, they
stood firm, equal in depth and breadth, and the plains of Thrinacia,
stretching far in all ways, were filled with their multitude. Again,
when the yellow and the dappled bulls were gathered into one herd
they stood in such a manner that their number, beginning from one,
grew slowly greater till it completed a triangular figure, there being
no bulls of other colours in their midst nor none of them lacking.
If thou art able, O stranger, to find out all these things and gather
them together in your mind, giving all the relations, thou shalt depart
crowned with glory and knowing that thou hast been adjudged perfect
in this species of wisdom.
\end{itemize}

If $W,X,Y,Z$ represents the number of white, black, yellow,
dappled bulls, you will get 
a systems of 7 linear equations, the first two being 
\begin{align*}
  W &= (1/2 + 1/3)X + Z \\
  X &= (1/4 + 1/5)Y + Z
\end{align*}
together with some constraints such as $W + X$ must be a square.
After some manipulations, it can be shown that the equation to solve looks like
\[
  x^2 - 410286423278424 y^2 = 1
\]
What was Archimedes thinking? You are find information on the Archimedes Cattle Problem on the web.]
}


  \solutionlink{sol:semigroup-associativity-0}
  \qed
\end{ex} 
\begin{python0}
from solutions import *
add(label="ex:semigroup-associativity-0",
    srcfilename='exercises/semigroup-associativity-0/answer.tex') 
\end{python0}


\begin{ex} 
  \label{ex:semigroup-associativity-0}
  \tinysidebar{\debug{exercises/{exercise-12/question.tex}}}
\mbox{}
  \begin{myenum}
  \item
    Solve
    \[
      5x^2 + y^2 = 3
    \]
    % mod 5, squares = 0^2=0, 1^2=1, 2^2=4, 3^2=4, 4^2=1
    (HINT: You don't really need number theory for this one. Why?
    But if you want to, imitate the solution for the previous
    problem.)
  \item
    Solve
    \[
      11y^2 - 5x^2 = 3
    \]
    % mod 5, squares = 0,1,4
    (HINT: This is just a slight change from the
    previous problem. \textit{But} now you need number theory. Try mod 4.
    If it does not work, try mod 5. Repeat.)
    % 3x^2 + y^2 = 3
    % {0,3} + {0,1} = 3
    % 0, 1, 3, 0 = 3
    % So x = 1, y = 0 (4)
    %
    % y^2=3 (5)
  \item
    Solve
    \[
      y^2 - 5x^2 = 2
    \]
    % mod 5, squares = 0^2=0, 1^2=1, 2^2=4, 3^2=4, 4^2=1    

  \item
    What about this one:
    \[
      x^2 - 5y^2 = 1
    \]    
  \end{myenum}
  {\scriptsize
[\textsc{Aside.}
Integer solutions to $x^2 - dy^2 = 1$ has been studied since at least 400BC.
This equation appear the Cattle Problem of Archimedes:

\begin{itemize}
  \item[]
If thou art diligent and wise, O stranger, compute the number of
cattle of the Sun, who once upon a time grazed on the fields of the
Thrinacian isle of Sicily, divided into four herds of different colours,
one milk white, another a glossy black, the third yellow and the last
dappled. In each herd were bulls, mighty in number according to these
proportions: Understand, stranger, that the white bulls were equal to
a half and a third of the black together with the whole of the yellow,
while the black were equal to the fourth part of the dappled and
a fifth, together with, once more, the whole of the yellow. Observe
further that the remaining bulls, the dappled, were equal to a sixth
part of the white and a seventh, together with all the yellow. These
were the proportions of the cows: The white were precisely equal to the
third part and a fourth of the whole herd of the black; while the black
were equal to the fourth part once more of the dappled and with it a
fifth part, when all, including the bulls went to pasture together. Now
the dappled in four parts8 were equal in number to a fifth part and a
sixth of the yellow herd. Finally the yellow were in number equal to
a sixth part and a seventh of the white herd. If thou canst accurately
tell, O stranger, the number of cattle of the Sun, giving separately the
number of well-fed bulls and again the number of females according
to each colour, thou wouldst not be called unskilled or ignorant of
numbers, but not yet shall thou be numbered among the wise. But
come, understand also all these conditions regarding the cows of the
Sun. When the white bulls mingled their number with the black, they
stood firm, equal in depth and breadth, and the plains of Thrinacia,
stretching far in all ways, were filled with their multitude. Again,
when the yellow and the dappled bulls were gathered into one herd
they stood in such a manner that their number, beginning from one,
grew slowly greater till it completed a triangular figure, there being
no bulls of other colours in their midst nor none of them lacking.
If thou art able, O stranger, to find out all these things and gather
them together in your mind, giving all the relations, thou shalt depart
crowned with glory and knowing that thou hast been adjudged perfect
in this species of wisdom.
\end{itemize}

If $W,X,Y,Z$ represents the number of white, black, yellow,
dappled bulls, you will get 
a systems of 7 linear equations, the first two being 
\begin{align*}
  W &= (1/2 + 1/3)X + Z \\
  X &= (1/4 + 1/5)Y + Z
\end{align*}
together with some constraints such as $W + X$ must be a square.
After some manipulations, it can be shown that the equation to solve looks like
\[
  x^2 - 410286423278424 y^2 = 1
\]
What was Archimedes thinking? You are find information on the Archimedes Cattle Problem on the web.]
}


  \solutionlink{sol:semigroup-associativity-0}
  \qed
\end{ex} 
\begin{python0}
from solutions import *
add(label="ex:semigroup-associativity-0",
    srcfilename='exercises/semigroup-associativity-0/answer.tex') 
\end{python0}


Collecting up all the information above,
formally, the Hill's cipher of block size $n$ (just think of $n = 2$)
is defined as follows:
Let
\[
P = C = \{0,...,25\}^n = (\Z/26)^n
\]
and
\[
K = \operatorname{GL}_n(\Z/26)
\]
The encryption function $E$
\[
E : K \times P \rightarrow C
\]
is defined to be
\[
E(M, x) = Mx
\]
and
\[
D : K \times C \rightarrow P
\]
is defined to be
\[
D(M, x) = M^{-1}x
\]
And $(E,D)$ is a cipher because
\[
D(M, E(M, x)) = D(M, Mx) = M^{-1}Mx = Ix = x
\]


\begin{ex} 
  \label{ex:semigroup-associativity-0}
  \tinysidebar{\debug{exercises/{exercise-12/question.tex}}}
\mbox{}
  \begin{myenum}
  \item
    Solve
    \[
      5x^2 + y^2 = 3
    \]
    % mod 5, squares = 0^2=0, 1^2=1, 2^2=4, 3^2=4, 4^2=1
    (HINT: You don't really need number theory for this one. Why?
    But if you want to, imitate the solution for the previous
    problem.)
  \item
    Solve
    \[
      11y^2 - 5x^2 = 3
    \]
    % mod 5, squares = 0,1,4
    (HINT: This is just a slight change from the
    previous problem. \textit{But} now you need number theory. Try mod 4.
    If it does not work, try mod 5. Repeat.)
    % 3x^2 + y^2 = 3
    % {0,3} + {0,1} = 3
    % 0, 1, 3, 0 = 3
    % So x = 1, y = 0 (4)
    %
    % y^2=3 (5)
  \item
    Solve
    \[
      y^2 - 5x^2 = 2
    \]
    % mod 5, squares = 0^2=0, 1^2=1, 2^2=4, 3^2=4, 4^2=1    

  \item
    What about this one:
    \[
      x^2 - 5y^2 = 1
    \]    
  \end{myenum}
  {\scriptsize
[\textsc{Aside.}
Integer solutions to $x^2 - dy^2 = 1$ has been studied since at least 400BC.
This equation appear the Cattle Problem of Archimedes:

\begin{itemize}
  \item[]
If thou art diligent and wise, O stranger, compute the number of
cattle of the Sun, who once upon a time grazed on the fields of the
Thrinacian isle of Sicily, divided into four herds of different colours,
one milk white, another a glossy black, the third yellow and the last
dappled. In each herd were bulls, mighty in number according to these
proportions: Understand, stranger, that the white bulls were equal to
a half and a third of the black together with the whole of the yellow,
while the black were equal to the fourth part of the dappled and
a fifth, together with, once more, the whole of the yellow. Observe
further that the remaining bulls, the dappled, were equal to a sixth
part of the white and a seventh, together with all the yellow. These
were the proportions of the cows: The white were precisely equal to the
third part and a fourth of the whole herd of the black; while the black
were equal to the fourth part once more of the dappled and with it a
fifth part, when all, including the bulls went to pasture together. Now
the dappled in four parts8 were equal in number to a fifth part and a
sixth of the yellow herd. Finally the yellow were in number equal to
a sixth part and a seventh of the white herd. If thou canst accurately
tell, O stranger, the number of cattle of the Sun, giving separately the
number of well-fed bulls and again the number of females according
to each colour, thou wouldst not be called unskilled or ignorant of
numbers, but not yet shall thou be numbered among the wise. But
come, understand also all these conditions regarding the cows of the
Sun. When the white bulls mingled their number with the black, they
stood firm, equal in depth and breadth, and the plains of Thrinacia,
stretching far in all ways, were filled with their multitude. Again,
when the yellow and the dappled bulls were gathered into one herd
they stood in such a manner that their number, beginning from one,
grew slowly greater till it completed a triangular figure, there being
no bulls of other colours in their midst nor none of them lacking.
If thou art able, O stranger, to find out all these things and gather
them together in your mind, giving all the relations, thou shalt depart
crowned with glory and knowing that thou hast been adjudged perfect
in this species of wisdom.
\end{itemize}

If $W,X,Y,Z$ represents the number of white, black, yellow,
dappled bulls, you will get 
a systems of 7 linear equations, the first two being 
\begin{align*}
  W &= (1/2 + 1/3)X + Z \\
  X &= (1/4 + 1/5)Y + Z
\end{align*}
together with some constraints such as $W + X$ must be a square.
After some manipulations, it can be shown that the equation to solve looks like
\[
  x^2 - 410286423278424 y^2 = 1
\]
What was Archimedes thinking? You are find information on the Archimedes Cattle Problem on the web.]
}


  \solutionlink{sol:semigroup-associativity-0}
  \qed
\end{ex} 
\begin{python0}
from solutions import *
add(label="ex:semigroup-associativity-0",
    srcfilename='exercises/semigroup-associativity-0/answer.tex') 
\end{python0}


\newpage
\textsc{Breaking Hill's cipher}

Now how do you break the Hill's cipher?

Let's stick to a 2-by-2 key.
Suppose you have the following:
you have the two plaintexts
$x_1 = (A, B)$ and $x_2 = (C, D)$
and their encryptions
$y_1 = (A', B')$ and $y_2 = (C', D')$.
Therefore
\[
\begin{pmatrix}
  a & b \\
  c & d 
\end{pmatrix}
\begin{pmatrix}
  A \\
  B 
\end{pmatrix}
=
\begin{pmatrix}
  aA + bB \\
  cA + dB 
\end{pmatrix}
=
\begin{pmatrix}
  A' \\
  B' 
\end{pmatrix}
\]
and
\[
\begin{pmatrix}
  a & b \\
  c & d 
\end{pmatrix}
\begin{pmatrix}
  C \\
  D 
\end{pmatrix}
=
\begin{pmatrix}
  aC + bD \\
  cC + dD 
\end{pmatrix}
=
\begin{pmatrix}
  A' \\
  B' 
\end{pmatrix}
\]
In the above, the unknowns are $a,b,c,d$, i.e., the key.
These two equations can be combined to get an equation involving
the product of 2-by-2 matrices:
\[
\begin{pmatrix}
  a & b \\
  c & d 
\end{pmatrix}
\begin{pmatrix}
  A & C \\
  B & D 
\end{pmatrix}
=
\begin{pmatrix}
  aA + bB & aC + bD \\
  cA + dB & cC + dD \\
\end{pmatrix}
=
\begin{pmatrix}
  A' & C' \\
  B' & D'
\end{pmatrix}
\]
You now have this matrix equation
\[
\begin{pmatrix}
  a & b \\
  c & d 
\end{pmatrix}
\begin{pmatrix}
  A & C \\
  B & D 
\end{pmatrix}
=
\begin{pmatrix}
  A' & C' \\
  B' & D'
\end{pmatrix}
\]
Note that you have the values of $A,B,C,D,A',B',C',D'$.
The unknowns are $a,b,c,d$.
What do you do?
You multiply on the right by the inverse matrix of
$\begin{pmatrix}
  A & C \\
  B & D 
\end{pmatrix}
$
to get
\[
\begin{pmatrix}
  a & b \\
  c & d 
\end{pmatrix}
\begin{pmatrix}
  A & C \\
  B & D 
\end{pmatrix}
\begin{pmatrix}
  A & C \\
  B & D 
\end{pmatrix}^{-1}
=
\begin{pmatrix}
  A' & C' \\
  B' & D'
\end{pmatrix}
\begin{pmatrix}
  A & C \\
  B & D 
\end{pmatrix}^{-1}
\]
which gives you
\[
\begin{pmatrix}
  a & b \\
  c & d 
\end{pmatrix}
I
=
\begin{pmatrix}
  A' & C' \\
  B' & D'
\end{pmatrix}
\begin{pmatrix}
  A & C \\
  B & D 
\end{pmatrix}^{-1}
\]
which finally gives you the key
\[
\begin{pmatrix}
  a & b \\
  c & d 
\end{pmatrix}
=
\begin{pmatrix}
  A' & C' \\
  B' & D'
\end{pmatrix}
\begin{pmatrix}
  A & C \\
  B & D 
\end{pmatrix}^{-1}
\]
There's an assumption above:
we have to assume that
$
\begin{pmatrix}
  A & C \\
  B & D 
\end{pmatrix}
$
is invertible.
It's possible the first 4 characters of the original message
does not form an invertible matrix.
But if you have enough pairs of plaintext and corresponding ciphertext blocks,
you hope to be able to get an invertible matrix and perform
the above computation.
Therefore we are assuming the known plaintext attack model.
Or we are assuming the chosen plaintext attack model.
