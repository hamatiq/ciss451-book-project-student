\tinysidebar{\debug{exercises/{introduction-to-group-theory-3/answer.tex}}}

  \proof
  (a)
  Let $x \in G$.
  $x^1, x^2, x^3, ...$ are in $G$ (by the closure axiom).
  Since $G$ is finite $x^1, x^2, x^3, ...$ cannot all be pairwise distinct. 
  Suppose $x^i = x^j$ for $i < j$.
  Then $e = x^j (x^{i})^{-1} = x^{j} (x^{-i}) = x^{j - i}$ and $j - i > 0$.
  Hence $x$ has finite order.


  (b)
  Let $G = \{(x_1, x_2, ...) \mid x_i = 0, 1 \text{ for } i = 1, 2, ...\}$
  and
  \[
  (x_1, x_2, ...) * (y_1, y_2, ...)
  =
  (
  x_1 + y_1 \pmod{2},
  x_2 + y_2 \pmod{2},
  x_3 + y_3 \pmod{2},
  ...)
  \]
  It's not difficult to show that $G$ is a group.
  The neutral element is (0, 0, 0, ...).
  Every non-neutral element has order 2.
  (Also, see section on product of groups.)
  
  (c) $\Z$.     
  \qed
