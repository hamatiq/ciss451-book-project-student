\tinysidebar{\debug{exercises/{exercise-3/question.tex}}}
  Just like for boolean values, you can write down the complete
  behavior of the boolean and and boolean or and boolean not operators (these are called truth tables),
  you can also complete specify the complete behavior of
  addition in mod 26, \lq\lq negative of'' in mod 26,
  multiplication in mod 26, and also multiplicative inverse mod 26.
  The multiplicative inverse of $x$ in mod 26 is just the number $y$ mod 26
  such that
  \[
    xy \equiv 1 \pmod{26}
  \]
  The multiplicative inverse of $x \pmod{26}$ is written
  $x^{-1} \pmod{26}$ -- this is an integer mod 26!!! It's not a fraction in $\R$!!!
  Sometimes $x^{-1} \pmod{26}$ might not exist. In that case write None.
  Write down these 4 tables.

Addition table for $\Z/26$:
\begin{python}
from latextool_basic import *
p = Plot()
m00 = [['$+$']]
m10 = [['%s' % i] for i in range(26)]
m01 = [['%s' % i for i in range(26)],
      ]
m11 =[['' for i in range(26)] for j in range(26)
     ]
M = [[m00, m01],
[m10, m11]]
N = table3(p, M, width=0.6, height=0.6)
print(p)
\end{python}
In the above, when I write $5$, I meant of course $5 \pmod{26}$.


Multiplication table for $\Z/26$:
\begin{python}
from latextool_basic import *
p = Plot()
m00 = [[r'$\times$']]
m10 = [['%s' % i] for i in range(26)]
m01 = [['%s' % i for i in range(26)],
      ]
m11 =[['' for i in range(26)] for j in range(26)
     ]
M = [[m00, m01],
[m10, m11]]
N = table3(p, M, width=0.6, height=0.6)
print(p)
\end{python}


Negative of table for $\Z/26$:
\begin{python}
from latextool_basic import *
p = Plot()
m00 = [['$x \pmod{26}$']]
m10 = [['%s' % i] for i in range(26)]
m01 = [['$-x \pmod{26}$']]
m11 = [[''] for i in range(26)]
M = [[m00, m01],
     [m10, m11]]
N = table3(p, M, width=3, height=0.6)
print(p)
\end{python}


Multiplicative inverse table for $\Z/26$:
\begin{python}
from latextool_basic import *
p = Plot()
m00 = [['$x \pmod{26}$']]
m10 = [['%s' % i] for i in range(26)]
m01 = [['$x^{-1} \pmod{26}$']]
m11 = [['None']] + [[''] for i in range(25)]
M = [[m00, m01],
     [m10, m11]]
N = table3(p, M, width=3, height=0.6)
print(p)
\end{python}
It should be clear that $0 \pmod{26}$ does not have an inverse.

    
