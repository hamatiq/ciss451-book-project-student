\tinysidebar{\debug{exercises/{exercise-24/question.tex}}}
\mbox{}
  \begin{myenum}
    \item
      Solve the following
      \begin{align*}
        9x + 5y \equiv 3 \pmod{26} \\
        5x + 7y \equiv 1 \pmod{26}
      \end{align*}
      First solving by writing a program that performs a brute force search for solutions.
      Next, try to solve it algebraically by hand.
    \item What about this one:
      \begin{align*}
        3x - y &\equiv 2 \pmod{26} \\
        2x + 19y &\equiv 14 \pmod{26}
      \end{align*}
    \item And this one:
      \begin{align*}
        9x + y &\equiv 2 \pmod{26} \\
        19x + 5y &\equiv 7 \pmod{26}
      \end{align*}

    \item Write a program that solves
      \begin{align*}
        ax + b &\equiv c \pmod{26} \\
        dx + ey &\equiv f \pmod{26}
      \end{align*}
      for $a,b,c,d,e,f$ in $\Z/26$ by brute force search.
      Then write a program that randomly
      picks $a,b,c,d,e,f$ in $\Z/26$ and ask you to solve it.
      Print all the cases where $a,b,c,d,e,f$ provides a linear system of
      two equations or two unknowns where there is no solution.
      Do you notice a pattern in these degenerate cases?
\end{myenum}
