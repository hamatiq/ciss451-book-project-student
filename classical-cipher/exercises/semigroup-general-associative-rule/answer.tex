\tinysidebar{\debug{exercises/{semigroup-general-associative-rule/answer.tex}}}

Let $e$ be a fully parenthesized expression of $x_1, ..., x_n$.
Then (of course) there has to be a last $*$ to be evaluated in $e$ if
$e$ has more than one term.
Suppose the two subexpression for this last $*$ are $e_1, e_2$, i.e.,
\[
e = e_1 * e_2
\]
then clearly $e_1, e_2$ will both involve $< n$ terms from $\{x_1, ..., x_n\}$.
Therefore we should use induction so that some induction hypothesis can be applied to
$e_1$ and $e_2$.
Note that instead of proving all fully parenthesized expressions evaluate to the
same value, we can fix one preferred expression and prove that
all fully parenthesized expressions evaluate to this preferred expression.
For instance I can choose the left-associative expression
\[
\prod_{i=1}^n x_i = ( \cdots ((x_1 * x_2) * x_3) * \cdots * x_n)
\]
Then for
\[
e = e_1 * e_2
\]
I want to say $e$ evaluates to the same value as the left associative form
$\prod_{i=1}^n x_i$.
$e_1$ and $e_2$ are shorter than $e$.
Therefore with a suitable induction $e_1$ and $e_2$ are in the left associative
form.
Now what?
Well I need to reparenthesize $e_1 * e_2$ to get \textit{one} left associative
form for $e$.
I don't want two.
If $e_2$ has only one term, then $e_1 * e_2$ is already in the left associative
form from the recursive definition of $\prod$.
But what if not?
For instance say
\[
e_1 * e_2 = ((x_1 * x_2) * x_3) * ((x_4 * x_5) * x_6)
\]
Aha ... since $e_2$ is in the left associative form,
$e_2 = e_2' * x_6$
where $e_2'$ is in the left associative form (but the fact that it's
in the left associative form is not important -- the important
thing is $e_2'$ must have at least one term.)
Then I have
\[
e_1 * e_2 = A * (B * x_6)
\]
where $A = (x_1 * x_2) * x_3$ and $B = x_4 * x_5$.
By the associative axiom,
\[
e_1 * e_2 = (A * B) * x_6
\]
By induction magic, $A*B$ has $n - 1$ terms,
which means that it evaluates to the same value as the left
associate form for $x_1 * x_2 * x_3 * x_4 * x_5$.
And as above adding $x_6$ to this is also in the left associate form.
Done!
Now I'm ready to write the proof properly.

Note the technique in the proof strategy above.
I could have proven that given \textit{two} fully parenthesized expressions,
I can slowly adjust the parenthesization from one to match the other.
But I prefer to compare \textit{one} fully parenthesized to a \lq\lq standard"
one (the left associative version).
This is a very common technique.
For instance many fraction have the same value: $1/2 = 2/4 = 3/6 = \ldots$.
In the proof of the irrationality of $\sqrt{2}$, we prove this by
contradiction that if $\sqrt{2}$ is a fraction $m/n$, something will go wrong.
And we deliberately chose $m/n$ to be \textit{in reduced form}.

\proof
Let $(G, *)$ be a semigroup. Let $P(n)$ be the proposition
\[
P(n) = \left[
\text{a fully parenthesized product of } x_1, x_2, ..., x_n \in G
\text{ has the same value as }
\prod_{i = 1}^n x_i
\right]
\]
We will prove $P(n)$ holds for all $n \geq 1$ by strong induction.
($P(0)$ is actually also true since the product of an empty collection
of group elements is defined to be $e$.)

\textsc{Base case}. $P(1)$ is clearly true.

\textsc{Inductive case}.
Assume $n \geq 1$ and $P(k)$ holds for $1 \leq k \leq n$.
We want to prove that $P(n + 1)$ holds.
Let $e$ be some fully parenthesized product of $x_1, \ldots x_{n + 1}$
(in that order).
Then
\[
e = e_1 * e_2
\]
where
$e_1$ is a fully parenthesized product of $x_1, \ldots, x_k$
and
$e_2$ is a fully parenthesized product of $x_{k+1}, \ldots x_{n+1}$
where $1 \leq k \leq n$.
Note that each of $e_1$ and $e_2$ contains $< n + 1$ group elements.
($e_2$ has $n + 1 - k$ terms.
From $k \leq n$, we have $0 \leq n - k$ and hence $1 \leq n - k + 1$.
From $k \geq 1$, $n - k + 1 \leq n - 1 + 1 = n$.
Hence $1 \leq n - k + 1 \leq n$.)
By induction hypothesis, 
\begin{align*}
e_1 = \prod_{i = 1}^k x_i, \,\,\,\,\, e_2 = \prod_{i = k + 1}^{n + 1} x_i
\end{align*}

\textsc{Case 1:} $e_2$ contains one term.
In this case
\begin{align*}
e_1 = \prod_{i = 1}^n x_i, \,\,\,\,\, e_2 = \prod_{i = n + 1}^{n + 1} x_i = x_{n+1}
\end{align*}
Hence
\[
e = e_1*e_2 = \left( \prod_{i = 1}^n x_i \right) * x_{n+1} = \prod_{i = 1}^{n+1} x_i
\]

\textsc{Case 2:} $e_2$ contains more than two terms.
In this case
\[
e_2 = \left(\prod_{i = k + 1}^{n} x_i\right) * x_{n + 1} = e_2' * x_{n + 1}
\]
where
\[
e_2' = \prod_{i = k + 1}^{n} x_i
\]
Then
\begin{align*}
e
&= e_1 * e_2 \\
&= e_1 * (e_2' * x_{n + 1}) \\
&= (e_1 * e_2') * x_{n + 1} & & \text{ by associative axiom}\\
\end{align*}
Note that $e_1 * e_2'$ is a product of $n$ terms.
Therefore by $P(n)$
\[
e_1 * e_2' = \prod_{i=1}^n x_i
\]
Hence
\begin{align*}
e
&= \left( \prod_{i=1}^n x_i  \right) * x_{n + 1} \\
&= \prod_{i=1}^{n+1} x_i
\end{align*}
Hence $P(n+1)$ in both cases

Therefore by strong induction $P(n)$ holds for all $n \geq 1$.
Hence any fully parenthesized product of $x_1, x_2, ..., x_n \in G$
has the same value as $\prod_{i = 1}^n x_i$.
\qed
