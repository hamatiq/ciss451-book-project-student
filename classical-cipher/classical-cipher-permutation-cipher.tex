\sectionthree{Permutation cipher}
\begin{python0}
from solutions import *; clear()
\end{python0}

In the above ciphers, a character is replaced by another.
The
\defone{permutation cipher}
is different:
each character is \textit{moved} to a different
\textit{location} in the plaintext
to form the ciphertext.

Here's an example.
Look at this:
\[
  \pi =
  \begin{pmatrix}
    1 & 2 & 3 & 4 & 5 & 6 \\
    3 & 1 & 2 & 5 & 4 & 6 \\
  \end{pmatrix}
  \]
  It's just the function where a value in the top row maps to the
  corresponding value at the bottom row: 
  \begin{align*}
    \pi(1) &= 3 \\
    \pi(2) &= 1 \\
    \pi(3) &= 2 \\
    \vdots &\textwhite{=} \ \vdots \\
    \pi(6) &= 6 \\
  \end{align*}
Let $\pi$ is called a
permutation
of $\{1,2,3,4,5,6\}$.
A
\defone{permutation}
is simply a bijection (1--1 and onto) function
from
$\{1, 2, 3, ..., n\}$
to
$\{1, 2, 3, ..., n\}$.
As a shorthand, the above permutation
\[
  \pi =
  \begin{pmatrix}
    1 & 2 & 3 & 4 & 5 & 6 \\
    3 & 1 & 2 & 5 & 4 & 6 \\
  \end{pmatrix}
\]
is also written as
\[
  \pi = (1 \ \ 3 \ \ 2)(4 \ \ 5)(6)
\]
This is called the
\defone{cycle notation}
of $\pi$.
In the above example, we say that the permutation $\pi$ has
\defone{length}
6.


\begin{ex} 
  \label{ex:semigroup-associativity-0}
  \tinysidebar{\debug{exercises/{exercise-12/question.tex}}}
\mbox{}
  \begin{myenum}
  \item
    Solve
    \[
      5x^2 + y^2 = 3
    \]
    % mod 5, squares = 0^2=0, 1^2=1, 2^2=4, 3^2=4, 4^2=1
    (HINT: You don't really need number theory for this one. Why?
    But if you want to, imitate the solution for the previous
    problem.)
  \item
    Solve
    \[
      11y^2 - 5x^2 = 3
    \]
    % mod 5, squares = 0,1,4
    (HINT: This is just a slight change from the
    previous problem. \textit{But} now you need number theory. Try mod 4.
    If it does not work, try mod 5. Repeat.)
    % 3x^2 + y^2 = 3
    % {0,3} + {0,1} = 3
    % 0, 1, 3, 0 = 3
    % So x = 1, y = 0 (4)
    %
    % y^2=3 (5)
  \item
    Solve
    \[
      y^2 - 5x^2 = 2
    \]
    % mod 5, squares = 0^2=0, 1^2=1, 2^2=4, 3^2=4, 4^2=1    

  \item
    What about this one:
    \[
      x^2 - 5y^2 = 1
    \]    
  \end{myenum}
  {\scriptsize
[\textsc{Aside.}
Integer solutions to $x^2 - dy^2 = 1$ has been studied since at least 400BC.
This equation appear the Cattle Problem of Archimedes:

\begin{itemize}
  \item[]
If thou art diligent and wise, O stranger, compute the number of
cattle of the Sun, who once upon a time grazed on the fields of the
Thrinacian isle of Sicily, divided into four herds of different colours,
one milk white, another a glossy black, the third yellow and the last
dappled. In each herd were bulls, mighty in number according to these
proportions: Understand, stranger, that the white bulls were equal to
a half and a third of the black together with the whole of the yellow,
while the black were equal to the fourth part of the dappled and
a fifth, together with, once more, the whole of the yellow. Observe
further that the remaining bulls, the dappled, were equal to a sixth
part of the white and a seventh, together with all the yellow. These
were the proportions of the cows: The white were precisely equal to the
third part and a fourth of the whole herd of the black; while the black
were equal to the fourth part once more of the dappled and with it a
fifth part, when all, including the bulls went to pasture together. Now
the dappled in four parts8 were equal in number to a fifth part and a
sixth of the yellow herd. Finally the yellow were in number equal to
a sixth part and a seventh of the white herd. If thou canst accurately
tell, O stranger, the number of cattle of the Sun, giving separately the
number of well-fed bulls and again the number of females according
to each colour, thou wouldst not be called unskilled or ignorant of
numbers, but not yet shall thou be numbered among the wise. But
come, understand also all these conditions regarding the cows of the
Sun. When the white bulls mingled their number with the black, they
stood firm, equal in depth and breadth, and the plains of Thrinacia,
stretching far in all ways, were filled with their multitude. Again,
when the yellow and the dappled bulls were gathered into one herd
they stood in such a manner that their number, beginning from one,
grew slowly greater till it completed a triangular figure, there being
no bulls of other colours in their midst nor none of them lacking.
If thou art able, O stranger, to find out all these things and gather
them together in your mind, giving all the relations, thou shalt depart
crowned with glory and knowing that thou hast been adjudged perfect
in this species of wisdom.
\end{itemize}

If $W,X,Y,Z$ represents the number of white, black, yellow,
dappled bulls, you will get 
a systems of 7 linear equations, the first two being 
\begin{align*}
  W &= (1/2 + 1/3)X + Z \\
  X &= (1/4 + 1/5)Y + Z
\end{align*}
together with some constraints such as $W + X$ must be a square.
After some manipulations, it can be shown that the equation to solve looks like
\[
  x^2 - 410286423278424 y^2 = 1
\]
What was Archimedes thinking? You are find information on the Archimedes Cattle Problem on the web.]
}


  \solutionlink{sol:semigroup-associativity-0}
  \qed
\end{ex} 
\begin{python0}
from solutions import *
add(label="ex:semigroup-associativity-0",
    srcfilename='exercises/semigroup-associativity-0/answer.tex') 
\end{python0}


How does a permutation give you an encryption?
For the permutation $\pi = (1 \ \ 3 \ \ 2)(4 \ \ 5)(6)$,
you do this:
\begin{enumerate}[nosep]
  \item
    character at position 1 goes to position 3,
  \item
    character at position 2 goes to position 1,
  \item etc.
\end{enumerate}
In the example above, the $n$ is 6.
For instance for the string \texttt{marvin}
and the permutation $\pi = (1 \ \ 3 \ \ 2)(4 \ \ 5)(6)$,
the since in the permutation $1 \mapsto 3$, the character at
position 1 goes to position 3:
\begin{smallconsole}
    1 2 3 4 5 6
    m a r v i n
        m
\end{smallconsole}
since $3 \mapsto 2$, the character at
position 3 goes to position 2:
\begin{smallconsole}
    1 2 3 4 5 6
    m a r v i n
      r m
\end{smallconsole}
Etc.
Altogether I get
\[
E(\pi, \texttt{marvin}) = \texttt{armivn}
\]
This assumes your string has a length which is a multiple of 6.
So you might need to add some dummy data at the end of your string.


Note that in the above example, you have to encrypt substrings of length 5.
Therefore if the original plaintext does not have a length that is a
multiple of 5, you would need to pad it with some characters until the
length is a multiple of 5.


\begin{ex} 
  \label{ex:semigroup-associativity-0}
  \tinysidebar{\debug{exercises/{exercise-12/question.tex}}}
\mbox{}
  \begin{myenum}
  \item
    Solve
    \[
      5x^2 + y^2 = 3
    \]
    % mod 5, squares = 0^2=0, 1^2=1, 2^2=4, 3^2=4, 4^2=1
    (HINT: You don't really need number theory for this one. Why?
    But if you want to, imitate the solution for the previous
    problem.)
  \item
    Solve
    \[
      11y^2 - 5x^2 = 3
    \]
    % mod 5, squares = 0,1,4
    (HINT: This is just a slight change from the
    previous problem. \textit{But} now you need number theory. Try mod 4.
    If it does not work, try mod 5. Repeat.)
    % 3x^2 + y^2 = 3
    % {0,3} + {0,1} = 3
    % 0, 1, 3, 0 = 3
    % So x = 1, y = 0 (4)
    %
    % y^2=3 (5)
  \item
    Solve
    \[
      y^2 - 5x^2 = 2
    \]
    % mod 5, squares = 0^2=0, 1^2=1, 2^2=4, 3^2=4, 4^2=1    

  \item
    What about this one:
    \[
      x^2 - 5y^2 = 1
    \]    
  \end{myenum}
  {\scriptsize
[\textsc{Aside.}
Integer solutions to $x^2 - dy^2 = 1$ has been studied since at least 400BC.
This equation appear the Cattle Problem of Archimedes:

\begin{itemize}
  \item[]
If thou art diligent and wise, O stranger, compute the number of
cattle of the Sun, who once upon a time grazed on the fields of the
Thrinacian isle of Sicily, divided into four herds of different colours,
one milk white, another a glossy black, the third yellow and the last
dappled. In each herd were bulls, mighty in number according to these
proportions: Understand, stranger, that the white bulls were equal to
a half and a third of the black together with the whole of the yellow,
while the black were equal to the fourth part of the dappled and
a fifth, together with, once more, the whole of the yellow. Observe
further that the remaining bulls, the dappled, were equal to a sixth
part of the white and a seventh, together with all the yellow. These
were the proportions of the cows: The white were precisely equal to the
third part and a fourth of the whole herd of the black; while the black
were equal to the fourth part once more of the dappled and with it a
fifth part, when all, including the bulls went to pasture together. Now
the dappled in four parts8 were equal in number to a fifth part and a
sixth of the yellow herd. Finally the yellow were in number equal to
a sixth part and a seventh of the white herd. If thou canst accurately
tell, O stranger, the number of cattle of the Sun, giving separately the
number of well-fed bulls and again the number of females according
to each colour, thou wouldst not be called unskilled or ignorant of
numbers, but not yet shall thou be numbered among the wise. But
come, understand also all these conditions regarding the cows of the
Sun. When the white bulls mingled their number with the black, they
stood firm, equal in depth and breadth, and the plains of Thrinacia,
stretching far in all ways, were filled with their multitude. Again,
when the yellow and the dappled bulls were gathered into one herd
they stood in such a manner that their number, beginning from one,
grew slowly greater till it completed a triangular figure, there being
no bulls of other colours in their midst nor none of them lacking.
If thou art able, O stranger, to find out all these things and gather
them together in your mind, giving all the relations, thou shalt depart
crowned with glory and knowing that thou hast been adjudged perfect
in this species of wisdom.
\end{itemize}

If $W,X,Y,Z$ represents the number of white, black, yellow,
dappled bulls, you will get 
a systems of 7 linear equations, the first two being 
\begin{align*}
  W &= (1/2 + 1/3)X + Z \\
  X &= (1/4 + 1/5)Y + Z
\end{align*}
together with some constraints such as $W + X$ must be a square.
After some manipulations, it can be shown that the equation to solve looks like
\[
  x^2 - 410286423278424 y^2 = 1
\]
What was Archimedes thinking? You are find information on the Archimedes Cattle Problem on the web.]
}


  \solutionlink{sol:semigroup-associativity-0}
  \qed
\end{ex} 
\begin{python0}
from solutions import *
add(label="ex:semigroup-associativity-0",
    srcfilename='exercises/semigroup-associativity-0/answer.tex') 
\end{python0}


The decryption is the same algorithm as the encryption
except that the permutation is read in the \lq\lq opposite direction''.
If
\[
  \pi =
  \begin{pmatrix}
    1 & 2 & 3 & 4 & 5 \\
    3 & 1 & 2 & 5 & 4 \\
  \end{pmatrix}
\]
then the inverse of $\pi$, denoted by $\pi^{-1}$ is the \lq\lq opposite''
of the above:
\[
  \pi^{-1} =
  \begin{pmatrix}
    3 & 1 & 2 & 5 & 4 \\
    1 & 2 & 3 & 4 & 5 \\
  \end{pmatrix}
\]
(do you see the row switch?) and to make things look nice, you arrange the
top row:
\[
  \pi^{-1} =
  \begin{pmatrix}
    1 & 2 & 3 & 4 & 5 \\
    2 & 3 & 1 & 5 & 4 \\
  \end{pmatrix}
\]
Using the cycle notation,
\[
  \pi^{-1} = (1 \ \ 2 \ \ 3)(4 \ \ 5)
\]
So
\[
  D(\pi, y)
\]
is the same as
\[
  E(\pi^{-1}, y)
\]
So
\begin{align*}
  D((1 \ \ 3 \ \ 2)(4 \ \ 5), \texttt{armivn})
  &= E((1 \ \ 3 \ \ 2)(4 \ \ 5)^{-1}, \texttt{armivn}) \\
  &= E((1 \ \ 2 \ \ 3)(4 \ \ 5), \texttt{armivn}) \\
  &= \texttt{marvin}
\end{align*}


\begin{ex} 
  \label{ex:semigroup-associativity-0}
  \tinysidebar{\debug{exercises/{exercise-12/question.tex}}}
\mbox{}
  \begin{myenum}
  \item
    Solve
    \[
      5x^2 + y^2 = 3
    \]
    % mod 5, squares = 0^2=0, 1^2=1, 2^2=4, 3^2=4, 4^2=1
    (HINT: You don't really need number theory for this one. Why?
    But if you want to, imitate the solution for the previous
    problem.)
  \item
    Solve
    \[
      11y^2 - 5x^2 = 3
    \]
    % mod 5, squares = 0,1,4
    (HINT: This is just a slight change from the
    previous problem. \textit{But} now you need number theory. Try mod 4.
    If it does not work, try mod 5. Repeat.)
    % 3x^2 + y^2 = 3
    % {0,3} + {0,1} = 3
    % 0, 1, 3, 0 = 3
    % So x = 1, y = 0 (4)
    %
    % y^2=3 (5)
  \item
    Solve
    \[
      y^2 - 5x^2 = 2
    \]
    % mod 5, squares = 0^2=0, 1^2=1, 2^2=4, 3^2=4, 4^2=1    

  \item
    What about this one:
    \[
      x^2 - 5y^2 = 1
    \]    
  \end{myenum}
  {\scriptsize
[\textsc{Aside.}
Integer solutions to $x^2 - dy^2 = 1$ has been studied since at least 400BC.
This equation appear the Cattle Problem of Archimedes:

\begin{itemize}
  \item[]
If thou art diligent and wise, O stranger, compute the number of
cattle of the Sun, who once upon a time grazed on the fields of the
Thrinacian isle of Sicily, divided into four herds of different colours,
one milk white, another a glossy black, the third yellow and the last
dappled. In each herd were bulls, mighty in number according to these
proportions: Understand, stranger, that the white bulls were equal to
a half and a third of the black together with the whole of the yellow,
while the black were equal to the fourth part of the dappled and
a fifth, together with, once more, the whole of the yellow. Observe
further that the remaining bulls, the dappled, were equal to a sixth
part of the white and a seventh, together with all the yellow. These
were the proportions of the cows: The white were precisely equal to the
third part and a fourth of the whole herd of the black; while the black
were equal to the fourth part once more of the dappled and with it a
fifth part, when all, including the bulls went to pasture together. Now
the dappled in four parts8 were equal in number to a fifth part and a
sixth of the yellow herd. Finally the yellow were in number equal to
a sixth part and a seventh of the white herd. If thou canst accurately
tell, O stranger, the number of cattle of the Sun, giving separately the
number of well-fed bulls and again the number of females according
to each colour, thou wouldst not be called unskilled or ignorant of
numbers, but not yet shall thou be numbered among the wise. But
come, understand also all these conditions regarding the cows of the
Sun. When the white bulls mingled their number with the black, they
stood firm, equal in depth and breadth, and the plains of Thrinacia,
stretching far in all ways, were filled with their multitude. Again,
when the yellow and the dappled bulls were gathered into one herd
they stood in such a manner that their number, beginning from one,
grew slowly greater till it completed a triangular figure, there being
no bulls of other colours in their midst nor none of them lacking.
If thou art able, O stranger, to find out all these things and gather
them together in your mind, giving all the relations, thou shalt depart
crowned with glory and knowing that thou hast been adjudged perfect
in this species of wisdom.
\end{itemize}

If $W,X,Y,Z$ represents the number of white, black, yellow,
dappled bulls, you will get 
a systems of 7 linear equations, the first two being 
\begin{align*}
  W &= (1/2 + 1/3)X + Z \\
  X &= (1/4 + 1/5)Y + Z
\end{align*}
together with some constraints such as $W + X$ must be a square.
After some manipulations, it can be shown that the equation to solve looks like
\[
  x^2 - 410286423278424 y^2 = 1
\]
What was Archimedes thinking? You are find information on the Archimedes Cattle Problem on the web.]
}


  \solutionlink{sol:semigroup-associativity-0}
  \qed
\end{ex} 
\begin{python0}
from solutions import *
add(label="ex:semigroup-associativity-0",
    srcfilename='exercises/semigroup-associativity-0/answer.tex') 
\end{python0}


\begin{ex} 
  \label{ex:semigroup-associativity-0}
  \tinysidebar{\debug{exercises/{exercise-12/question.tex}}}
\mbox{}
  \begin{myenum}
  \item
    Solve
    \[
      5x^2 + y^2 = 3
    \]
    % mod 5, squares = 0^2=0, 1^2=1, 2^2=4, 3^2=4, 4^2=1
    (HINT: You don't really need number theory for this one. Why?
    But if you want to, imitate the solution for the previous
    problem.)
  \item
    Solve
    \[
      11y^2 - 5x^2 = 3
    \]
    % mod 5, squares = 0,1,4
    (HINT: This is just a slight change from the
    previous problem. \textit{But} now you need number theory. Try mod 4.
    If it does not work, try mod 5. Repeat.)
    % 3x^2 + y^2 = 3
    % {0,3} + {0,1} = 3
    % 0, 1, 3, 0 = 3
    % So x = 1, y = 0 (4)
    %
    % y^2=3 (5)
  \item
    Solve
    \[
      y^2 - 5x^2 = 2
    \]
    % mod 5, squares = 0^2=0, 1^2=1, 2^2=4, 3^2=4, 4^2=1    

  \item
    What about this one:
    \[
      x^2 - 5y^2 = 1
    \]    
  \end{myenum}
  {\scriptsize
[\textsc{Aside.}
Integer solutions to $x^2 - dy^2 = 1$ has been studied since at least 400BC.
This equation appear the Cattle Problem of Archimedes:

\begin{itemize}
  \item[]
If thou art diligent and wise, O stranger, compute the number of
cattle of the Sun, who once upon a time grazed on the fields of the
Thrinacian isle of Sicily, divided into four herds of different colours,
one milk white, another a glossy black, the third yellow and the last
dappled. In each herd were bulls, mighty in number according to these
proportions: Understand, stranger, that the white bulls were equal to
a half and a third of the black together with the whole of the yellow,
while the black were equal to the fourth part of the dappled and
a fifth, together with, once more, the whole of the yellow. Observe
further that the remaining bulls, the dappled, were equal to a sixth
part of the white and a seventh, together with all the yellow. These
were the proportions of the cows: The white were precisely equal to the
third part and a fourth of the whole herd of the black; while the black
were equal to the fourth part once more of the dappled and with it a
fifth part, when all, including the bulls went to pasture together. Now
the dappled in four parts8 were equal in number to a fifth part and a
sixth of the yellow herd. Finally the yellow were in number equal to
a sixth part and a seventh of the white herd. If thou canst accurately
tell, O stranger, the number of cattle of the Sun, giving separately the
number of well-fed bulls and again the number of females according
to each colour, thou wouldst not be called unskilled or ignorant of
numbers, but not yet shall thou be numbered among the wise. But
come, understand also all these conditions regarding the cows of the
Sun. When the white bulls mingled their number with the black, they
stood firm, equal in depth and breadth, and the plains of Thrinacia,
stretching far in all ways, were filled with their multitude. Again,
when the yellow and the dappled bulls were gathered into one herd
they stood in such a manner that their number, beginning from one,
grew slowly greater till it completed a triangular figure, there being
no bulls of other colours in their midst nor none of them lacking.
If thou art able, O stranger, to find out all these things and gather
them together in your mind, giving all the relations, thou shalt depart
crowned with glory and knowing that thou hast been adjudged perfect
in this species of wisdom.
\end{itemize}

If $W,X,Y,Z$ represents the number of white, black, yellow,
dappled bulls, you will get 
a systems of 7 linear equations, the first two being 
\begin{align*}
  W &= (1/2 + 1/3)X + Z \\
  X &= (1/4 + 1/5)Y + Z
\end{align*}
together with some constraints such as $W + X$ must be a square.
After some manipulations, it can be shown that the equation to solve looks like
\[
  x^2 - 410286423278424 y^2 = 1
\]
What was Archimedes thinking? You are find information on the Archimedes Cattle Problem on the web.]
}


  \solutionlink{sol:semigroup-associativity-0}
  \qed
\end{ex} 
\begin{python0}
from solutions import *
add(label="ex:semigroup-associativity-0",
    srcfilename='exercises/semigroup-associativity-0/answer.tex') 
\end{python0}


\begin{ex} 
  \label{ex:semigroup-associativity-0}
  \tinysidebar{\debug{exercises/{exercise-12/question.tex}}}
\mbox{}
  \begin{myenum}
  \item
    Solve
    \[
      5x^2 + y^2 = 3
    \]
    % mod 5, squares = 0^2=0, 1^2=1, 2^2=4, 3^2=4, 4^2=1
    (HINT: You don't really need number theory for this one. Why?
    But if you want to, imitate the solution for the previous
    problem.)
  \item
    Solve
    \[
      11y^2 - 5x^2 = 3
    \]
    % mod 5, squares = 0,1,4
    (HINT: This is just a slight change from the
    previous problem. \textit{But} now you need number theory. Try mod 4.
    If it does not work, try mod 5. Repeat.)
    % 3x^2 + y^2 = 3
    % {0,3} + {0,1} = 3
    % 0, 1, 3, 0 = 3
    % So x = 1, y = 0 (4)
    %
    % y^2=3 (5)
  \item
    Solve
    \[
      y^2 - 5x^2 = 2
    \]
    % mod 5, squares = 0^2=0, 1^2=1, 2^2=4, 3^2=4, 4^2=1    

  \item
    What about this one:
    \[
      x^2 - 5y^2 = 1
    \]    
  \end{myenum}
  {\scriptsize
[\textsc{Aside.}
Integer solutions to $x^2 - dy^2 = 1$ has been studied since at least 400BC.
This equation appear the Cattle Problem of Archimedes:

\begin{itemize}
  \item[]
If thou art diligent and wise, O stranger, compute the number of
cattle of the Sun, who once upon a time grazed on the fields of the
Thrinacian isle of Sicily, divided into four herds of different colours,
one milk white, another a glossy black, the third yellow and the last
dappled. In each herd were bulls, mighty in number according to these
proportions: Understand, stranger, that the white bulls were equal to
a half and a third of the black together with the whole of the yellow,
while the black were equal to the fourth part of the dappled and
a fifth, together with, once more, the whole of the yellow. Observe
further that the remaining bulls, the dappled, were equal to a sixth
part of the white and a seventh, together with all the yellow. These
were the proportions of the cows: The white were precisely equal to the
third part and a fourth of the whole herd of the black; while the black
were equal to the fourth part once more of the dappled and with it a
fifth part, when all, including the bulls went to pasture together. Now
the dappled in four parts8 were equal in number to a fifth part and a
sixth of the yellow herd. Finally the yellow were in number equal to
a sixth part and a seventh of the white herd. If thou canst accurately
tell, O stranger, the number of cattle of the Sun, giving separately the
number of well-fed bulls and again the number of females according
to each colour, thou wouldst not be called unskilled or ignorant of
numbers, but not yet shall thou be numbered among the wise. But
come, understand also all these conditions regarding the cows of the
Sun. When the white bulls mingled their number with the black, they
stood firm, equal in depth and breadth, and the plains of Thrinacia,
stretching far in all ways, were filled with their multitude. Again,
when the yellow and the dappled bulls were gathered into one herd
they stood in such a manner that their number, beginning from one,
grew slowly greater till it completed a triangular figure, there being
no bulls of other colours in their midst nor none of them lacking.
If thou art able, O stranger, to find out all these things and gather
them together in your mind, giving all the relations, thou shalt depart
crowned with glory and knowing that thou hast been adjudged perfect
in this species of wisdom.
\end{itemize}

If $W,X,Y,Z$ represents the number of white, black, yellow,
dappled bulls, you will get 
a systems of 7 linear equations, the first two being 
\begin{align*}
  W &= (1/2 + 1/3)X + Z \\
  X &= (1/4 + 1/5)Y + Z
\end{align*}
together with some constraints such as $W + X$ must be a square.
After some manipulations, it can be shown that the equation to solve looks like
\[
  x^2 - 410286423278424 y^2 = 1
\]
What was Archimedes thinking? You are find information on the Archimedes Cattle Problem on the web.]
}


  \solutionlink{sol:semigroup-associativity-0}
  \qed
\end{ex} 
\begin{python0}
from solutions import *
add(label="ex:semigroup-associativity-0",
    srcfilename='exercises/semigroup-associativity-0/answer.tex') 
\end{python0}



\textsc{Breaking the permutation cipher}

Frequency analysis obviously won't help: each letter is not replaced by another.
It's the position of a letter that is changed.

Now suppose the permutation is
\[
  \pi =
  \begin{pmatrix}
    1 & 2 & 3 & 4 & 5 \\
    3 & 1 & 2 & 5 & 4 \\
  \end{pmatrix}
\]
Suppose you know the length of the permutation is 5.
After breaking your ciphertext into substrings of length 5,
suppose one of these substrings is
\begin{Verbatim}
  hatoe
\end{Verbatim}
We know that \verb!th! is a commonly occurring digram.
Then it's natural to suspect that this comes from the encryption of 5 letters containing
\verb!th!,
either
\begin{Verbatim}
  plaintext : ..... th... ...
  ciphertext: ..... hatoe ...
\end{Verbatim}
or
\begin{Verbatim}
  plaintext : ..... .th.. ...
  ciphertext: ..... hatoe ...
\end{Verbatim}
or
\begin{Verbatim}
  plaintext : ..... ..th. ...
  ciphertext: ..... hatoe ...
\end{Verbatim}
or
\begin{Verbatim}
  plaintext : ..... ...th ...
  ciphertext: ..... hatoe ...
\end{Verbatim}
Whereas for a substring of length 5, there are $5! = 120$ possible
permutations,
once two is fixed, there are $3! = 6$ possible permutations.
This might cut down on the search for the key and help break the ciphertext.
If \verb!th! is not found or the above analysis involving \verb!th!
does not work, you go on to the next commonly occurring digram.

Of course you do not know if the length of the permutation is length 5.
You therefore have to do a loop over all possible permutation lengths.
Of course the key length cannot be 1 since that would be doing nothing!
For key of size two, the permutation must be $(1, 2)$.
For a key of size three, there are $3! = 6$ permutations:
There are 6 possible permutations:
$(1)(2)(3)$, $(1)(2, 3)$, $(2)(1, 3)$, $(3)(1, 2)$,
$(1, 2, 3)$, $(1, 3, 2)$.
Since there are not many permutations of length 3, for this case,
you might want to simply try all permutations.
In general, you try key lengths dividing the length of the ciphertext.

Note that it's also possible that the \verb!th! in the plaintext
is broken up with the \verb!t! is one substring of length 5
and \verb!h! in the following substring.
For instance, here is a case:
\begin{Verbatim}[commandchars=\\\{\}]
  plaintext : cant\underline{h} \underline{e}fish         
  ciphertext: acn\underline{h}t f\underline{e}ihs
\end{Verbatim}
Of course it's also OK if you do not wish to consider cases where the
digram is split across two substrings, since in the above case,
you would have picked up the \verb!he! when you consider a key of length (say) 10.

By looking at lots of digrams and trigrams, if the ciphertext is long enough, you
should be able to break the code.

\newpage
\begin{eg}
  Let me break the following permutation ciphertext
\[
  \texttt{imahsnrwnesotanfusefvitieedoslwswma}
\]
\end{eg}

\SOLUTION

The length is 35. Since the length of the permutation must divide 35,
it must be 1, 5, 7, 35.


\textsc{Length 1.}
Of course if the length is 1, then the ciphertext is the same as the
plaintext. But the plaintext is meaningless. Therefore the length
cannot be 1.

\textsc{Length 5.}
Assume the key length is 5.
I break up the string
above into substrings of length 5 to make it easier to read:
\begin{Verbatim}
  imahs nrwne sotan fusef vitie edosl wswma
\end{Verbatim}


\textsc{Length 5. Digram \texttt{th}.}
I don't see any \verb!th! in each substrings or consecutive substrings.


\textsc{Length 5. Digram \texttt{he}.}
I do see \verb!he!:
\begin{Verbatim}[commandchars=\\\{\}]
  ima\underline{h}s nrwn\underline{e} sotan fusef vitie edosl wswma
\end{Verbatim}
If you assume these two characters come from \verb!he!, then it has to come from this:
\begin{Verbatim}[commandchars=\\\{\}]
 plaintext:  ....h e.... ..... ..... ..... ..... .....
 ciphertext: ima\underline{h}s nrwn\underline{e} sotan fusef vitie edosl wswma
\end{Verbatim}
which means the permutation contains $1 \mapsto 5, 5 \mapsto 4$.
If that's the case, we have
\begin{Verbatim}[commandchars=\\\{\}]
 plaintext:  s...h e...n n...a f...e e...i l...s a...m
 ciphertext: imahs nrwne sotan fusef vitie edosl wswma
\end{Verbatim}
Look at just the first group of 5 characters
the \verb!s...h! can only be
\verb!simah!,
\verb!siamh!,
\verb!saimh!,
\verb!samih!,
\verb!smiah!, or
\verb!smaih!.
The only promising one is \verb!siamh! (maybe \lq\lq Siam has cats ...?").
Using this we get
\begin{Verbatim}[commandchars=\\\{\}]
 plaintext:  siamh enwrn nstoa ffsue evtii leods awwsm
 ciphertext: imahs nrwne sotan fusef vitie edosl wswma
\end{Verbatim}
which is meaningless.

\newpage
[force newpage]

\textsc{Length 5. Digram \texttt{in}.}
The next digram to try is \verb!in!:
\begin{Verbatim}[commandchars=\\\{\}]
 plaintext:  ....i n.... ..... ..... ..... ..... .....
 ciphertext: \underline{i}mahs \underline{n}rwne sotan fusef vitie edosl wswma
\end{Verbatim}
or
\begin{Verbatim}[commandchars=\\\{\}]
 plaintext:  ....i n.... ..... ..... ..... ..... .....
 ciphertext: \underline{i}mahs nrw\underline{n}e sotan fusef vitie edosl wswma
\end{Verbatim}
The first is not possible. (Why?)
% the i means 5->1 and the n means 1->1. cannot have 5->1,1->1 for permutation.
The second gives us
\begin{Verbatim}[commandchars=\\\{\}]
 plaintext:  h...i n...n a...s e...f i...v s...e m...w
 ciphertext: imahs nrwne sotan fusef vitie edosl wswma
\end{Verbatim}
Looking at just the first 5 characters,
the only possible decryptions are
\verb!hmasi!,
\verb!hmsai!,
\verb!hamsi!,
\verb!hasmi!,
\verb!hsami!,
\verb!hsmai!.
Only
\verb!hamsi!,
\verb!hasmi!,
are promising.
From \verb!hamsi!,
\begin{Verbatim}[commandchars=\\\{\}]
 plaintext:  hamsi nwren atons esuff itiev sodle mwsaw
 ciphertext: imahs nrwne sotan fusef vitie edosl wswma
\end{Verbatim}
which is meaningless.
From \verb!hasmi!,
\begin{Verbatim}[commandchars=\\\{\}]
 plaintext:  hasmi nwern atnos esfuf iteiv solde mwasw
 ciphertext: imahs nrwne sotan fusef vitie edosl wswma
\end{Verbatim}
which is meaningless.

\textsc{Length 5. Digram \texttt{er}.}
The next digram to try is \verb!er!:
\begin{itemize}
  \li From
  \begin{Verbatim}[commandchars=\\\{\}]
 plaintext:  ..... er... ..... ..... ..... ..... .....
 ciphertext: imahs nrwne sotan fusef vitie edosl wswma
\end{Verbatim}
we get
  \begin{Verbatim}[commandchars=\\\{\}]
 plaintext:  sm... er... no... fu... ei... ld... as...
 ciphertext: imahs nrwne sotan fusef vitie edosl wswma
\end{Verbatim}
The first 5 characters can only be
\verb!smahs!,
\verb!smash!,
\verb!smhas!,
\verb!smhsa!,
\verb!smsah!,
\verb!smsha!. Only \verb!smash! seems to form a word.
(\verb!smhas! might be for instance \texttt{a\ cha\underline{sm\ has}\ separated\ us},
but the \verb!smhas! is the \textit{beginning} of the plaintext.)
Using that, we get
  \begin{Verbatim}[commandchars=\\\{\}]
 plaintext:  smash erwen notna fusfe eitei ldols aswam
 ciphertext: imahs nrwne sotan fusef vitie edosl wswma
\end{Verbatim}
which is not meaningful.

  \li From
  \begin{Verbatim}[commandchars=\\\{\}]
 plaintext:  ..... .er.. ..... ..... ..... ..... .....
 ciphertext: imahs nrwne sotan fusef vitie edosl wswma
\end{Verbatim}
we get
  \begin{Verbatim}[commandchars=\\\{\}]
 plaintext:  .sm.. .er.. .no.. .fu.. .ei.. .ld.. .as..
 ciphertext: imahs nrwne sotan fusef vitie edosl wswma
\end{Verbatim}
The first 5 characters can only be
\verb!ismah!,
\verb!ismha!,
\verb!asmih!,
\verb!asmhi!,
\verb!hsmia!,
\verb!hsmai!,
none of which is meaningful,
except possibly for
\verb!ismah! (example: \texttt{\underline{is\ mah}i\ mahi\ a\ fish?}),
\verb!ismha! (example: \texttt{a sch\underline{ism\ ha}ppened overnight}),
\verb!asmih! (example: \texttt{protopl\underline{asm i h}ave}),
\verb!asmhi! (example: \texttt{protopl\underline{asm h}ave i})
but you can check that these do not lead
to anything meaningful.


\li From
  \begin{Verbatim}[commandchars=\\\{\}]
 plaintext:  ..... ..er. ..... ..... ..... ..... .....
 ciphertext: imahs nrwne sotan fusef vitie edosl wswma
\end{Verbatim}
we get
  \begin{Verbatim}[commandchars=\\\{\}]
 plaintext:  ..sm. ..er. ..no. ..fu. ..ei. ..ld. ..as.
 ciphertext: imahs nrwne sotan fusef vitie edosl wswma
\end{Verbatim}
The first 5 characters can only be
\verb!iasmh!,
\verb!ihsma!,
\verb!aismh!,
\verb!ahsmi!,
\verb!hisma!,
\verb!hasmi!.
Only
\verb!ahsmi!,
\verb!hisma!
and
\verb!hasmi!
seem promising.
For \verb!ahsmi!, we get
  \begin{Verbatim}[commandchars=\\\{\}]
 plaintext:  ahsmi wnern tanos sefuf tieiv oslde wmasw
 ciphertext: imahs nrwne sotan fusef vitie edosl wswma
\end{Verbatim}
which is meaningless.
For \verb!hisma!, we get
  \begin{Verbatim}[commandchars=\\\{\}]
 plaintext:  hisma nnerw asnot effus iveit seldo mwasw
 ciphertext: imahs nrwne sotan fusef vitie edosl wswma
 \end{Verbatim}
 which finally gives us the plaintext
 \[
   \texttt{hismannerwasnoteffusiveitseldomwasw}
 \]
 This gives us
 \[
    \texttt{His manner was not effusive.\ It seldom was.}
 \]
 % hismannerwasnoteffusiveitseldomwasw
 The last \verb!w! is redundant (it's a padding).
 \end{itemize}

 The key (i.e., permutation) is
 $(1,4,2)(3,5)$ with a key length of 5.
 \qed

\begin{comment}
plainttext: His manner was not effusive. It seldom was.
key = (1, 4, 2)(3, 5)

hisma - imahs
nnerw - nrwne
asnot - sotan
effus - fusef
iveit - vitie
seldo - edosl
mwasw - wswma
\end{comment}


\begin{ex} 
  \label{ex:semigroup-associativity-0}
  \tinysidebar{\debug{exercises/{exercise-12/question.tex}}}
\mbox{}
  \begin{myenum}
  \item
    Solve
    \[
      5x^2 + y^2 = 3
    \]
    % mod 5, squares = 0^2=0, 1^2=1, 2^2=4, 3^2=4, 4^2=1
    (HINT: You don't really need number theory for this one. Why?
    But if you want to, imitate the solution for the previous
    problem.)
  \item
    Solve
    \[
      11y^2 - 5x^2 = 3
    \]
    % mod 5, squares = 0,1,4
    (HINT: This is just a slight change from the
    previous problem. \textit{But} now you need number theory. Try mod 4.
    If it does not work, try mod 5. Repeat.)
    % 3x^2 + y^2 = 3
    % {0,3} + {0,1} = 3
    % 0, 1, 3, 0 = 3
    % So x = 1, y = 0 (4)
    %
    % y^2=3 (5)
  \item
    Solve
    \[
      y^2 - 5x^2 = 2
    \]
    % mod 5, squares = 0^2=0, 1^2=1, 2^2=4, 3^2=4, 4^2=1    

  \item
    What about this one:
    \[
      x^2 - 5y^2 = 1
    \]    
  \end{myenum}
  {\scriptsize
[\textsc{Aside.}
Integer solutions to $x^2 - dy^2 = 1$ has been studied since at least 400BC.
This equation appear the Cattle Problem of Archimedes:

\begin{itemize}
  \item[]
If thou art diligent and wise, O stranger, compute the number of
cattle of the Sun, who once upon a time grazed on the fields of the
Thrinacian isle of Sicily, divided into four herds of different colours,
one milk white, another a glossy black, the third yellow and the last
dappled. In each herd were bulls, mighty in number according to these
proportions: Understand, stranger, that the white bulls were equal to
a half and a third of the black together with the whole of the yellow,
while the black were equal to the fourth part of the dappled and
a fifth, together with, once more, the whole of the yellow. Observe
further that the remaining bulls, the dappled, were equal to a sixth
part of the white and a seventh, together with all the yellow. These
were the proportions of the cows: The white were precisely equal to the
third part and a fourth of the whole herd of the black; while the black
were equal to the fourth part once more of the dappled and with it a
fifth part, when all, including the bulls went to pasture together. Now
the dappled in four parts8 were equal in number to a fifth part and a
sixth of the yellow herd. Finally the yellow were in number equal to
a sixth part and a seventh of the white herd. If thou canst accurately
tell, O stranger, the number of cattle of the Sun, giving separately the
number of well-fed bulls and again the number of females according
to each colour, thou wouldst not be called unskilled or ignorant of
numbers, but not yet shall thou be numbered among the wise. But
come, understand also all these conditions regarding the cows of the
Sun. When the white bulls mingled their number with the black, they
stood firm, equal in depth and breadth, and the plains of Thrinacia,
stretching far in all ways, were filled with their multitude. Again,
when the yellow and the dappled bulls were gathered into one herd
they stood in such a manner that their number, beginning from one,
grew slowly greater till it completed a triangular figure, there being
no bulls of other colours in their midst nor none of them lacking.
If thou art able, O stranger, to find out all these things and gather
them together in your mind, giving all the relations, thou shalt depart
crowned with glory and knowing that thou hast been adjudged perfect
in this species of wisdom.
\end{itemize}

If $W,X,Y,Z$ represents the number of white, black, yellow,
dappled bulls, you will get 
a systems of 7 linear equations, the first two being 
\begin{align*}
  W &= (1/2 + 1/3)X + Z \\
  X &= (1/4 + 1/5)Y + Z
\end{align*}
together with some constraints such as $W + X$ must be a square.
After some manipulations, it can be shown that the equation to solve looks like
\[
  x^2 - 410286423278424 y^2 = 1
\]
What was Archimedes thinking? You are find information on the Archimedes Cattle Problem on the web.]
}


  \solutionlink{sol:semigroup-associativity-0}
  \qed
\end{ex} 
\begin{python0}
from solutions import *
add(label="ex:semigroup-associativity-0",
    srcfilename='exercises/semigroup-associativity-0/answer.tex') 
\end{python0}


\begin{ex} 
  \label{ex:semigroup-associativity-0}
  \tinysidebar{\debug{exercises/{exercise-12/question.tex}}}
\mbox{}
  \begin{myenum}
  \item
    Solve
    \[
      5x^2 + y^2 = 3
    \]
    % mod 5, squares = 0^2=0, 1^2=1, 2^2=4, 3^2=4, 4^2=1
    (HINT: You don't really need number theory for this one. Why?
    But if you want to, imitate the solution for the previous
    problem.)
  \item
    Solve
    \[
      11y^2 - 5x^2 = 3
    \]
    % mod 5, squares = 0,1,4
    (HINT: This is just a slight change from the
    previous problem. \textit{But} now you need number theory. Try mod 4.
    If it does not work, try mod 5. Repeat.)
    % 3x^2 + y^2 = 3
    % {0,3} + {0,1} = 3
    % 0, 1, 3, 0 = 3
    % So x = 1, y = 0 (4)
    %
    % y^2=3 (5)
  \item
    Solve
    \[
      y^2 - 5x^2 = 2
    \]
    % mod 5, squares = 0^2=0, 1^2=1, 2^2=4, 3^2=4, 4^2=1    

  \item
    What about this one:
    \[
      x^2 - 5y^2 = 1
    \]    
  \end{myenum}
  {\scriptsize
[\textsc{Aside.}
Integer solutions to $x^2 - dy^2 = 1$ has been studied since at least 400BC.
This equation appear the Cattle Problem of Archimedes:

\begin{itemize}
  \item[]
If thou art diligent and wise, O stranger, compute the number of
cattle of the Sun, who once upon a time grazed on the fields of the
Thrinacian isle of Sicily, divided into four herds of different colours,
one milk white, another a glossy black, the third yellow and the last
dappled. In each herd were bulls, mighty in number according to these
proportions: Understand, stranger, that the white bulls were equal to
a half and a third of the black together with the whole of the yellow,
while the black were equal to the fourth part of the dappled and
a fifth, together with, once more, the whole of the yellow. Observe
further that the remaining bulls, the dappled, were equal to a sixth
part of the white and a seventh, together with all the yellow. These
were the proportions of the cows: The white were precisely equal to the
third part and a fourth of the whole herd of the black; while the black
were equal to the fourth part once more of the dappled and with it a
fifth part, when all, including the bulls went to pasture together. Now
the dappled in four parts8 were equal in number to a fifth part and a
sixth of the yellow herd. Finally the yellow were in number equal to
a sixth part and a seventh of the white herd. If thou canst accurately
tell, O stranger, the number of cattle of the Sun, giving separately the
number of well-fed bulls and again the number of females according
to each colour, thou wouldst not be called unskilled or ignorant of
numbers, but not yet shall thou be numbered among the wise. But
come, understand also all these conditions regarding the cows of the
Sun. When the white bulls mingled their number with the black, they
stood firm, equal in depth and breadth, and the plains of Thrinacia,
stretching far in all ways, were filled with their multitude. Again,
when the yellow and the dappled bulls were gathered into one herd
they stood in such a manner that their number, beginning from one,
grew slowly greater till it completed a triangular figure, there being
no bulls of other colours in their midst nor none of them lacking.
If thou art able, O stranger, to find out all these things and gather
them together in your mind, giving all the relations, thou shalt depart
crowned with glory and knowing that thou hast been adjudged perfect
in this species of wisdom.
\end{itemize}

If $W,X,Y,Z$ represents the number of white, black, yellow,
dappled bulls, you will get 
a systems of 7 linear equations, the first two being 
\begin{align*}
  W &= (1/2 + 1/3)X + Z \\
  X &= (1/4 + 1/5)Y + Z
\end{align*}
together with some constraints such as $W + X$ must be a square.
After some manipulations, it can be shown that the equation to solve looks like
\[
  x^2 - 410286423278424 y^2 = 1
\]
What was Archimedes thinking? You are find information on the Archimedes Cattle Problem on the web.]
}


  \solutionlink{sol:semigroup-associativity-0}
  \qed
\end{ex} 
\begin{python0}
from solutions import *
add(label="ex:semigroup-associativity-0",
    srcfilename='exercises/semigroup-associativity-0/answer.tex') 
\end{python0}


\begin{ex} 
  \label{ex:semigroup-associativity-0}
  \tinysidebar{\debug{exercises/{exercise-12/question.tex}}}
\mbox{}
  \begin{myenum}
  \item
    Solve
    \[
      5x^2 + y^2 = 3
    \]
    % mod 5, squares = 0^2=0, 1^2=1, 2^2=4, 3^2=4, 4^2=1
    (HINT: You don't really need number theory for this one. Why?
    But if you want to, imitate the solution for the previous
    problem.)
  \item
    Solve
    \[
      11y^2 - 5x^2 = 3
    \]
    % mod 5, squares = 0,1,4
    (HINT: This is just a slight change from the
    previous problem. \textit{But} now you need number theory. Try mod 4.
    If it does not work, try mod 5. Repeat.)
    % 3x^2 + y^2 = 3
    % {0,3} + {0,1} = 3
    % 0, 1, 3, 0 = 3
    % So x = 1, y = 0 (4)
    %
    % y^2=3 (5)
  \item
    Solve
    \[
      y^2 - 5x^2 = 2
    \]
    % mod 5, squares = 0^2=0, 1^2=1, 2^2=4, 3^2=4, 4^2=1    

  \item
    What about this one:
    \[
      x^2 - 5y^2 = 1
    \]    
  \end{myenum}
  {\scriptsize
[\textsc{Aside.}
Integer solutions to $x^2 - dy^2 = 1$ has been studied since at least 400BC.
This equation appear the Cattle Problem of Archimedes:

\begin{itemize}
  \item[]
If thou art diligent and wise, O stranger, compute the number of
cattle of the Sun, who once upon a time grazed on the fields of the
Thrinacian isle of Sicily, divided into four herds of different colours,
one milk white, another a glossy black, the third yellow and the last
dappled. In each herd were bulls, mighty in number according to these
proportions: Understand, stranger, that the white bulls were equal to
a half and a third of the black together with the whole of the yellow,
while the black were equal to the fourth part of the dappled and
a fifth, together with, once more, the whole of the yellow. Observe
further that the remaining bulls, the dappled, were equal to a sixth
part of the white and a seventh, together with all the yellow. These
were the proportions of the cows: The white were precisely equal to the
third part and a fourth of the whole herd of the black; while the black
were equal to the fourth part once more of the dappled and with it a
fifth part, when all, including the bulls went to pasture together. Now
the dappled in four parts8 were equal in number to a fifth part and a
sixth of the yellow herd. Finally the yellow were in number equal to
a sixth part and a seventh of the white herd. If thou canst accurately
tell, O stranger, the number of cattle of the Sun, giving separately the
number of well-fed bulls and again the number of females according
to each colour, thou wouldst not be called unskilled or ignorant of
numbers, but not yet shall thou be numbered among the wise. But
come, understand also all these conditions regarding the cows of the
Sun. When the white bulls mingled their number with the black, they
stood firm, equal in depth and breadth, and the plains of Thrinacia,
stretching far in all ways, were filled with their multitude. Again,
when the yellow and the dappled bulls were gathered into one herd
they stood in such a manner that their number, beginning from one,
grew slowly greater till it completed a triangular figure, there being
no bulls of other colours in their midst nor none of them lacking.
If thou art able, O stranger, to find out all these things and gather
them together in your mind, giving all the relations, thou shalt depart
crowned with glory and knowing that thou hast been adjudged perfect
in this species of wisdom.
\end{itemize}

If $W,X,Y,Z$ represents the number of white, black, yellow,
dappled bulls, you will get 
a systems of 7 linear equations, the first two being 
\begin{align*}
  W &= (1/2 + 1/3)X + Z \\
  X &= (1/4 + 1/5)Y + Z
\end{align*}
together with some constraints such as $W + X$ must be a square.
After some manipulations, it can be shown that the equation to solve looks like
\[
  x^2 - 410286423278424 y^2 = 1
\]
What was Archimedes thinking? You are find information on the Archimedes Cattle Problem on the web.]
}


  \solutionlink{sol:semigroup-associativity-0}
  \qed
\end{ex} 
\begin{python0}
from solutions import *
add(label="ex:semigroup-associativity-0",
    srcfilename='exercises/semigroup-associativity-0/answer.tex') 
\end{python0}


\begin{ex} 
  \label{ex:semigroup-associativity-0}
  \tinysidebar{\debug{exercises/{exercise-12/question.tex}}}
\mbox{}
  \begin{myenum}
  \item
    Solve
    \[
      5x^2 + y^2 = 3
    \]
    % mod 5, squares = 0^2=0, 1^2=1, 2^2=4, 3^2=4, 4^2=1
    (HINT: You don't really need number theory for this one. Why?
    But if you want to, imitate the solution for the previous
    problem.)
  \item
    Solve
    \[
      11y^2 - 5x^2 = 3
    \]
    % mod 5, squares = 0,1,4
    (HINT: This is just a slight change from the
    previous problem. \textit{But} now you need number theory. Try mod 4.
    If it does not work, try mod 5. Repeat.)
    % 3x^2 + y^2 = 3
    % {0,3} + {0,1} = 3
    % 0, 1, 3, 0 = 3
    % So x = 1, y = 0 (4)
    %
    % y^2=3 (5)
  \item
    Solve
    \[
      y^2 - 5x^2 = 2
    \]
    % mod 5, squares = 0^2=0, 1^2=1, 2^2=4, 3^2=4, 4^2=1    

  \item
    What about this one:
    \[
      x^2 - 5y^2 = 1
    \]    
  \end{myenum}
  {\scriptsize
[\textsc{Aside.}
Integer solutions to $x^2 - dy^2 = 1$ has been studied since at least 400BC.
This equation appear the Cattle Problem of Archimedes:

\begin{itemize}
  \item[]
If thou art diligent and wise, O stranger, compute the number of
cattle of the Sun, who once upon a time grazed on the fields of the
Thrinacian isle of Sicily, divided into four herds of different colours,
one milk white, another a glossy black, the third yellow and the last
dappled. In each herd were bulls, mighty in number according to these
proportions: Understand, stranger, that the white bulls were equal to
a half and a third of the black together with the whole of the yellow,
while the black were equal to the fourth part of the dappled and
a fifth, together with, once more, the whole of the yellow. Observe
further that the remaining bulls, the dappled, were equal to a sixth
part of the white and a seventh, together with all the yellow. These
were the proportions of the cows: The white were precisely equal to the
third part and a fourth of the whole herd of the black; while the black
were equal to the fourth part once more of the dappled and with it a
fifth part, when all, including the bulls went to pasture together. Now
the dappled in four parts8 were equal in number to a fifth part and a
sixth of the yellow herd. Finally the yellow were in number equal to
a sixth part and a seventh of the white herd. If thou canst accurately
tell, O stranger, the number of cattle of the Sun, giving separately the
number of well-fed bulls and again the number of females according
to each colour, thou wouldst not be called unskilled or ignorant of
numbers, but not yet shall thou be numbered among the wise. But
come, understand also all these conditions regarding the cows of the
Sun. When the white bulls mingled their number with the black, they
stood firm, equal in depth and breadth, and the plains of Thrinacia,
stretching far in all ways, were filled with their multitude. Again,
when the yellow and the dappled bulls were gathered into one herd
they stood in such a manner that their number, beginning from one,
grew slowly greater till it completed a triangular figure, there being
no bulls of other colours in their midst nor none of them lacking.
If thou art able, O stranger, to find out all these things and gather
them together in your mind, giving all the relations, thou shalt depart
crowned with glory and knowing that thou hast been adjudged perfect
in this species of wisdom.
\end{itemize}

If $W,X,Y,Z$ represents the number of white, black, yellow,
dappled bulls, you will get 
a systems of 7 linear equations, the first two being 
\begin{align*}
  W &= (1/2 + 1/3)X + Z \\
  X &= (1/4 + 1/5)Y + Z
\end{align*}
together with some constraints such as $W + X$ must be a square.
After some manipulations, it can be shown that the equation to solve looks like
\[
  x^2 - 410286423278424 y^2 = 1
\]
What was Archimedes thinking? You are find information on the Archimedes Cattle Problem on the web.]
}


  \solutionlink{sol:semigroup-associativity-0}
  \qed
\end{ex} 
\begin{python0}
from solutions import *
add(label="ex:semigroup-associativity-0",
    srcfilename='exercises/semigroup-associativity-0/answer.tex') 
\end{python0}


\begin{ex} 
  \label{ex:semigroup-associativity-0}
  \tinysidebar{\debug{exercises/{exercise-12/question.tex}}}
\mbox{}
  \begin{myenum}
  \item
    Solve
    \[
      5x^2 + y^2 = 3
    \]
    % mod 5, squares = 0^2=0, 1^2=1, 2^2=4, 3^2=4, 4^2=1
    (HINT: You don't really need number theory for this one. Why?
    But if you want to, imitate the solution for the previous
    problem.)
  \item
    Solve
    \[
      11y^2 - 5x^2 = 3
    \]
    % mod 5, squares = 0,1,4
    (HINT: This is just a slight change from the
    previous problem. \textit{But} now you need number theory. Try mod 4.
    If it does not work, try mod 5. Repeat.)
    % 3x^2 + y^2 = 3
    % {0,3} + {0,1} = 3
    % 0, 1, 3, 0 = 3
    % So x = 1, y = 0 (4)
    %
    % y^2=3 (5)
  \item
    Solve
    \[
      y^2 - 5x^2 = 2
    \]
    % mod 5, squares = 0^2=0, 1^2=1, 2^2=4, 3^2=4, 4^2=1    

  \item
    What about this one:
    \[
      x^2 - 5y^2 = 1
    \]    
  \end{myenum}
  {\scriptsize
[\textsc{Aside.}
Integer solutions to $x^2 - dy^2 = 1$ has been studied since at least 400BC.
This equation appear the Cattle Problem of Archimedes:

\begin{itemize}
  \item[]
If thou art diligent and wise, O stranger, compute the number of
cattle of the Sun, who once upon a time grazed on the fields of the
Thrinacian isle of Sicily, divided into four herds of different colours,
one milk white, another a glossy black, the third yellow and the last
dappled. In each herd were bulls, mighty in number according to these
proportions: Understand, stranger, that the white bulls were equal to
a half and a third of the black together with the whole of the yellow,
while the black were equal to the fourth part of the dappled and
a fifth, together with, once more, the whole of the yellow. Observe
further that the remaining bulls, the dappled, were equal to a sixth
part of the white and a seventh, together with all the yellow. These
were the proportions of the cows: The white were precisely equal to the
third part and a fourth of the whole herd of the black; while the black
were equal to the fourth part once more of the dappled and with it a
fifth part, when all, including the bulls went to pasture together. Now
the dappled in four parts8 were equal in number to a fifth part and a
sixth of the yellow herd. Finally the yellow were in number equal to
a sixth part and a seventh of the white herd. If thou canst accurately
tell, O stranger, the number of cattle of the Sun, giving separately the
number of well-fed bulls and again the number of females according
to each colour, thou wouldst not be called unskilled or ignorant of
numbers, but not yet shall thou be numbered among the wise. But
come, understand also all these conditions regarding the cows of the
Sun. When the white bulls mingled their number with the black, they
stood firm, equal in depth and breadth, and the plains of Thrinacia,
stretching far in all ways, were filled with their multitude. Again,
when the yellow and the dappled bulls were gathered into one herd
they stood in such a manner that their number, beginning from one,
grew slowly greater till it completed a triangular figure, there being
no bulls of other colours in their midst nor none of them lacking.
If thou art able, O stranger, to find out all these things and gather
them together in your mind, giving all the relations, thou shalt depart
crowned with glory and knowing that thou hast been adjudged perfect
in this species of wisdom.
\end{itemize}

If $W,X,Y,Z$ represents the number of white, black, yellow,
dappled bulls, you will get 
a systems of 7 linear equations, the first two being 
\begin{align*}
  W &= (1/2 + 1/3)X + Z \\
  X &= (1/4 + 1/5)Y + Z
\end{align*}
together with some constraints such as $W + X$ must be a square.
After some manipulations, it can be shown that the equation to solve looks like
\[
  x^2 - 410286423278424 y^2 = 1
\]
What was Archimedes thinking? You are find information on the Archimedes Cattle Problem on the web.]
}


  \solutionlink{sol:semigroup-associativity-0}
  \qed
\end{ex} 
\begin{python0}
from solutions import *
add(label="ex:semigroup-associativity-0",
    srcfilename='exercises/semigroup-associativity-0/answer.tex') 
\end{python0}


\begin{ex} 
  \label{ex:semigroup-associativity-0}
  \tinysidebar{\debug{exercises/{exercise-12/question.tex}}}
\mbox{}
  \begin{myenum}
  \item
    Solve
    \[
      5x^2 + y^2 = 3
    \]
    % mod 5, squares = 0^2=0, 1^2=1, 2^2=4, 3^2=4, 4^2=1
    (HINT: You don't really need number theory for this one. Why?
    But if you want to, imitate the solution for the previous
    problem.)
  \item
    Solve
    \[
      11y^2 - 5x^2 = 3
    \]
    % mod 5, squares = 0,1,4
    (HINT: This is just a slight change from the
    previous problem. \textit{But} now you need number theory. Try mod 4.
    If it does not work, try mod 5. Repeat.)
    % 3x^2 + y^2 = 3
    % {0,3} + {0,1} = 3
    % 0, 1, 3, 0 = 3
    % So x = 1, y = 0 (4)
    %
    % y^2=3 (5)
  \item
    Solve
    \[
      y^2 - 5x^2 = 2
    \]
    % mod 5, squares = 0^2=0, 1^2=1, 2^2=4, 3^2=4, 4^2=1    

  \item
    What about this one:
    \[
      x^2 - 5y^2 = 1
    \]    
  \end{myenum}
  {\scriptsize
[\textsc{Aside.}
Integer solutions to $x^2 - dy^2 = 1$ has been studied since at least 400BC.
This equation appear the Cattle Problem of Archimedes:

\begin{itemize}
  \item[]
If thou art diligent and wise, O stranger, compute the number of
cattle of the Sun, who once upon a time grazed on the fields of the
Thrinacian isle of Sicily, divided into four herds of different colours,
one milk white, another a glossy black, the third yellow and the last
dappled. In each herd were bulls, mighty in number according to these
proportions: Understand, stranger, that the white bulls were equal to
a half and a third of the black together with the whole of the yellow,
while the black were equal to the fourth part of the dappled and
a fifth, together with, once more, the whole of the yellow. Observe
further that the remaining bulls, the dappled, were equal to a sixth
part of the white and a seventh, together with all the yellow. These
were the proportions of the cows: The white were precisely equal to the
third part and a fourth of the whole herd of the black; while the black
were equal to the fourth part once more of the dappled and with it a
fifth part, when all, including the bulls went to pasture together. Now
the dappled in four parts8 were equal in number to a fifth part and a
sixth of the yellow herd. Finally the yellow were in number equal to
a sixth part and a seventh of the white herd. If thou canst accurately
tell, O stranger, the number of cattle of the Sun, giving separately the
number of well-fed bulls and again the number of females according
to each colour, thou wouldst not be called unskilled or ignorant of
numbers, but not yet shall thou be numbered among the wise. But
come, understand also all these conditions regarding the cows of the
Sun. When the white bulls mingled their number with the black, they
stood firm, equal in depth and breadth, and the plains of Thrinacia,
stretching far in all ways, were filled with their multitude. Again,
when the yellow and the dappled bulls were gathered into one herd
they stood in such a manner that their number, beginning from one,
grew slowly greater till it completed a triangular figure, there being
no bulls of other colours in their midst nor none of them lacking.
If thou art able, O stranger, to find out all these things and gather
them together in your mind, giving all the relations, thou shalt depart
crowned with glory and knowing that thou hast been adjudged perfect
in this species of wisdom.
\end{itemize}

If $W,X,Y,Z$ represents the number of white, black, yellow,
dappled bulls, you will get 
a systems of 7 linear equations, the first two being 
\begin{align*}
  W &= (1/2 + 1/3)X + Z \\
  X &= (1/4 + 1/5)Y + Z
\end{align*}
together with some constraints such as $W + X$ must be a square.
After some manipulations, it can be shown that the equation to solve looks like
\[
  x^2 - 410286423278424 y^2 = 1
\]
What was Archimedes thinking? You are find information on the Archimedes Cattle Problem on the web.]
}


  \solutionlink{sol:semigroup-associativity-0}
  \qed
\end{ex} 
\begin{python0}
from solutions import *
add(label="ex:semigroup-associativity-0",
    srcfilename='exercises/semigroup-associativity-0/answer.tex') 
\end{python0}



