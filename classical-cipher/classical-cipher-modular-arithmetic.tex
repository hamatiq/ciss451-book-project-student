\sectionthree{Modular arithmetic}
\begin{python0}
from solutions import *; clear()
\end{python0}

Number theory is basically the study of whole numbers.
That however is an over-simplification.
The study of number theory involves almost all areas of mathematics.
In fact many areas of mathematics were created just to study 
certain problems in number theory.

Number theory is an extremely huge area of study in Mathematics and Computer
Science. It is also extremely fascinating. Many problems is number theory can
be stated very simply so that even a high school student can understand the
statement of the problem. And yet the techniques used to solve some of these
problems require the mathematical tools from almost every area of Math. Gauss
once said that ``Mathematics is the queen of sciences and number theory is the
queen of mathematics".

Although there are many branches within Number Theory, right now
we only need to know a little bit about Elementary Number Theory.
``Elementary" here does not mean simple (although it will be for
us since you're only seeing a small part of Elementary Number
Theory). It means we are studying Number Theory using only
properties of whole numbers (integers). Research in Number Theory
requires real numbers, complex numbers, calculus, geometry,
complex analysis, etc.

This will be a very short introduction to the vast area of Number
Theory. In fact this is only a tiny fraction of Elementary Number
Theory. This is one of the oldest area of Mathematics and one of
the fascinating because of its history. If you want to learn more
about number theory, just let me know. I can easily find a project
for you to work on.

For now, we will look at modular arithmetic.
Besides cryptography,
modular arithmetic is also used in
data compression and error correction codes.

The set of integers $\{..., -3, -2, -1, 0, 1, 2, 3, ...\}$
denoted $\Z$ has two operations $+$ and $\cdot$.
In terms of the algebraic structure (i.e. the operations), $\Z$ is known as
a \textbf{commutative ring}.
Basically a commutative ring is a set of \lq\lq things'' with
two operations, addition and multiplication, with rules that
look like the addition and multiplication rules for $\Z$.
For instance one such rule in $\Z$ is
\[
  a (b + c) = ab + ac
\]
This same rule holds true for $\Q$, $\R$, $\C$ and polynomials with coefficients in $\Z$.

The reason why mathematicians even bother defining this concept
of \lq\lq commutative ring"
is more or less
the same reason why we write functions in programs: for re-use.
There are \textit{many} naturally occurring rings.
So if while developing the theory for 
$\token{\operatorname{blah_1}}$
and
$\token{\operatorname{blah_2}}$
and they are both rings, then it's enough to prove a general fact that applies
to both and quote the fact.
This is also related to the concept of inheritance and abstract base classes.
You can think of $\Z, \Q, \R, \C$ as subclasses of \verb!CommutativeRing!.
Therefore if you have a function
\begin{console}
void f(CommutativeRing & r)
{
  ...
}
\end{console}
then \verb!f! can work with \verb!x! if \verb!x! is a object of $\Z, \Q, \R, \C$.

The mathematician develop  general results while the programmer
write general functions working on abstract base class objects.
The idea is the same.
The reason for generality is efficiency.

Here's a very important advice on studying rings, groups, fields, math, etc.
(in fact this applies to any area of study where there is a great deal of 
generalization):
Although the definition and theorems are general, you
\textit{always} keep a
couple of standard examples in your mind as you read the statements.
While reading them, mentally substitute your examples into the facts so that 
you can associate it to something more concrete.
This is not just a learn technique for undergraduate students.
Even researchers do that when they read research papers.
For the definition of ring below, think of the ring of integers.

I will try to be as informal as possible for now
so that you can develop some feel/intuition for what we need for now.
Later I'm going to come back to this topic and redo the whole thing rigorously.
The focus for now is to understand modulo 26 arithmetic.

Anyway, a \textbf{commutatative ring} $R$ 
\sidenote{
Example: the integer $\Z$,...$+$ and $\cdot$ of $\Z$ ...
}
is a set of \lq\lq things''
with two operations abstractly denoted by 
$\oplus$ and $\odot$ (\lq\lq addition'' and \lq\lq multiplication'')
Furthermore there are two special \lq\lq things'' in $R$ which we will call
$0_R$ and $1_R$.
The properties satisfied by $R,0_R,1_R,\oplus,\odot$ 
\sidebar{
0 and 1 of $\Z$ ...
}
(of course there must be
something satisfied by them!) are as follows.
For $\oplus$ the properties are:
\begin{myenum}
\item $r \oplus s$ is in $R$ for all $r,s$ in $R$
\sidebar{$i + j$ is in $\Z$ for integers $i$, $j$, ...}
\item 
$(r \oplus s) \oplus t
=
r \oplus (s \oplus t)$
for all $r,s,t$ in $R$
\sidebar{
$(i + j) + k = i + (j + k)$
for integers $i,j,k$ ...
}
\item
$r \oplus 0_R = r = 0_R \oplus r$ 
for all
$r$ in $R$
\sidebar{
$i + 0 = i = 0 + i$
for integer $i$ ...
}
\item For 
\sidebar{
If $i$ is an integer, then $-i$ is also an integer and 
$i + (-i)= 0 =$
$(-i)+ i$.
}
all $r$ in $R$ there is something in $R$ which we will call $r'$
such that $r \oplus r' =$ $0_R = r' \oplus r$
\end{myenum}
For $\odot$ the properties are:
\begin{myenum}
\item $r \odot s$ is in $R$ for all $r, s$ in $R$
\sidebar{$ij = ji$ for integer $i,j$,...}
\item $(r \odot s) \odot t = r \odot (s \odot t)$ for all $r,s,t$ in $R$
\sidebar{
$(ij)k = i(jk)$ for integers $i,j,k$
}
\item $r \odot 1_R = r = 1_R \odot r$ for all $r$ in $R$
\sidebar{
$i1 = i = 1i$ for integer $i$ ...
}
\item $r \odot s = s \odot r$ for all $r,s$ in $R$ 
\sidebar{
$ij = ji$ for integers $i,j$
}
\end{myenum}
The property involving both $\oplus$ and $\odot$ is
\begin{myenum}
\item 
$r \odot (s \oplus t) =$
$r \odot s \oplus r\odot t$
\sidebar{
$i(j + k) = ij + ik$ for integers $i,j,k$.
Phew! So $\Z$ is a commutative ring.
}
\end{myenum}

Just remember this: 
A ring is a set of things with addition and multiplication.
And when you're lost just think of the set of integers and its operations.

Now we'll be working with the alphabet a,b,...,z.
We'll call them by their new identities: $0,1,...,25$.
This is not exactly all of $\Z$.
The formulas for the encryption and decryption of Caesar's cipher
involves addition and
substraction.
What if we go beyond? 
What is 3 + 25 (i.e., d + z)?
No problem, we will take remainders mod 26.
\sidebar{In C\texttt{++}-speak, we take $\% 26$.}
So instead of 3 + 25 we think of (3 + 25) mod 26 instead.
Of course the remainders are 0, 1, ..., 25 which is exactly what we want. 

To indicate that we're only interested in remainders or more accurately, we ignore multiples of 26, we write
\begin{align*}
3 + 25 
&= 28 \\
&\equiv 2 \pmod{26}
\end{align*}
In general, if $x$ and $y$ are integers, we write
\[
x \equiv y \pmod{26}
\]
if 26 divides $x - y$.
It does not mean that $x$ is equal to $y$.
It means that $x$ and $y$ are the same if you ignore additive multiples of 26, i.e.,
\[
x \equiv y \pmod{26}
\]
is the same as saying
\[
  x = y + (\text{... some multiple of 26 ...})
\]
This is the same as saying the remainder when $x$ is divided by 26
is the same as the remainder when $y$ is divided by 26.

If two numbers differ by a multiple of 26, we say that they are
\defone{congruent} mod 26.

Note that
\begin{align*}
26 &\equiv 0 \pmod{26} \\   
27 &\equiv 1 \pmod{26} \\
  28 &\equiv 2 \pmod{26} \\
  ...
\end{align*}
and
\begin{align*}
-1 &\equiv 25 \pmod{26} \\   
-2 &\equiv 24 \pmod{26} \\
-3 &\equiv 23 \pmod{26} \\
  ...
\end{align*}
So in the mod 26 world, since you are ignoring multiples of 26,
\textit{in some sense} there are only 26 numbers:
\[
  0, 1, 2, 3, ..., 25
\]
I'll write $\Z/26$ for this world of 26 values.
Remember that in this world, you can write the symbol
\[
  28
\]
but this is the same as 2 in $\Z/26$:
\[
  28 \equiv 2 \pmod{26}
\]
Of course in $\Z$, these symbols, i.e. $28$ and $2$, are different.

Most of the algebraic rules involving $+, -, *, 0, 1$ applies when
working with integers mod 26.
For instance suppose 
\[
  x \equiv y \pmod{26}
\]
where $x$ and $y$ are integers
(i.e., $x$ differs from $y$ by a multiple of 26), then
\[
  x + z \equiv y + z\pmod{26}
\]
where $z$ is an integer.
Likewise
from
\[
  x \equiv y \pmod{26}
\]
we get
\[
  x z \equiv y z \pmod{26}
\]
It's also true that
\[
  0 + x \equiv x \pmod{26}
\]
and 
\[
  1 \cdot x \equiv x \pmod{26}
\]

To be more precise, $\Z/26$ is a commutative ring.
It's a finite commutative ring with 26 values.
Let me rewrite the axioms for a commutative ring for $\Z/26$.

For $+$ on $\Z/26$, the properties are:
\begin{myenum}
\item $(r + s) \pmod{26}$ is in $\Z/26$ for all $r,s$ in $\Z/26$
\item 
  $(r + s) + t
  \equiv
  r + (s + t) \pmod{26}$
for all $r,s,t$ in $\Z/26$
\item
$r + 0 \equiv r \equiv 0 + r \pmod{26}$ 
for all
$r$ in $\Z/26$.
In fact $r'$ is just $(26 - r) \pmod{26}$.
For instance for $r = 2 \pmod{26}$, $r' = 26 - 2 = 24 \pmod{26}$.
\item For 
all $r$ in $\Z/26$ there is something in $\Z/26$ which we will call $r'$
such that $r + r' \equiv 0 \equiv r' \oplus r \pmod{26}$
\end{myenum}

For $\cdot$ the properties are:
\begin{myenum}
\item $r \cdot s \pmod{26}$ is in $\Z/26$ for all $r, s$ in $\Z/26$
\item $(r \cdot s) \cdot t \equiv r \cdot (s \cdot t) \pmod{26}$ for all $r,s,t$ in $\Z/26$
\item $r \cdot 1 \equiv r \equiv 1 \cdot r \pmod{26}$ for all $r$ in $\Z/26$
\item $r \cdot s \equiv s \cdot r \pmod{26}$ for all $r,s$ in $\Z/26$ 
\end{myenum}
The property involving both $+$ and $\cdot$ is
\begin{myenum}
\item 
$r \cdot (s + t) \equiv$
$r \cdot s + r \cdot t \pmod{26}$
\end{myenum}

It's also not too surprising that you can talk about mod $n$ for any
positive integer $n$.

If I don't say so, when I want you write some $x \pmod{26}$, I mean the $x$
such that $0 \leq x < 26$.
Of course there's no difference in mod 26 between 2 and 28.
But in mod 26, the values in $[0, 26)$ is the preferred range.
Also, when I say simplify $27 \pmod{26}$, I mean $1 \pmod{26}$.

Note that since $\Z/26$ is \textit{finite}, you can always solve
any equation in mod 26.
This is similar to boolean values: there are only two.
To solve a boolean equation, you just try all possible boolean values.
So to solve a $\Z/26$ equation, you just try all the 26 possible
values on all the variables that appear in the equation.


\begin{ex} 
  \label{ex:semigroup-associativity-0}
  \tinysidebar{\debug{exercises/{exercise-12/question.tex}}}
\mbox{}
  \begin{myenum}
  \item
    Solve
    \[
      5x^2 + y^2 = 3
    \]
    % mod 5, squares = 0^2=0, 1^2=1, 2^2=4, 3^2=4, 4^2=1
    (HINT: You don't really need number theory for this one. Why?
    But if you want to, imitate the solution for the previous
    problem.)
  \item
    Solve
    \[
      11y^2 - 5x^2 = 3
    \]
    % mod 5, squares = 0,1,4
    (HINT: This is just a slight change from the
    previous problem. \textit{But} now you need number theory. Try mod 4.
    If it does not work, try mod 5. Repeat.)
    % 3x^2 + y^2 = 3
    % {0,3} + {0,1} = 3
    % 0, 1, 3, 0 = 3
    % So x = 1, y = 0 (4)
    %
    % y^2=3 (5)
  \item
    Solve
    \[
      y^2 - 5x^2 = 2
    \]
    % mod 5, squares = 0^2=0, 1^2=1, 2^2=4, 3^2=4, 4^2=1    

  \item
    What about this one:
    \[
      x^2 - 5y^2 = 1
    \]    
  \end{myenum}
  {\scriptsize
[\textsc{Aside.}
Integer solutions to $x^2 - dy^2 = 1$ has been studied since at least 400BC.
This equation appear the Cattle Problem of Archimedes:

\begin{itemize}
  \item[]
If thou art diligent and wise, O stranger, compute the number of
cattle of the Sun, who once upon a time grazed on the fields of the
Thrinacian isle of Sicily, divided into four herds of different colours,
one milk white, another a glossy black, the third yellow and the last
dappled. In each herd were bulls, mighty in number according to these
proportions: Understand, stranger, that the white bulls were equal to
a half and a third of the black together with the whole of the yellow,
while the black were equal to the fourth part of the dappled and
a fifth, together with, once more, the whole of the yellow. Observe
further that the remaining bulls, the dappled, were equal to a sixth
part of the white and a seventh, together with all the yellow. These
were the proportions of the cows: The white were precisely equal to the
third part and a fourth of the whole herd of the black; while the black
were equal to the fourth part once more of the dappled and with it a
fifth part, when all, including the bulls went to pasture together. Now
the dappled in four parts8 were equal in number to a fifth part and a
sixth of the yellow herd. Finally the yellow were in number equal to
a sixth part and a seventh of the white herd. If thou canst accurately
tell, O stranger, the number of cattle of the Sun, giving separately the
number of well-fed bulls and again the number of females according
to each colour, thou wouldst not be called unskilled or ignorant of
numbers, but not yet shall thou be numbered among the wise. But
come, understand also all these conditions regarding the cows of the
Sun. When the white bulls mingled their number with the black, they
stood firm, equal in depth and breadth, and the plains of Thrinacia,
stretching far in all ways, were filled with their multitude. Again,
when the yellow and the dappled bulls were gathered into one herd
they stood in such a manner that their number, beginning from one,
grew slowly greater till it completed a triangular figure, there being
no bulls of other colours in their midst nor none of them lacking.
If thou art able, O stranger, to find out all these things and gather
them together in your mind, giving all the relations, thou shalt depart
crowned with glory and knowing that thou hast been adjudged perfect
in this species of wisdom.
\end{itemize}

If $W,X,Y,Z$ represents the number of white, black, yellow,
dappled bulls, you will get 
a systems of 7 linear equations, the first two being 
\begin{align*}
  W &= (1/2 + 1/3)X + Z \\
  X &= (1/4 + 1/5)Y + Z
\end{align*}
together with some constraints such as $W + X$ must be a square.
After some manipulations, it can be shown that the equation to solve looks like
\[
  x^2 - 410286423278424 y^2 = 1
\]
What was Archimedes thinking? You are find information on the Archimedes Cattle Problem on the web.]
}


  \solutionlink{sol:semigroup-associativity-0}
  \qed
\end{ex} 
\begin{python0}
from solutions import *
add(label="ex:semigroup-associativity-0",
    srcfilename='exercises/semigroup-associativity-0/answer.tex') 
\end{python0}


\begin{ex} 
  \label{ex:semigroup-associativity-0}
  \tinysidebar{\debug{exercises/{exercise-12/question.tex}}}
\mbox{}
  \begin{myenum}
  \item
    Solve
    \[
      5x^2 + y^2 = 3
    \]
    % mod 5, squares = 0^2=0, 1^2=1, 2^2=4, 3^2=4, 4^2=1
    (HINT: You don't really need number theory for this one. Why?
    But if you want to, imitate the solution for the previous
    problem.)
  \item
    Solve
    \[
      11y^2 - 5x^2 = 3
    \]
    % mod 5, squares = 0,1,4
    (HINT: This is just a slight change from the
    previous problem. \textit{But} now you need number theory. Try mod 4.
    If it does not work, try mod 5. Repeat.)
    % 3x^2 + y^2 = 3
    % {0,3} + {0,1} = 3
    % 0, 1, 3, 0 = 3
    % So x = 1, y = 0 (4)
    %
    % y^2=3 (5)
  \item
    Solve
    \[
      y^2 - 5x^2 = 2
    \]
    % mod 5, squares = 0^2=0, 1^2=1, 2^2=4, 3^2=4, 4^2=1    

  \item
    What about this one:
    \[
      x^2 - 5y^2 = 1
    \]    
  \end{myenum}
  {\scriptsize
[\textsc{Aside.}
Integer solutions to $x^2 - dy^2 = 1$ has been studied since at least 400BC.
This equation appear the Cattle Problem of Archimedes:

\begin{itemize}
  \item[]
If thou art diligent and wise, O stranger, compute the number of
cattle of the Sun, who once upon a time grazed on the fields of the
Thrinacian isle of Sicily, divided into four herds of different colours,
one milk white, another a glossy black, the third yellow and the last
dappled. In each herd were bulls, mighty in number according to these
proportions: Understand, stranger, that the white bulls were equal to
a half and a third of the black together with the whole of the yellow,
while the black were equal to the fourth part of the dappled and
a fifth, together with, once more, the whole of the yellow. Observe
further that the remaining bulls, the dappled, were equal to a sixth
part of the white and a seventh, together with all the yellow. These
were the proportions of the cows: The white were precisely equal to the
third part and a fourth of the whole herd of the black; while the black
were equal to the fourth part once more of the dappled and with it a
fifth part, when all, including the bulls went to pasture together. Now
the dappled in four parts8 were equal in number to a fifth part and a
sixth of the yellow herd. Finally the yellow were in number equal to
a sixth part and a seventh of the white herd. If thou canst accurately
tell, O stranger, the number of cattle of the Sun, giving separately the
number of well-fed bulls and again the number of females according
to each colour, thou wouldst not be called unskilled or ignorant of
numbers, but not yet shall thou be numbered among the wise. But
come, understand also all these conditions regarding the cows of the
Sun. When the white bulls mingled their number with the black, they
stood firm, equal in depth and breadth, and the plains of Thrinacia,
stretching far in all ways, were filled with their multitude. Again,
when the yellow and the dappled bulls were gathered into one herd
they stood in such a manner that their number, beginning from one,
grew slowly greater till it completed a triangular figure, there being
no bulls of other colours in their midst nor none of them lacking.
If thou art able, O stranger, to find out all these things and gather
them together in your mind, giving all the relations, thou shalt depart
crowned with glory and knowing that thou hast been adjudged perfect
in this species of wisdom.
\end{itemize}

If $W,X,Y,Z$ represents the number of white, black, yellow,
dappled bulls, you will get 
a systems of 7 linear equations, the first two being 
\begin{align*}
  W &= (1/2 + 1/3)X + Z \\
  X &= (1/4 + 1/5)Y + Z
\end{align*}
together with some constraints such as $W + X$ must be a square.
After some manipulations, it can be shown that the equation to solve looks like
\[
  x^2 - 410286423278424 y^2 = 1
\]
What was Archimedes thinking? You are find information on the Archimedes Cattle Problem on the web.]
}


  \solutionlink{sol:semigroup-associativity-0}
  \qed
\end{ex} 
\begin{python0}
from solutions import *
add(label="ex:semigroup-associativity-0",
    srcfilename='exercises/semigroup-associativity-0/answer.tex') 
\end{python0}


\begin{ex} 
  \label{ex:semigroup-associativity-0}
  \tinysidebar{\debug{exercises/{exercise-12/question.tex}}}
\mbox{}
  \begin{myenum}
  \item
    Solve
    \[
      5x^2 + y^2 = 3
    \]
    % mod 5, squares = 0^2=0, 1^2=1, 2^2=4, 3^2=4, 4^2=1
    (HINT: You don't really need number theory for this one. Why?
    But if you want to, imitate the solution for the previous
    problem.)
  \item
    Solve
    \[
      11y^2 - 5x^2 = 3
    \]
    % mod 5, squares = 0,1,4
    (HINT: This is just a slight change from the
    previous problem. \textit{But} now you need number theory. Try mod 4.
    If it does not work, try mod 5. Repeat.)
    % 3x^2 + y^2 = 3
    % {0,3} + {0,1} = 3
    % 0, 1, 3, 0 = 3
    % So x = 1, y = 0 (4)
    %
    % y^2=3 (5)
  \item
    Solve
    \[
      y^2 - 5x^2 = 2
    \]
    % mod 5, squares = 0^2=0, 1^2=1, 2^2=4, 3^2=4, 4^2=1    

  \item
    What about this one:
    \[
      x^2 - 5y^2 = 1
    \]    
  \end{myenum}
  {\scriptsize
[\textsc{Aside.}
Integer solutions to $x^2 - dy^2 = 1$ has been studied since at least 400BC.
This equation appear the Cattle Problem of Archimedes:

\begin{itemize}
  \item[]
If thou art diligent and wise, O stranger, compute the number of
cattle of the Sun, who once upon a time grazed on the fields of the
Thrinacian isle of Sicily, divided into four herds of different colours,
one milk white, another a glossy black, the third yellow and the last
dappled. In each herd were bulls, mighty in number according to these
proportions: Understand, stranger, that the white bulls were equal to
a half and a third of the black together with the whole of the yellow,
while the black were equal to the fourth part of the dappled and
a fifth, together with, once more, the whole of the yellow. Observe
further that the remaining bulls, the dappled, were equal to a sixth
part of the white and a seventh, together with all the yellow. These
were the proportions of the cows: The white were precisely equal to the
third part and a fourth of the whole herd of the black; while the black
were equal to the fourth part once more of the dappled and with it a
fifth part, when all, including the bulls went to pasture together. Now
the dappled in four parts8 were equal in number to a fifth part and a
sixth of the yellow herd. Finally the yellow were in number equal to
a sixth part and a seventh of the white herd. If thou canst accurately
tell, O stranger, the number of cattle of the Sun, giving separately the
number of well-fed bulls and again the number of females according
to each colour, thou wouldst not be called unskilled or ignorant of
numbers, but not yet shall thou be numbered among the wise. But
come, understand also all these conditions regarding the cows of the
Sun. When the white bulls mingled their number with the black, they
stood firm, equal in depth and breadth, and the plains of Thrinacia,
stretching far in all ways, were filled with their multitude. Again,
when the yellow and the dappled bulls were gathered into one herd
they stood in such a manner that their number, beginning from one,
grew slowly greater till it completed a triangular figure, there being
no bulls of other colours in their midst nor none of them lacking.
If thou art able, O stranger, to find out all these things and gather
them together in your mind, giving all the relations, thou shalt depart
crowned with glory and knowing that thou hast been adjudged perfect
in this species of wisdom.
\end{itemize}

If $W,X,Y,Z$ represents the number of white, black, yellow,
dappled bulls, you will get 
a systems of 7 linear equations, the first two being 
\begin{align*}
  W &= (1/2 + 1/3)X + Z \\
  X &= (1/4 + 1/5)Y + Z
\end{align*}
together with some constraints such as $W + X$ must be a square.
After some manipulations, it can be shown that the equation to solve looks like
\[
  x^2 - 410286423278424 y^2 = 1
\]
What was Archimedes thinking? You are find information on the Archimedes Cattle Problem on the web.]
}


  \solutionlink{sol:semigroup-associativity-0}
  \qed
\end{ex} 
\begin{python0}
from solutions import *
add(label="ex:semigroup-associativity-0",
    srcfilename='exercises/semigroup-associativity-0/answer.tex') 
\end{python0}


\begin{ex} 
  \label{ex:semigroup-associativity-0}
  \tinysidebar{\debug{exercises/{exercise-12/question.tex}}}
\mbox{}
  \begin{myenum}
  \item
    Solve
    \[
      5x^2 + y^2 = 3
    \]
    % mod 5, squares = 0^2=0, 1^2=1, 2^2=4, 3^2=4, 4^2=1
    (HINT: You don't really need number theory for this one. Why?
    But if you want to, imitate the solution for the previous
    problem.)
  \item
    Solve
    \[
      11y^2 - 5x^2 = 3
    \]
    % mod 5, squares = 0,1,4
    (HINT: This is just a slight change from the
    previous problem. \textit{But} now you need number theory. Try mod 4.
    If it does not work, try mod 5. Repeat.)
    % 3x^2 + y^2 = 3
    % {0,3} + {0,1} = 3
    % 0, 1, 3, 0 = 3
    % So x = 1, y = 0 (4)
    %
    % y^2=3 (5)
  \item
    Solve
    \[
      y^2 - 5x^2 = 2
    \]
    % mod 5, squares = 0^2=0, 1^2=1, 2^2=4, 3^2=4, 4^2=1    

  \item
    What about this one:
    \[
      x^2 - 5y^2 = 1
    \]    
  \end{myenum}
  {\scriptsize
[\textsc{Aside.}
Integer solutions to $x^2 - dy^2 = 1$ has been studied since at least 400BC.
This equation appear the Cattle Problem of Archimedes:

\begin{itemize}
  \item[]
If thou art diligent and wise, O stranger, compute the number of
cattle of the Sun, who once upon a time grazed on the fields of the
Thrinacian isle of Sicily, divided into four herds of different colours,
one milk white, another a glossy black, the third yellow and the last
dappled. In each herd were bulls, mighty in number according to these
proportions: Understand, stranger, that the white bulls were equal to
a half and a third of the black together with the whole of the yellow,
while the black were equal to the fourth part of the dappled and
a fifth, together with, once more, the whole of the yellow. Observe
further that the remaining bulls, the dappled, were equal to a sixth
part of the white and a seventh, together with all the yellow. These
were the proportions of the cows: The white were precisely equal to the
third part and a fourth of the whole herd of the black; while the black
were equal to the fourth part once more of the dappled and with it a
fifth part, when all, including the bulls went to pasture together. Now
the dappled in four parts8 were equal in number to a fifth part and a
sixth of the yellow herd. Finally the yellow were in number equal to
a sixth part and a seventh of the white herd. If thou canst accurately
tell, O stranger, the number of cattle of the Sun, giving separately the
number of well-fed bulls and again the number of females according
to each colour, thou wouldst not be called unskilled or ignorant of
numbers, but not yet shall thou be numbered among the wise. But
come, understand also all these conditions regarding the cows of the
Sun. When the white bulls mingled their number with the black, they
stood firm, equal in depth and breadth, and the plains of Thrinacia,
stretching far in all ways, were filled with their multitude. Again,
when the yellow and the dappled bulls were gathered into one herd
they stood in such a manner that their number, beginning from one,
grew slowly greater till it completed a triangular figure, there being
no bulls of other colours in their midst nor none of them lacking.
If thou art able, O stranger, to find out all these things and gather
them together in your mind, giving all the relations, thou shalt depart
crowned with glory and knowing that thou hast been adjudged perfect
in this species of wisdom.
\end{itemize}

If $W,X,Y,Z$ represents the number of white, black, yellow,
dappled bulls, you will get 
a systems of 7 linear equations, the first two being 
\begin{align*}
  W &= (1/2 + 1/3)X + Z \\
  X &= (1/4 + 1/5)Y + Z
\end{align*}
together with some constraints such as $W + X$ must be a square.
After some manipulations, it can be shown that the equation to solve looks like
\[
  x^2 - 410286423278424 y^2 = 1
\]
What was Archimedes thinking? You are find information on the Archimedes Cattle Problem on the web.]
}


  \solutionlink{sol:semigroup-associativity-0}
  \qed
\end{ex} 
\begin{python0}
from solutions import *
add(label="ex:semigroup-associativity-0",
    srcfilename='exercises/semigroup-associativity-0/answer.tex') 
\end{python0}


\begin{ex} 
  \label{ex:semigroup-associativity-0}
  \tinysidebar{\debug{exercises/{exercise-12/question.tex}}}
\mbox{}
  \begin{myenum}
  \item
    Solve
    \[
      5x^2 + y^2 = 3
    \]
    % mod 5, squares = 0^2=0, 1^2=1, 2^2=4, 3^2=4, 4^2=1
    (HINT: You don't really need number theory for this one. Why?
    But if you want to, imitate the solution for the previous
    problem.)
  \item
    Solve
    \[
      11y^2 - 5x^2 = 3
    \]
    % mod 5, squares = 0,1,4
    (HINT: This is just a slight change from the
    previous problem. \textit{But} now you need number theory. Try mod 4.
    If it does not work, try mod 5. Repeat.)
    % 3x^2 + y^2 = 3
    % {0,3} + {0,1} = 3
    % 0, 1, 3, 0 = 3
    % So x = 1, y = 0 (4)
    %
    % y^2=3 (5)
  \item
    Solve
    \[
      y^2 - 5x^2 = 2
    \]
    % mod 5, squares = 0^2=0, 1^2=1, 2^2=4, 3^2=4, 4^2=1    

  \item
    What about this one:
    \[
      x^2 - 5y^2 = 1
    \]    
  \end{myenum}
  {\scriptsize
[\textsc{Aside.}
Integer solutions to $x^2 - dy^2 = 1$ has been studied since at least 400BC.
This equation appear the Cattle Problem of Archimedes:

\begin{itemize}
  \item[]
If thou art diligent and wise, O stranger, compute the number of
cattle of the Sun, who once upon a time grazed on the fields of the
Thrinacian isle of Sicily, divided into four herds of different colours,
one milk white, another a glossy black, the third yellow and the last
dappled. In each herd were bulls, mighty in number according to these
proportions: Understand, stranger, that the white bulls were equal to
a half and a third of the black together with the whole of the yellow,
while the black were equal to the fourth part of the dappled and
a fifth, together with, once more, the whole of the yellow. Observe
further that the remaining bulls, the dappled, were equal to a sixth
part of the white and a seventh, together with all the yellow. These
were the proportions of the cows: The white were precisely equal to the
third part and a fourth of the whole herd of the black; while the black
were equal to the fourth part once more of the dappled and with it a
fifth part, when all, including the bulls went to pasture together. Now
the dappled in four parts8 were equal in number to a fifth part and a
sixth of the yellow herd. Finally the yellow were in number equal to
a sixth part and a seventh of the white herd. If thou canst accurately
tell, O stranger, the number of cattle of the Sun, giving separately the
number of well-fed bulls and again the number of females according
to each colour, thou wouldst not be called unskilled or ignorant of
numbers, but not yet shall thou be numbered among the wise. But
come, understand also all these conditions regarding the cows of the
Sun. When the white bulls mingled their number with the black, they
stood firm, equal in depth and breadth, and the plains of Thrinacia,
stretching far in all ways, were filled with their multitude. Again,
when the yellow and the dappled bulls were gathered into one herd
they stood in such a manner that their number, beginning from one,
grew slowly greater till it completed a triangular figure, there being
no bulls of other colours in their midst nor none of them lacking.
If thou art able, O stranger, to find out all these things and gather
them together in your mind, giving all the relations, thou shalt depart
crowned with glory and knowing that thou hast been adjudged perfect
in this species of wisdom.
\end{itemize}

If $W,X,Y,Z$ represents the number of white, black, yellow,
dappled bulls, you will get 
a systems of 7 linear equations, the first two being 
\begin{align*}
  W &= (1/2 + 1/3)X + Z \\
  X &= (1/4 + 1/5)Y + Z
\end{align*}
together with some constraints such as $W + X$ must be a square.
After some manipulations, it can be shown that the equation to solve looks like
\[
  x^2 - 410286423278424 y^2 = 1
\]
What was Archimedes thinking? You are find information on the Archimedes Cattle Problem on the web.]
}


  \solutionlink{sol:semigroup-associativity-0}
  \qed
\end{ex} 
\begin{python0}
from solutions import *
add(label="ex:semigroup-associativity-0",
    srcfilename='exercises/semigroup-associativity-0/answer.tex') 
\end{python0}


\begin{ex} 
  \label{ex:semigroup-associativity-0}
  \tinysidebar{\debug{exercises/{exercise-12/question.tex}}}
\mbox{}
  \begin{myenum}
  \item
    Solve
    \[
      5x^2 + y^2 = 3
    \]
    % mod 5, squares = 0^2=0, 1^2=1, 2^2=4, 3^2=4, 4^2=1
    (HINT: You don't really need number theory for this one. Why?
    But if you want to, imitate the solution for the previous
    problem.)
  \item
    Solve
    \[
      11y^2 - 5x^2 = 3
    \]
    % mod 5, squares = 0,1,4
    (HINT: This is just a slight change from the
    previous problem. \textit{But} now you need number theory. Try mod 4.
    If it does not work, try mod 5. Repeat.)
    % 3x^2 + y^2 = 3
    % {0,3} + {0,1} = 3
    % 0, 1, 3, 0 = 3
    % So x = 1, y = 0 (4)
    %
    % y^2=3 (5)
  \item
    Solve
    \[
      y^2 - 5x^2 = 2
    \]
    % mod 5, squares = 0^2=0, 1^2=1, 2^2=4, 3^2=4, 4^2=1    

  \item
    What about this one:
    \[
      x^2 - 5y^2 = 1
    \]    
  \end{myenum}
  {\scriptsize
[\textsc{Aside.}
Integer solutions to $x^2 - dy^2 = 1$ has been studied since at least 400BC.
This equation appear the Cattle Problem of Archimedes:

\begin{itemize}
  \item[]
If thou art diligent and wise, O stranger, compute the number of
cattle of the Sun, who once upon a time grazed on the fields of the
Thrinacian isle of Sicily, divided into four herds of different colours,
one milk white, another a glossy black, the third yellow and the last
dappled. In each herd were bulls, mighty in number according to these
proportions: Understand, stranger, that the white bulls were equal to
a half and a third of the black together with the whole of the yellow,
while the black were equal to the fourth part of the dappled and
a fifth, together with, once more, the whole of the yellow. Observe
further that the remaining bulls, the dappled, were equal to a sixth
part of the white and a seventh, together with all the yellow. These
were the proportions of the cows: The white were precisely equal to the
third part and a fourth of the whole herd of the black; while the black
were equal to the fourth part once more of the dappled and with it a
fifth part, when all, including the bulls went to pasture together. Now
the dappled in four parts8 were equal in number to a fifth part and a
sixth of the yellow herd. Finally the yellow were in number equal to
a sixth part and a seventh of the white herd. If thou canst accurately
tell, O stranger, the number of cattle of the Sun, giving separately the
number of well-fed bulls and again the number of females according
to each colour, thou wouldst not be called unskilled or ignorant of
numbers, but not yet shall thou be numbered among the wise. But
come, understand also all these conditions regarding the cows of the
Sun. When the white bulls mingled their number with the black, they
stood firm, equal in depth and breadth, and the plains of Thrinacia,
stretching far in all ways, were filled with their multitude. Again,
when the yellow and the dappled bulls were gathered into one herd
they stood in such a manner that their number, beginning from one,
grew slowly greater till it completed a triangular figure, there being
no bulls of other colours in their midst nor none of them lacking.
If thou art able, O stranger, to find out all these things and gather
them together in your mind, giving all the relations, thou shalt depart
crowned with glory and knowing that thou hast been adjudged perfect
in this species of wisdom.
\end{itemize}

If $W,X,Y,Z$ represents the number of white, black, yellow,
dappled bulls, you will get 
a systems of 7 linear equations, the first two being 
\begin{align*}
  W &= (1/2 + 1/3)X + Z \\
  X &= (1/4 + 1/5)Y + Z
\end{align*}
together with some constraints such as $W + X$ must be a square.
After some manipulations, it can be shown that the equation to solve looks like
\[
  x^2 - 410286423278424 y^2 = 1
\]
What was Archimedes thinking? You are find information on the Archimedes Cattle Problem on the web.]
}


  \solutionlink{sol:semigroup-associativity-0}
  \qed
\end{ex} 
\begin{python0}
from solutions import *
add(label="ex:semigroup-associativity-0",
    srcfilename='exercises/semigroup-associativity-0/answer.tex') 
\end{python0}


\begin{ex} 
  \label{ex:semigroup-associativity-0}
  \tinysidebar{\debug{exercises/{exercise-12/question.tex}}}
\mbox{}
  \begin{myenum}
  \item
    Solve
    \[
      5x^2 + y^2 = 3
    \]
    % mod 5, squares = 0^2=0, 1^2=1, 2^2=4, 3^2=4, 4^2=1
    (HINT: You don't really need number theory for this one. Why?
    But if you want to, imitate the solution for the previous
    problem.)
  \item
    Solve
    \[
      11y^2 - 5x^2 = 3
    \]
    % mod 5, squares = 0,1,4
    (HINT: This is just a slight change from the
    previous problem. \textit{But} now you need number theory. Try mod 4.
    If it does not work, try mod 5. Repeat.)
    % 3x^2 + y^2 = 3
    % {0,3} + {0,1} = 3
    % 0, 1, 3, 0 = 3
    % So x = 1, y = 0 (4)
    %
    % y^2=3 (5)
  \item
    Solve
    \[
      y^2 - 5x^2 = 2
    \]
    % mod 5, squares = 0^2=0, 1^2=1, 2^2=4, 3^2=4, 4^2=1    

  \item
    What about this one:
    \[
      x^2 - 5y^2 = 1
    \]    
  \end{myenum}
  {\scriptsize
[\textsc{Aside.}
Integer solutions to $x^2 - dy^2 = 1$ has been studied since at least 400BC.
This equation appear the Cattle Problem of Archimedes:

\begin{itemize}
  \item[]
If thou art diligent and wise, O stranger, compute the number of
cattle of the Sun, who once upon a time grazed on the fields of the
Thrinacian isle of Sicily, divided into four herds of different colours,
one milk white, another a glossy black, the third yellow and the last
dappled. In each herd were bulls, mighty in number according to these
proportions: Understand, stranger, that the white bulls were equal to
a half and a third of the black together with the whole of the yellow,
while the black were equal to the fourth part of the dappled and
a fifth, together with, once more, the whole of the yellow. Observe
further that the remaining bulls, the dappled, were equal to a sixth
part of the white and a seventh, together with all the yellow. These
were the proportions of the cows: The white were precisely equal to the
third part and a fourth of the whole herd of the black; while the black
were equal to the fourth part once more of the dappled and with it a
fifth part, when all, including the bulls went to pasture together. Now
the dappled in four parts8 were equal in number to a fifth part and a
sixth of the yellow herd. Finally the yellow were in number equal to
a sixth part and a seventh of the white herd. If thou canst accurately
tell, O stranger, the number of cattle of the Sun, giving separately the
number of well-fed bulls and again the number of females according
to each colour, thou wouldst not be called unskilled or ignorant of
numbers, but not yet shall thou be numbered among the wise. But
come, understand also all these conditions regarding the cows of the
Sun. When the white bulls mingled their number with the black, they
stood firm, equal in depth and breadth, and the plains of Thrinacia,
stretching far in all ways, were filled with their multitude. Again,
when the yellow and the dappled bulls were gathered into one herd
they stood in such a manner that their number, beginning from one,
grew slowly greater till it completed a triangular figure, there being
no bulls of other colours in their midst nor none of them lacking.
If thou art able, O stranger, to find out all these things and gather
them together in your mind, giving all the relations, thou shalt depart
crowned with glory and knowing that thou hast been adjudged perfect
in this species of wisdom.
\end{itemize}

If $W,X,Y,Z$ represents the number of white, black, yellow,
dappled bulls, you will get 
a systems of 7 linear equations, the first two being 
\begin{align*}
  W &= (1/2 + 1/3)X + Z \\
  X &= (1/4 + 1/5)Y + Z
\end{align*}
together with some constraints such as $W + X$ must be a square.
After some manipulations, it can be shown that the equation to solve looks like
\[
  x^2 - 410286423278424 y^2 = 1
\]
What was Archimedes thinking? You are find information on the Archimedes Cattle Problem on the web.]
}


  \solutionlink{sol:semigroup-associativity-0}
  \qed
\end{ex} 
\begin{python0}
from solutions import *
add(label="ex:semigroup-associativity-0",
    srcfilename='exercises/semigroup-associativity-0/answer.tex') 
\end{python0}


\begin{ex} 
  \label{ex:semigroup-associativity-0}
  \tinysidebar{\debug{exercises/{exercise-12/question.tex}}}
\mbox{}
  \begin{myenum}
  \item
    Solve
    \[
      5x^2 + y^2 = 3
    \]
    % mod 5, squares = 0^2=0, 1^2=1, 2^2=4, 3^2=4, 4^2=1
    (HINT: You don't really need number theory for this one. Why?
    But if you want to, imitate the solution for the previous
    problem.)
  \item
    Solve
    \[
      11y^2 - 5x^2 = 3
    \]
    % mod 5, squares = 0,1,4
    (HINT: This is just a slight change from the
    previous problem. \textit{But} now you need number theory. Try mod 4.
    If it does not work, try mod 5. Repeat.)
    % 3x^2 + y^2 = 3
    % {0,3} + {0,1} = 3
    % 0, 1, 3, 0 = 3
    % So x = 1, y = 0 (4)
    %
    % y^2=3 (5)
  \item
    Solve
    \[
      y^2 - 5x^2 = 2
    \]
    % mod 5, squares = 0^2=0, 1^2=1, 2^2=4, 3^2=4, 4^2=1    

  \item
    What about this one:
    \[
      x^2 - 5y^2 = 1
    \]    
  \end{myenum}
  {\scriptsize
[\textsc{Aside.}
Integer solutions to $x^2 - dy^2 = 1$ has been studied since at least 400BC.
This equation appear the Cattle Problem of Archimedes:

\begin{itemize}
  \item[]
If thou art diligent and wise, O stranger, compute the number of
cattle of the Sun, who once upon a time grazed on the fields of the
Thrinacian isle of Sicily, divided into four herds of different colours,
one milk white, another a glossy black, the third yellow and the last
dappled. In each herd were bulls, mighty in number according to these
proportions: Understand, stranger, that the white bulls were equal to
a half and a third of the black together with the whole of the yellow,
while the black were equal to the fourth part of the dappled and
a fifth, together with, once more, the whole of the yellow. Observe
further that the remaining bulls, the dappled, were equal to a sixth
part of the white and a seventh, together with all the yellow. These
were the proportions of the cows: The white were precisely equal to the
third part and a fourth of the whole herd of the black; while the black
were equal to the fourth part once more of the dappled and with it a
fifth part, when all, including the bulls went to pasture together. Now
the dappled in four parts8 were equal in number to a fifth part and a
sixth of the yellow herd. Finally the yellow were in number equal to
a sixth part and a seventh of the white herd. If thou canst accurately
tell, O stranger, the number of cattle of the Sun, giving separately the
number of well-fed bulls and again the number of females according
to each colour, thou wouldst not be called unskilled or ignorant of
numbers, but not yet shall thou be numbered among the wise. But
come, understand also all these conditions regarding the cows of the
Sun. When the white bulls mingled their number with the black, they
stood firm, equal in depth and breadth, and the plains of Thrinacia,
stretching far in all ways, were filled with their multitude. Again,
when the yellow and the dappled bulls were gathered into one herd
they stood in such a manner that their number, beginning from one,
grew slowly greater till it completed a triangular figure, there being
no bulls of other colours in their midst nor none of them lacking.
If thou art able, O stranger, to find out all these things and gather
them together in your mind, giving all the relations, thou shalt depart
crowned with glory and knowing that thou hast been adjudged perfect
in this species of wisdom.
\end{itemize}

If $W,X,Y,Z$ represents the number of white, black, yellow,
dappled bulls, you will get 
a systems of 7 linear equations, the first two being 
\begin{align*}
  W &= (1/2 + 1/3)X + Z \\
  X &= (1/4 + 1/5)Y + Z
\end{align*}
together with some constraints such as $W + X$ must be a square.
After some manipulations, it can be shown that the equation to solve looks like
\[
  x^2 - 410286423278424 y^2 = 1
\]
What was Archimedes thinking? You are find information on the Archimedes Cattle Problem on the web.]
}


  \solutionlink{sol:semigroup-associativity-0}
  \qed
\end{ex} 
\begin{python0}
from solutions import *
add(label="ex:semigroup-associativity-0",
    srcfilename='exercises/semigroup-associativity-0/answer.tex') 
\end{python0}


\begin{ex} 
  \label{ex:semigroup-associativity-0}
  \tinysidebar{\debug{exercises/{exercise-12/question.tex}}}
\mbox{}
  \begin{myenum}
  \item
    Solve
    \[
      5x^2 + y^2 = 3
    \]
    % mod 5, squares = 0^2=0, 1^2=1, 2^2=4, 3^2=4, 4^2=1
    (HINT: You don't really need number theory for this one. Why?
    But if you want to, imitate the solution for the previous
    problem.)
  \item
    Solve
    \[
      11y^2 - 5x^2 = 3
    \]
    % mod 5, squares = 0,1,4
    (HINT: This is just a slight change from the
    previous problem. \textit{But} now you need number theory. Try mod 4.
    If it does not work, try mod 5. Repeat.)
    % 3x^2 + y^2 = 3
    % {0,3} + {0,1} = 3
    % 0, 1, 3, 0 = 3
    % So x = 1, y = 0 (4)
    %
    % y^2=3 (5)
  \item
    Solve
    \[
      y^2 - 5x^2 = 2
    \]
    % mod 5, squares = 0^2=0, 1^2=1, 2^2=4, 3^2=4, 4^2=1    

  \item
    What about this one:
    \[
      x^2 - 5y^2 = 1
    \]    
  \end{myenum}
  {\scriptsize
[\textsc{Aside.}
Integer solutions to $x^2 - dy^2 = 1$ has been studied since at least 400BC.
This equation appear the Cattle Problem of Archimedes:

\begin{itemize}
  \item[]
If thou art diligent and wise, O stranger, compute the number of
cattle of the Sun, who once upon a time grazed on the fields of the
Thrinacian isle of Sicily, divided into four herds of different colours,
one milk white, another a glossy black, the third yellow and the last
dappled. In each herd were bulls, mighty in number according to these
proportions: Understand, stranger, that the white bulls were equal to
a half and a third of the black together with the whole of the yellow,
while the black were equal to the fourth part of the dappled and
a fifth, together with, once more, the whole of the yellow. Observe
further that the remaining bulls, the dappled, were equal to a sixth
part of the white and a seventh, together with all the yellow. These
were the proportions of the cows: The white were precisely equal to the
third part and a fourth of the whole herd of the black; while the black
were equal to the fourth part once more of the dappled and with it a
fifth part, when all, including the bulls went to pasture together. Now
the dappled in four parts8 were equal in number to a fifth part and a
sixth of the yellow herd. Finally the yellow were in number equal to
a sixth part and a seventh of the white herd. If thou canst accurately
tell, O stranger, the number of cattle of the Sun, giving separately the
number of well-fed bulls and again the number of females according
to each colour, thou wouldst not be called unskilled or ignorant of
numbers, but not yet shall thou be numbered among the wise. But
come, understand also all these conditions regarding the cows of the
Sun. When the white bulls mingled their number with the black, they
stood firm, equal in depth and breadth, and the plains of Thrinacia,
stretching far in all ways, were filled with their multitude. Again,
when the yellow and the dappled bulls were gathered into one herd
they stood in such a manner that their number, beginning from one,
grew slowly greater till it completed a triangular figure, there being
no bulls of other colours in their midst nor none of them lacking.
If thou art able, O stranger, to find out all these things and gather
them together in your mind, giving all the relations, thou shalt depart
crowned with glory and knowing that thou hast been adjudged perfect
in this species of wisdom.
\end{itemize}

If $W,X,Y,Z$ represents the number of white, black, yellow,
dappled bulls, you will get 
a systems of 7 linear equations, the first two being 
\begin{align*}
  W &= (1/2 + 1/3)X + Z \\
  X &= (1/4 + 1/5)Y + Z
\end{align*}
together with some constraints such as $W + X$ must be a square.
After some manipulations, it can be shown that the equation to solve looks like
\[
  x^2 - 410286423278424 y^2 = 1
\]
What was Archimedes thinking? You are find information on the Archimedes Cattle Problem on the web.]
}


  \solutionlink{sol:semigroup-associativity-0}
  \qed
\end{ex} 
\begin{python0}
from solutions import *
add(label="ex:semigroup-associativity-0",
    srcfilename='exercises/semigroup-associativity-0/answer.tex') 
\end{python0}


\begin{ex} 
  \label{ex:semigroup-associativity-0}
  \tinysidebar{\debug{exercises/{exercise-12/question.tex}}}
\mbox{}
  \begin{myenum}
  \item
    Solve
    \[
      5x^2 + y^2 = 3
    \]
    % mod 5, squares = 0^2=0, 1^2=1, 2^2=4, 3^2=4, 4^2=1
    (HINT: You don't really need number theory for this one. Why?
    But if you want to, imitate the solution for the previous
    problem.)
  \item
    Solve
    \[
      11y^2 - 5x^2 = 3
    \]
    % mod 5, squares = 0,1,4
    (HINT: This is just a slight change from the
    previous problem. \textit{But} now you need number theory. Try mod 4.
    If it does not work, try mod 5. Repeat.)
    % 3x^2 + y^2 = 3
    % {0,3} + {0,1} = 3
    % 0, 1, 3, 0 = 3
    % So x = 1, y = 0 (4)
    %
    % y^2=3 (5)
  \item
    Solve
    \[
      y^2 - 5x^2 = 2
    \]
    % mod 5, squares = 0^2=0, 1^2=1, 2^2=4, 3^2=4, 4^2=1    

  \item
    What about this one:
    \[
      x^2 - 5y^2 = 1
    \]    
  \end{myenum}
  {\scriptsize
[\textsc{Aside.}
Integer solutions to $x^2 - dy^2 = 1$ has been studied since at least 400BC.
This equation appear the Cattle Problem of Archimedes:

\begin{itemize}
  \item[]
If thou art diligent and wise, O stranger, compute the number of
cattle of the Sun, who once upon a time grazed on the fields of the
Thrinacian isle of Sicily, divided into four herds of different colours,
one milk white, another a glossy black, the third yellow and the last
dappled. In each herd were bulls, mighty in number according to these
proportions: Understand, stranger, that the white bulls were equal to
a half and a third of the black together with the whole of the yellow,
while the black were equal to the fourth part of the dappled and
a fifth, together with, once more, the whole of the yellow. Observe
further that the remaining bulls, the dappled, were equal to a sixth
part of the white and a seventh, together with all the yellow. These
were the proportions of the cows: The white were precisely equal to the
third part and a fourth of the whole herd of the black; while the black
were equal to the fourth part once more of the dappled and with it a
fifth part, when all, including the bulls went to pasture together. Now
the dappled in four parts8 were equal in number to a fifth part and a
sixth of the yellow herd. Finally the yellow were in number equal to
a sixth part and a seventh of the white herd. If thou canst accurately
tell, O stranger, the number of cattle of the Sun, giving separately the
number of well-fed bulls and again the number of females according
to each colour, thou wouldst not be called unskilled or ignorant of
numbers, but not yet shall thou be numbered among the wise. But
come, understand also all these conditions regarding the cows of the
Sun. When the white bulls mingled their number with the black, they
stood firm, equal in depth and breadth, and the plains of Thrinacia,
stretching far in all ways, were filled with their multitude. Again,
when the yellow and the dappled bulls were gathered into one herd
they stood in such a manner that their number, beginning from one,
grew slowly greater till it completed a triangular figure, there being
no bulls of other colours in their midst nor none of them lacking.
If thou art able, O stranger, to find out all these things and gather
them together in your mind, giving all the relations, thou shalt depart
crowned with glory and knowing that thou hast been adjudged perfect
in this species of wisdom.
\end{itemize}

If $W,X,Y,Z$ represents the number of white, black, yellow,
dappled bulls, you will get 
a systems of 7 linear equations, the first two being 
\begin{align*}
  W &= (1/2 + 1/3)X + Z \\
  X &= (1/4 + 1/5)Y + Z
\end{align*}
together with some constraints such as $W + X$ must be a square.
After some manipulations, it can be shown that the equation to solve looks like
\[
  x^2 - 410286423278424 y^2 = 1
\]
What was Archimedes thinking? You are find information on the Archimedes Cattle Problem on the web.]
}


  \solutionlink{sol:semigroup-associativity-0}
  \qed
\end{ex} 
\begin{python0}
from solutions import *
add(label="ex:semigroup-associativity-0",
    srcfilename='exercises/semigroup-associativity-0/answer.tex') 
\end{python0}


\begin{ex} 
  \label{ex:semigroup-associativity-0}
  \tinysidebar{\debug{exercises/{exercise-12/question.tex}}}
\mbox{}
  \begin{myenum}
  \item
    Solve
    \[
      5x^2 + y^2 = 3
    \]
    % mod 5, squares = 0^2=0, 1^2=1, 2^2=4, 3^2=4, 4^2=1
    (HINT: You don't really need number theory for this one. Why?
    But if you want to, imitate the solution for the previous
    problem.)
  \item
    Solve
    \[
      11y^2 - 5x^2 = 3
    \]
    % mod 5, squares = 0,1,4
    (HINT: This is just a slight change from the
    previous problem. \textit{But} now you need number theory. Try mod 4.
    If it does not work, try mod 5. Repeat.)
    % 3x^2 + y^2 = 3
    % {0,3} + {0,1} = 3
    % 0, 1, 3, 0 = 3
    % So x = 1, y = 0 (4)
    %
    % y^2=3 (5)
  \item
    Solve
    \[
      y^2 - 5x^2 = 2
    \]
    % mod 5, squares = 0^2=0, 1^2=1, 2^2=4, 3^2=4, 4^2=1    

  \item
    What about this one:
    \[
      x^2 - 5y^2 = 1
    \]    
  \end{myenum}
  {\scriptsize
[\textsc{Aside.}
Integer solutions to $x^2 - dy^2 = 1$ has been studied since at least 400BC.
This equation appear the Cattle Problem of Archimedes:

\begin{itemize}
  \item[]
If thou art diligent and wise, O stranger, compute the number of
cattle of the Sun, who once upon a time grazed on the fields of the
Thrinacian isle of Sicily, divided into four herds of different colours,
one milk white, another a glossy black, the third yellow and the last
dappled. In each herd were bulls, mighty in number according to these
proportions: Understand, stranger, that the white bulls were equal to
a half and a third of the black together with the whole of the yellow,
while the black were equal to the fourth part of the dappled and
a fifth, together with, once more, the whole of the yellow. Observe
further that the remaining bulls, the dappled, were equal to a sixth
part of the white and a seventh, together with all the yellow. These
were the proportions of the cows: The white were precisely equal to the
third part and a fourth of the whole herd of the black; while the black
were equal to the fourth part once more of the dappled and with it a
fifth part, when all, including the bulls went to pasture together. Now
the dappled in four parts8 were equal in number to a fifth part and a
sixth of the yellow herd. Finally the yellow were in number equal to
a sixth part and a seventh of the white herd. If thou canst accurately
tell, O stranger, the number of cattle of the Sun, giving separately the
number of well-fed bulls and again the number of females according
to each colour, thou wouldst not be called unskilled or ignorant of
numbers, but not yet shall thou be numbered among the wise. But
come, understand also all these conditions regarding the cows of the
Sun. When the white bulls mingled their number with the black, they
stood firm, equal in depth and breadth, and the plains of Thrinacia,
stretching far in all ways, were filled with their multitude. Again,
when the yellow and the dappled bulls were gathered into one herd
they stood in such a manner that their number, beginning from one,
grew slowly greater till it completed a triangular figure, there being
no bulls of other colours in their midst nor none of them lacking.
If thou art able, O stranger, to find out all these things and gather
them together in your mind, giving all the relations, thou shalt depart
crowned with glory and knowing that thou hast been adjudged perfect
in this species of wisdom.
\end{itemize}

If $W,X,Y,Z$ represents the number of white, black, yellow,
dappled bulls, you will get 
a systems of 7 linear equations, the first two being 
\begin{align*}
  W &= (1/2 + 1/3)X + Z \\
  X &= (1/4 + 1/5)Y + Z
\end{align*}
together with some constraints such as $W + X$ must be a square.
After some manipulations, it can be shown that the equation to solve looks like
\[
  x^2 - 410286423278424 y^2 = 1
\]
What was Archimedes thinking? You are find information on the Archimedes Cattle Problem on the web.]
}


  \solutionlink{sol:semigroup-associativity-0}
  \qed
\end{ex} 
\begin{python0}
from solutions import *
add(label="ex:semigroup-associativity-0",
    srcfilename='exercises/semigroup-associativity-0/answer.tex') 
\end{python0}


\begin{ex} 
  \label{ex:semigroup-associativity-0}
  \tinysidebar{\debug{exercises/{exercise-12/question.tex}}}
\mbox{}
  \begin{myenum}
  \item
    Solve
    \[
      5x^2 + y^2 = 3
    \]
    % mod 5, squares = 0^2=0, 1^2=1, 2^2=4, 3^2=4, 4^2=1
    (HINT: You don't really need number theory for this one. Why?
    But if you want to, imitate the solution for the previous
    problem.)
  \item
    Solve
    \[
      11y^2 - 5x^2 = 3
    \]
    % mod 5, squares = 0,1,4
    (HINT: This is just a slight change from the
    previous problem. \textit{But} now you need number theory. Try mod 4.
    If it does not work, try mod 5. Repeat.)
    % 3x^2 + y^2 = 3
    % {0,3} + {0,1} = 3
    % 0, 1, 3, 0 = 3
    % So x = 1, y = 0 (4)
    %
    % y^2=3 (5)
  \item
    Solve
    \[
      y^2 - 5x^2 = 2
    \]
    % mod 5, squares = 0^2=0, 1^2=1, 2^2=4, 3^2=4, 4^2=1    

  \item
    What about this one:
    \[
      x^2 - 5y^2 = 1
    \]    
  \end{myenum}
  {\scriptsize
[\textsc{Aside.}
Integer solutions to $x^2 - dy^2 = 1$ has been studied since at least 400BC.
This equation appear the Cattle Problem of Archimedes:

\begin{itemize}
  \item[]
If thou art diligent and wise, O stranger, compute the number of
cattle of the Sun, who once upon a time grazed on the fields of the
Thrinacian isle of Sicily, divided into four herds of different colours,
one milk white, another a glossy black, the third yellow and the last
dappled. In each herd were bulls, mighty in number according to these
proportions: Understand, stranger, that the white bulls were equal to
a half and a third of the black together with the whole of the yellow,
while the black were equal to the fourth part of the dappled and
a fifth, together with, once more, the whole of the yellow. Observe
further that the remaining bulls, the dappled, were equal to a sixth
part of the white and a seventh, together with all the yellow. These
were the proportions of the cows: The white were precisely equal to the
third part and a fourth of the whole herd of the black; while the black
were equal to the fourth part once more of the dappled and with it a
fifth part, when all, including the bulls went to pasture together. Now
the dappled in four parts8 were equal in number to a fifth part and a
sixth of the yellow herd. Finally the yellow were in number equal to
a sixth part and a seventh of the white herd. If thou canst accurately
tell, O stranger, the number of cattle of the Sun, giving separately the
number of well-fed bulls and again the number of females according
to each colour, thou wouldst not be called unskilled or ignorant of
numbers, but not yet shall thou be numbered among the wise. But
come, understand also all these conditions regarding the cows of the
Sun. When the white bulls mingled their number with the black, they
stood firm, equal in depth and breadth, and the plains of Thrinacia,
stretching far in all ways, were filled with their multitude. Again,
when the yellow and the dappled bulls were gathered into one herd
they stood in such a manner that their number, beginning from one,
grew slowly greater till it completed a triangular figure, there being
no bulls of other colours in their midst nor none of them lacking.
If thou art able, O stranger, to find out all these things and gather
them together in your mind, giving all the relations, thou shalt depart
crowned with glory and knowing that thou hast been adjudged perfect
in this species of wisdom.
\end{itemize}

If $W,X,Y,Z$ represents the number of white, black, yellow,
dappled bulls, you will get 
a systems of 7 linear equations, the first two being 
\begin{align*}
  W &= (1/2 + 1/3)X + Z \\
  X &= (1/4 + 1/5)Y + Z
\end{align*}
together with some constraints such as $W + X$ must be a square.
After some manipulations, it can be shown that the equation to solve looks like
\[
  x^2 - 410286423278424 y^2 = 1
\]
What was Archimedes thinking? You are find information on the Archimedes Cattle Problem on the web.]
}


  \solutionlink{sol:semigroup-associativity-0}
  \qed
\end{ex} 
\begin{python0}
from solutions import *
add(label="ex:semigroup-associativity-0",
    srcfilename='exercises/semigroup-associativity-0/answer.tex') 
\end{python0}


\begin{ex} 
  \label{ex:semigroup-associativity-0}
  \tinysidebar{\debug{exercises/{exercise-12/question.tex}}}
\mbox{}
  \begin{myenum}
  \item
    Solve
    \[
      5x^2 + y^2 = 3
    \]
    % mod 5, squares = 0^2=0, 1^2=1, 2^2=4, 3^2=4, 4^2=1
    (HINT: You don't really need number theory for this one. Why?
    But if you want to, imitate the solution for the previous
    problem.)
  \item
    Solve
    \[
      11y^2 - 5x^2 = 3
    \]
    % mod 5, squares = 0,1,4
    (HINT: This is just a slight change from the
    previous problem. \textit{But} now you need number theory. Try mod 4.
    If it does not work, try mod 5. Repeat.)
    % 3x^2 + y^2 = 3
    % {0,3} + {0,1} = 3
    % 0, 1, 3, 0 = 3
    % So x = 1, y = 0 (4)
    %
    % y^2=3 (5)
  \item
    Solve
    \[
      y^2 - 5x^2 = 2
    \]
    % mod 5, squares = 0^2=0, 1^2=1, 2^2=4, 3^2=4, 4^2=1    

  \item
    What about this one:
    \[
      x^2 - 5y^2 = 1
    \]    
  \end{myenum}
  {\scriptsize
[\textsc{Aside.}
Integer solutions to $x^2 - dy^2 = 1$ has been studied since at least 400BC.
This equation appear the Cattle Problem of Archimedes:

\begin{itemize}
  \item[]
If thou art diligent and wise, O stranger, compute the number of
cattle of the Sun, who once upon a time grazed on the fields of the
Thrinacian isle of Sicily, divided into four herds of different colours,
one milk white, another a glossy black, the third yellow and the last
dappled. In each herd were bulls, mighty in number according to these
proportions: Understand, stranger, that the white bulls were equal to
a half and a third of the black together with the whole of the yellow,
while the black were equal to the fourth part of the dappled and
a fifth, together with, once more, the whole of the yellow. Observe
further that the remaining bulls, the dappled, were equal to a sixth
part of the white and a seventh, together with all the yellow. These
were the proportions of the cows: The white were precisely equal to the
third part and a fourth of the whole herd of the black; while the black
were equal to the fourth part once more of the dappled and with it a
fifth part, when all, including the bulls went to pasture together. Now
the dappled in four parts8 were equal in number to a fifth part and a
sixth of the yellow herd. Finally the yellow were in number equal to
a sixth part and a seventh of the white herd. If thou canst accurately
tell, O stranger, the number of cattle of the Sun, giving separately the
number of well-fed bulls and again the number of females according
to each colour, thou wouldst not be called unskilled or ignorant of
numbers, but not yet shall thou be numbered among the wise. But
come, understand also all these conditions regarding the cows of the
Sun. When the white bulls mingled their number with the black, they
stood firm, equal in depth and breadth, and the plains of Thrinacia,
stretching far in all ways, were filled with their multitude. Again,
when the yellow and the dappled bulls were gathered into one herd
they stood in such a manner that their number, beginning from one,
grew slowly greater till it completed a triangular figure, there being
no bulls of other colours in their midst nor none of them lacking.
If thou art able, O stranger, to find out all these things and gather
them together in your mind, giving all the relations, thou shalt depart
crowned with glory and knowing that thou hast been adjudged perfect
in this species of wisdom.
\end{itemize}

If $W,X,Y,Z$ represents the number of white, black, yellow,
dappled bulls, you will get 
a systems of 7 linear equations, the first two being 
\begin{align*}
  W &= (1/2 + 1/3)X + Z \\
  X &= (1/4 + 1/5)Y + Z
\end{align*}
together with some constraints such as $W + X$ must be a square.
After some manipulations, it can be shown that the equation to solve looks like
\[
  x^2 - 410286423278424 y^2 = 1
\]
What was Archimedes thinking? You are find information on the Archimedes Cattle Problem on the web.]
}


  \solutionlink{sol:semigroup-associativity-0}
  \qed
\end{ex} 
\begin{python0}
from solutions import *
add(label="ex:semigroup-associativity-0",
    srcfilename='exercises/semigroup-associativity-0/answer.tex') 
\end{python0}


\begin{ex} 
  \label{ex:semigroup-associativity-0}
  \tinysidebar{\debug{exercises/{exercise-12/question.tex}}}
\mbox{}
  \begin{myenum}
  \item
    Solve
    \[
      5x^2 + y^2 = 3
    \]
    % mod 5, squares = 0^2=0, 1^2=1, 2^2=4, 3^2=4, 4^2=1
    (HINT: You don't really need number theory for this one. Why?
    But if you want to, imitate the solution for the previous
    problem.)
  \item
    Solve
    \[
      11y^2 - 5x^2 = 3
    \]
    % mod 5, squares = 0,1,4
    (HINT: This is just a slight change from the
    previous problem. \textit{But} now you need number theory. Try mod 4.
    If it does not work, try mod 5. Repeat.)
    % 3x^2 + y^2 = 3
    % {0,3} + {0,1} = 3
    % 0, 1, 3, 0 = 3
    % So x = 1, y = 0 (4)
    %
    % y^2=3 (5)
  \item
    Solve
    \[
      y^2 - 5x^2 = 2
    \]
    % mod 5, squares = 0^2=0, 1^2=1, 2^2=4, 3^2=4, 4^2=1    

  \item
    What about this one:
    \[
      x^2 - 5y^2 = 1
    \]    
  \end{myenum}
  {\scriptsize
[\textsc{Aside.}
Integer solutions to $x^2 - dy^2 = 1$ has been studied since at least 400BC.
This equation appear the Cattle Problem of Archimedes:

\begin{itemize}
  \item[]
If thou art diligent and wise, O stranger, compute the number of
cattle of the Sun, who once upon a time grazed on the fields of the
Thrinacian isle of Sicily, divided into four herds of different colours,
one milk white, another a glossy black, the third yellow and the last
dappled. In each herd were bulls, mighty in number according to these
proportions: Understand, stranger, that the white bulls were equal to
a half and a third of the black together with the whole of the yellow,
while the black were equal to the fourth part of the dappled and
a fifth, together with, once more, the whole of the yellow. Observe
further that the remaining bulls, the dappled, were equal to a sixth
part of the white and a seventh, together with all the yellow. These
were the proportions of the cows: The white were precisely equal to the
third part and a fourth of the whole herd of the black; while the black
were equal to the fourth part once more of the dappled and with it a
fifth part, when all, including the bulls went to pasture together. Now
the dappled in four parts8 were equal in number to a fifth part and a
sixth of the yellow herd. Finally the yellow were in number equal to
a sixth part and a seventh of the white herd. If thou canst accurately
tell, O stranger, the number of cattle of the Sun, giving separately the
number of well-fed bulls and again the number of females according
to each colour, thou wouldst not be called unskilled or ignorant of
numbers, but not yet shall thou be numbered among the wise. But
come, understand also all these conditions regarding the cows of the
Sun. When the white bulls mingled their number with the black, they
stood firm, equal in depth and breadth, and the plains of Thrinacia,
stretching far in all ways, were filled with their multitude. Again,
when the yellow and the dappled bulls were gathered into one herd
they stood in such a manner that their number, beginning from one,
grew slowly greater till it completed a triangular figure, there being
no bulls of other colours in their midst nor none of them lacking.
If thou art able, O stranger, to find out all these things and gather
them together in your mind, giving all the relations, thou shalt depart
crowned with glory and knowing that thou hast been adjudged perfect
in this species of wisdom.
\end{itemize}

If $W,X,Y,Z$ represents the number of white, black, yellow,
dappled bulls, you will get 
a systems of 7 linear equations, the first two being 
\begin{align*}
  W &= (1/2 + 1/3)X + Z \\
  X &= (1/4 + 1/5)Y + Z
\end{align*}
together with some constraints such as $W + X$ must be a square.
After some manipulations, it can be shown that the equation to solve looks like
\[
  x^2 - 410286423278424 y^2 = 1
\]
What was Archimedes thinking? You are find information on the Archimedes Cattle Problem on the web.]
}


  \solutionlink{sol:semigroup-associativity-0}
  \qed
\end{ex} 
\begin{python0}
from solutions import *
add(label="ex:semigroup-associativity-0",
    srcfilename='exercises/semigroup-associativity-0/answer.tex') 
\end{python0}


\begin{ex} 
  \label{ex:semigroup-associativity-0}
  \tinysidebar{\debug{exercises/{exercise-12/question.tex}}}
\mbox{}
  \begin{myenum}
  \item
    Solve
    \[
      5x^2 + y^2 = 3
    \]
    % mod 5, squares = 0^2=0, 1^2=1, 2^2=4, 3^2=4, 4^2=1
    (HINT: You don't really need number theory for this one. Why?
    But if you want to, imitate the solution for the previous
    problem.)
  \item
    Solve
    \[
      11y^2 - 5x^2 = 3
    \]
    % mod 5, squares = 0,1,4
    (HINT: This is just a slight change from the
    previous problem. \textit{But} now you need number theory. Try mod 4.
    If it does not work, try mod 5. Repeat.)
    % 3x^2 + y^2 = 3
    % {0,3} + {0,1} = 3
    % 0, 1, 3, 0 = 3
    % So x = 1, y = 0 (4)
    %
    % y^2=3 (5)
  \item
    Solve
    \[
      y^2 - 5x^2 = 2
    \]
    % mod 5, squares = 0^2=0, 1^2=1, 2^2=4, 3^2=4, 4^2=1    

  \item
    What about this one:
    \[
      x^2 - 5y^2 = 1
    \]    
  \end{myenum}
  {\scriptsize
[\textsc{Aside.}
Integer solutions to $x^2 - dy^2 = 1$ has been studied since at least 400BC.
This equation appear the Cattle Problem of Archimedes:

\begin{itemize}
  \item[]
If thou art diligent and wise, O stranger, compute the number of
cattle of the Sun, who once upon a time grazed on the fields of the
Thrinacian isle of Sicily, divided into four herds of different colours,
one milk white, another a glossy black, the third yellow and the last
dappled. In each herd were bulls, mighty in number according to these
proportions: Understand, stranger, that the white bulls were equal to
a half and a third of the black together with the whole of the yellow,
while the black were equal to the fourth part of the dappled and
a fifth, together with, once more, the whole of the yellow. Observe
further that the remaining bulls, the dappled, were equal to a sixth
part of the white and a seventh, together with all the yellow. These
were the proportions of the cows: The white were precisely equal to the
third part and a fourth of the whole herd of the black; while the black
were equal to the fourth part once more of the dappled and with it a
fifth part, when all, including the bulls went to pasture together. Now
the dappled in four parts8 were equal in number to a fifth part and a
sixth of the yellow herd. Finally the yellow were in number equal to
a sixth part and a seventh of the white herd. If thou canst accurately
tell, O stranger, the number of cattle of the Sun, giving separately the
number of well-fed bulls and again the number of females according
to each colour, thou wouldst not be called unskilled or ignorant of
numbers, but not yet shall thou be numbered among the wise. But
come, understand also all these conditions regarding the cows of the
Sun. When the white bulls mingled their number with the black, they
stood firm, equal in depth and breadth, and the plains of Thrinacia,
stretching far in all ways, were filled with their multitude. Again,
when the yellow and the dappled bulls were gathered into one herd
they stood in such a manner that their number, beginning from one,
grew slowly greater till it completed a triangular figure, there being
no bulls of other colours in their midst nor none of them lacking.
If thou art able, O stranger, to find out all these things and gather
them together in your mind, giving all the relations, thou shalt depart
crowned with glory and knowing that thou hast been adjudged perfect
in this species of wisdom.
\end{itemize}

If $W,X,Y,Z$ represents the number of white, black, yellow,
dappled bulls, you will get 
a systems of 7 linear equations, the first two being 
\begin{align*}
  W &= (1/2 + 1/3)X + Z \\
  X &= (1/4 + 1/5)Y + Z
\end{align*}
together with some constraints such as $W + X$ must be a square.
After some manipulations, it can be shown that the equation to solve looks like
\[
  x^2 - 410286423278424 y^2 = 1
\]
What was Archimedes thinking? You are find information on the Archimedes Cattle Problem on the web.]
}


  \solutionlink{sol:semigroup-associativity-0}
  \qed
\end{ex} 
\begin{python0}
from solutions import *
add(label="ex:semigroup-associativity-0",
    srcfilename='exercises/semigroup-associativity-0/answer.tex') 
\end{python0}


\begin{ex} 
  \label{ex:semigroup-associativity-0}
  \tinysidebar{\debug{exercises/{exercise-12/question.tex}}}
\mbox{}
  \begin{myenum}
  \item
    Solve
    \[
      5x^2 + y^2 = 3
    \]
    % mod 5, squares = 0^2=0, 1^2=1, 2^2=4, 3^2=4, 4^2=1
    (HINT: You don't really need number theory for this one. Why?
    But if you want to, imitate the solution for the previous
    problem.)
  \item
    Solve
    \[
      11y^2 - 5x^2 = 3
    \]
    % mod 5, squares = 0,1,4
    (HINT: This is just a slight change from the
    previous problem. \textit{But} now you need number theory. Try mod 4.
    If it does not work, try mod 5. Repeat.)
    % 3x^2 + y^2 = 3
    % {0,3} + {0,1} = 3
    % 0, 1, 3, 0 = 3
    % So x = 1, y = 0 (4)
    %
    % y^2=3 (5)
  \item
    Solve
    \[
      y^2 - 5x^2 = 2
    \]
    % mod 5, squares = 0^2=0, 1^2=1, 2^2=4, 3^2=4, 4^2=1    

  \item
    What about this one:
    \[
      x^2 - 5y^2 = 1
    \]    
  \end{myenum}
  {\scriptsize
[\textsc{Aside.}
Integer solutions to $x^2 - dy^2 = 1$ has been studied since at least 400BC.
This equation appear the Cattle Problem of Archimedes:

\begin{itemize}
  \item[]
If thou art diligent and wise, O stranger, compute the number of
cattle of the Sun, who once upon a time grazed on the fields of the
Thrinacian isle of Sicily, divided into four herds of different colours,
one milk white, another a glossy black, the third yellow and the last
dappled. In each herd were bulls, mighty in number according to these
proportions: Understand, stranger, that the white bulls were equal to
a half and a third of the black together with the whole of the yellow,
while the black were equal to the fourth part of the dappled and
a fifth, together with, once more, the whole of the yellow. Observe
further that the remaining bulls, the dappled, were equal to a sixth
part of the white and a seventh, together with all the yellow. These
were the proportions of the cows: The white were precisely equal to the
third part and a fourth of the whole herd of the black; while the black
were equal to the fourth part once more of the dappled and with it a
fifth part, when all, including the bulls went to pasture together. Now
the dappled in four parts8 were equal in number to a fifth part and a
sixth of the yellow herd. Finally the yellow were in number equal to
a sixth part and a seventh of the white herd. If thou canst accurately
tell, O stranger, the number of cattle of the Sun, giving separately the
number of well-fed bulls and again the number of females according
to each colour, thou wouldst not be called unskilled or ignorant of
numbers, but not yet shall thou be numbered among the wise. But
come, understand also all these conditions regarding the cows of the
Sun. When the white bulls mingled their number with the black, they
stood firm, equal in depth and breadth, and the plains of Thrinacia,
stretching far in all ways, were filled with their multitude. Again,
when the yellow and the dappled bulls were gathered into one herd
they stood in such a manner that their number, beginning from one,
grew slowly greater till it completed a triangular figure, there being
no bulls of other colours in their midst nor none of them lacking.
If thou art able, O stranger, to find out all these things and gather
them together in your mind, giving all the relations, thou shalt depart
crowned with glory and knowing that thou hast been adjudged perfect
in this species of wisdom.
\end{itemize}

If $W,X,Y,Z$ represents the number of white, black, yellow,
dappled bulls, you will get 
a systems of 7 linear equations, the first two being 
\begin{align*}
  W &= (1/2 + 1/3)X + Z \\
  X &= (1/4 + 1/5)Y + Z
\end{align*}
together with some constraints such as $W + X$ must be a square.
After some manipulations, it can be shown that the equation to solve looks like
\[
  x^2 - 410286423278424 y^2 = 1
\]
What was Archimedes thinking? You are find information on the Archimedes Cattle Problem on the web.]
}


  \solutionlink{sol:semigroup-associativity-0}
  \qed
\end{ex} 
\begin{python0}
from solutions import *
add(label="ex:semigroup-associativity-0",
    srcfilename='exercises/semigroup-associativity-0/answer.tex') 
\end{python0}


\begin{ex} 
  \label{ex:semigroup-associativity-0}
  \tinysidebar{\debug{exercises/{exercise-12/question.tex}}}
\mbox{}
  \begin{myenum}
  \item
    Solve
    \[
      5x^2 + y^2 = 3
    \]
    % mod 5, squares = 0^2=0, 1^2=1, 2^2=4, 3^2=4, 4^2=1
    (HINT: You don't really need number theory for this one. Why?
    But if you want to, imitate the solution for the previous
    problem.)
  \item
    Solve
    \[
      11y^2 - 5x^2 = 3
    \]
    % mod 5, squares = 0,1,4
    (HINT: This is just a slight change from the
    previous problem. \textit{But} now you need number theory. Try mod 4.
    If it does not work, try mod 5. Repeat.)
    % 3x^2 + y^2 = 3
    % {0,3} + {0,1} = 3
    % 0, 1, 3, 0 = 3
    % So x = 1, y = 0 (4)
    %
    % y^2=3 (5)
  \item
    Solve
    \[
      y^2 - 5x^2 = 2
    \]
    % mod 5, squares = 0^2=0, 1^2=1, 2^2=4, 3^2=4, 4^2=1    

  \item
    What about this one:
    \[
      x^2 - 5y^2 = 1
    \]    
  \end{myenum}
  {\scriptsize
[\textsc{Aside.}
Integer solutions to $x^2 - dy^2 = 1$ has been studied since at least 400BC.
This equation appear the Cattle Problem of Archimedes:

\begin{itemize}
  \item[]
If thou art diligent and wise, O stranger, compute the number of
cattle of the Sun, who once upon a time grazed on the fields of the
Thrinacian isle of Sicily, divided into four herds of different colours,
one milk white, another a glossy black, the third yellow and the last
dappled. In each herd were bulls, mighty in number according to these
proportions: Understand, stranger, that the white bulls were equal to
a half and a third of the black together with the whole of the yellow,
while the black were equal to the fourth part of the dappled and
a fifth, together with, once more, the whole of the yellow. Observe
further that the remaining bulls, the dappled, were equal to a sixth
part of the white and a seventh, together with all the yellow. These
were the proportions of the cows: The white were precisely equal to the
third part and a fourth of the whole herd of the black; while the black
were equal to the fourth part once more of the dappled and with it a
fifth part, when all, including the bulls went to pasture together. Now
the dappled in four parts8 were equal in number to a fifth part and a
sixth of the yellow herd. Finally the yellow were in number equal to
a sixth part and a seventh of the white herd. If thou canst accurately
tell, O stranger, the number of cattle of the Sun, giving separately the
number of well-fed bulls and again the number of females according
to each colour, thou wouldst not be called unskilled or ignorant of
numbers, but not yet shall thou be numbered among the wise. But
come, understand also all these conditions regarding the cows of the
Sun. When the white bulls mingled their number with the black, they
stood firm, equal in depth and breadth, and the plains of Thrinacia,
stretching far in all ways, were filled with their multitude. Again,
when the yellow and the dappled bulls were gathered into one herd
they stood in such a manner that their number, beginning from one,
grew slowly greater till it completed a triangular figure, there being
no bulls of other colours in their midst nor none of them lacking.
If thou art able, O stranger, to find out all these things and gather
them together in your mind, giving all the relations, thou shalt depart
crowned with glory and knowing that thou hast been adjudged perfect
in this species of wisdom.
\end{itemize}

If $W,X,Y,Z$ represents the number of white, black, yellow,
dappled bulls, you will get 
a systems of 7 linear equations, the first two being 
\begin{align*}
  W &= (1/2 + 1/3)X + Z \\
  X &= (1/4 + 1/5)Y + Z
\end{align*}
together with some constraints such as $W + X$ must be a square.
After some manipulations, it can be shown that the equation to solve looks like
\[
  x^2 - 410286423278424 y^2 = 1
\]
What was Archimedes thinking? You are find information on the Archimedes Cattle Problem on the web.]
}


  \solutionlink{sol:semigroup-associativity-0}
  \qed
\end{ex} 
\begin{python0}
from solutions import *
add(label="ex:semigroup-associativity-0",
    srcfilename='exercises/semigroup-associativity-0/answer.tex') 
\end{python0}


\begin{ex} 
  \label{ex:semigroup-associativity-0}
  \tinysidebar{\debug{exercises/{exercise-12/question.tex}}}
\mbox{}
  \begin{myenum}
  \item
    Solve
    \[
      5x^2 + y^2 = 3
    \]
    % mod 5, squares = 0^2=0, 1^2=1, 2^2=4, 3^2=4, 4^2=1
    (HINT: You don't really need number theory for this one. Why?
    But if you want to, imitate the solution for the previous
    problem.)
  \item
    Solve
    \[
      11y^2 - 5x^2 = 3
    \]
    % mod 5, squares = 0,1,4
    (HINT: This is just a slight change from the
    previous problem. \textit{But} now you need number theory. Try mod 4.
    If it does not work, try mod 5. Repeat.)
    % 3x^2 + y^2 = 3
    % {0,3} + {0,1} = 3
    % 0, 1, 3, 0 = 3
    % So x = 1, y = 0 (4)
    %
    % y^2=3 (5)
  \item
    Solve
    \[
      y^2 - 5x^2 = 2
    \]
    % mod 5, squares = 0^2=0, 1^2=1, 2^2=4, 3^2=4, 4^2=1    

  \item
    What about this one:
    \[
      x^2 - 5y^2 = 1
    \]    
  \end{myenum}
  {\scriptsize
[\textsc{Aside.}
Integer solutions to $x^2 - dy^2 = 1$ has been studied since at least 400BC.
This equation appear the Cattle Problem of Archimedes:

\begin{itemize}
  \item[]
If thou art diligent and wise, O stranger, compute the number of
cattle of the Sun, who once upon a time grazed on the fields of the
Thrinacian isle of Sicily, divided into four herds of different colours,
one milk white, another a glossy black, the third yellow and the last
dappled. In each herd were bulls, mighty in number according to these
proportions: Understand, stranger, that the white bulls were equal to
a half and a third of the black together with the whole of the yellow,
while the black were equal to the fourth part of the dappled and
a fifth, together with, once more, the whole of the yellow. Observe
further that the remaining bulls, the dappled, were equal to a sixth
part of the white and a seventh, together with all the yellow. These
were the proportions of the cows: The white were precisely equal to the
third part and a fourth of the whole herd of the black; while the black
were equal to the fourth part once more of the dappled and with it a
fifth part, when all, including the bulls went to pasture together. Now
the dappled in four parts8 were equal in number to a fifth part and a
sixth of the yellow herd. Finally the yellow were in number equal to
a sixth part and a seventh of the white herd. If thou canst accurately
tell, O stranger, the number of cattle of the Sun, giving separately the
number of well-fed bulls and again the number of females according
to each colour, thou wouldst not be called unskilled or ignorant of
numbers, but not yet shall thou be numbered among the wise. But
come, understand also all these conditions regarding the cows of the
Sun. When the white bulls mingled their number with the black, they
stood firm, equal in depth and breadth, and the plains of Thrinacia,
stretching far in all ways, were filled with their multitude. Again,
when the yellow and the dappled bulls were gathered into one herd
they stood in such a manner that their number, beginning from one,
grew slowly greater till it completed a triangular figure, there being
no bulls of other colours in their midst nor none of them lacking.
If thou art able, O stranger, to find out all these things and gather
them together in your mind, giving all the relations, thou shalt depart
crowned with glory and knowing that thou hast been adjudged perfect
in this species of wisdom.
\end{itemize}

If $W,X,Y,Z$ represents the number of white, black, yellow,
dappled bulls, you will get 
a systems of 7 linear equations, the first two being 
\begin{align*}
  W &= (1/2 + 1/3)X + Z \\
  X &= (1/4 + 1/5)Y + Z
\end{align*}
together with some constraints such as $W + X$ must be a square.
After some manipulations, it can be shown that the equation to solve looks like
\[
  x^2 - 410286423278424 y^2 = 1
\]
What was Archimedes thinking? You are find information on the Archimedes Cattle Problem on the web.]
}


  \solutionlink{sol:semigroup-associativity-0}
  \qed
\end{ex} 
\begin{python0}
from solutions import *
add(label="ex:semigroup-associativity-0",
    srcfilename='exercises/semigroup-associativity-0/answer.tex') 
\end{python0}


\begin{ex} 
  \label{ex:semigroup-associativity-0}
  \tinysidebar{\debug{exercises/{exercise-12/question.tex}}}
\mbox{}
  \begin{myenum}
  \item
    Solve
    \[
      5x^2 + y^2 = 3
    \]
    % mod 5, squares = 0^2=0, 1^2=1, 2^2=4, 3^2=4, 4^2=1
    (HINT: You don't really need number theory for this one. Why?
    But if you want to, imitate the solution for the previous
    problem.)
  \item
    Solve
    \[
      11y^2 - 5x^2 = 3
    \]
    % mod 5, squares = 0,1,4
    (HINT: This is just a slight change from the
    previous problem. \textit{But} now you need number theory. Try mod 4.
    If it does not work, try mod 5. Repeat.)
    % 3x^2 + y^2 = 3
    % {0,3} + {0,1} = 3
    % 0, 1, 3, 0 = 3
    % So x = 1, y = 0 (4)
    %
    % y^2=3 (5)
  \item
    Solve
    \[
      y^2 - 5x^2 = 2
    \]
    % mod 5, squares = 0^2=0, 1^2=1, 2^2=4, 3^2=4, 4^2=1    

  \item
    What about this one:
    \[
      x^2 - 5y^2 = 1
    \]    
  \end{myenum}
  {\scriptsize
[\textsc{Aside.}
Integer solutions to $x^2 - dy^2 = 1$ has been studied since at least 400BC.
This equation appear the Cattle Problem of Archimedes:

\begin{itemize}
  \item[]
If thou art diligent and wise, O stranger, compute the number of
cattle of the Sun, who once upon a time grazed on the fields of the
Thrinacian isle of Sicily, divided into four herds of different colours,
one milk white, another a glossy black, the third yellow and the last
dappled. In each herd were bulls, mighty in number according to these
proportions: Understand, stranger, that the white bulls were equal to
a half and a third of the black together with the whole of the yellow,
while the black were equal to the fourth part of the dappled and
a fifth, together with, once more, the whole of the yellow. Observe
further that the remaining bulls, the dappled, were equal to a sixth
part of the white and a seventh, together with all the yellow. These
were the proportions of the cows: The white were precisely equal to the
third part and a fourth of the whole herd of the black; while the black
were equal to the fourth part once more of the dappled and with it a
fifth part, when all, including the bulls went to pasture together. Now
the dappled in four parts8 were equal in number to a fifth part and a
sixth of the yellow herd. Finally the yellow were in number equal to
a sixth part and a seventh of the white herd. If thou canst accurately
tell, O stranger, the number of cattle of the Sun, giving separately the
number of well-fed bulls and again the number of females according
to each colour, thou wouldst not be called unskilled or ignorant of
numbers, but not yet shall thou be numbered among the wise. But
come, understand also all these conditions regarding the cows of the
Sun. When the white bulls mingled their number with the black, they
stood firm, equal in depth and breadth, and the plains of Thrinacia,
stretching far in all ways, were filled with their multitude. Again,
when the yellow and the dappled bulls were gathered into one herd
they stood in such a manner that their number, beginning from one,
grew slowly greater till it completed a triangular figure, there being
no bulls of other colours in their midst nor none of them lacking.
If thou art able, O stranger, to find out all these things and gather
them together in your mind, giving all the relations, thou shalt depart
crowned with glory and knowing that thou hast been adjudged perfect
in this species of wisdom.
\end{itemize}

If $W,X,Y,Z$ represents the number of white, black, yellow,
dappled bulls, you will get 
a systems of 7 linear equations, the first two being 
\begin{align*}
  W &= (1/2 + 1/3)X + Z \\
  X &= (1/4 + 1/5)Y + Z
\end{align*}
together with some constraints such as $W + X$ must be a square.
After some manipulations, it can be shown that the equation to solve looks like
\[
  x^2 - 410286423278424 y^2 = 1
\]
What was Archimedes thinking? You are find information on the Archimedes Cattle Problem on the web.]
}


  \solutionlink{sol:semigroup-associativity-0}
  \qed
\end{ex} 
\begin{python0}
from solutions import *
add(label="ex:semigroup-associativity-0",
    srcfilename='exercises/semigroup-associativity-0/answer.tex') 
\end{python0}


\begin{ex} 
  \label{ex:semigroup-associativity-0}
  \tinysidebar{\debug{exercises/{exercise-12/question.tex}}}
\mbox{}
  \begin{myenum}
  \item
    Solve
    \[
      5x^2 + y^2 = 3
    \]
    % mod 5, squares = 0^2=0, 1^2=1, 2^2=4, 3^2=4, 4^2=1
    (HINT: You don't really need number theory for this one. Why?
    But if you want to, imitate the solution for the previous
    problem.)
  \item
    Solve
    \[
      11y^2 - 5x^2 = 3
    \]
    % mod 5, squares = 0,1,4
    (HINT: This is just a slight change from the
    previous problem. \textit{But} now you need number theory. Try mod 4.
    If it does not work, try mod 5. Repeat.)
    % 3x^2 + y^2 = 3
    % {0,3} + {0,1} = 3
    % 0, 1, 3, 0 = 3
    % So x = 1, y = 0 (4)
    %
    % y^2=3 (5)
  \item
    Solve
    \[
      y^2 - 5x^2 = 2
    \]
    % mod 5, squares = 0^2=0, 1^2=1, 2^2=4, 3^2=4, 4^2=1    

  \item
    What about this one:
    \[
      x^2 - 5y^2 = 1
    \]    
  \end{myenum}
  {\scriptsize
[\textsc{Aside.}
Integer solutions to $x^2 - dy^2 = 1$ has been studied since at least 400BC.
This equation appear the Cattle Problem of Archimedes:

\begin{itemize}
  \item[]
If thou art diligent and wise, O stranger, compute the number of
cattle of the Sun, who once upon a time grazed on the fields of the
Thrinacian isle of Sicily, divided into four herds of different colours,
one milk white, another a glossy black, the third yellow and the last
dappled. In each herd were bulls, mighty in number according to these
proportions: Understand, stranger, that the white bulls were equal to
a half and a third of the black together with the whole of the yellow,
while the black were equal to the fourth part of the dappled and
a fifth, together with, once more, the whole of the yellow. Observe
further that the remaining bulls, the dappled, were equal to a sixth
part of the white and a seventh, together with all the yellow. These
were the proportions of the cows: The white were precisely equal to the
third part and a fourth of the whole herd of the black; while the black
were equal to the fourth part once more of the dappled and with it a
fifth part, when all, including the bulls went to pasture together. Now
the dappled in four parts8 were equal in number to a fifth part and a
sixth of the yellow herd. Finally the yellow were in number equal to
a sixth part and a seventh of the white herd. If thou canst accurately
tell, O stranger, the number of cattle of the Sun, giving separately the
number of well-fed bulls and again the number of females according
to each colour, thou wouldst not be called unskilled or ignorant of
numbers, but not yet shall thou be numbered among the wise. But
come, understand also all these conditions regarding the cows of the
Sun. When the white bulls mingled their number with the black, they
stood firm, equal in depth and breadth, and the plains of Thrinacia,
stretching far in all ways, were filled with their multitude. Again,
when the yellow and the dappled bulls were gathered into one herd
they stood in such a manner that their number, beginning from one,
grew slowly greater till it completed a triangular figure, there being
no bulls of other colours in their midst nor none of them lacking.
If thou art able, O stranger, to find out all these things and gather
them together in your mind, giving all the relations, thou shalt depart
crowned with glory and knowing that thou hast been adjudged perfect
in this species of wisdom.
\end{itemize}

If $W,X,Y,Z$ represents the number of white, black, yellow,
dappled bulls, you will get 
a systems of 7 linear equations, the first two being 
\begin{align*}
  W &= (1/2 + 1/3)X + Z \\
  X &= (1/4 + 1/5)Y + Z
\end{align*}
together with some constraints such as $W + X$ must be a square.
After some manipulations, it can be shown that the equation to solve looks like
\[
  x^2 - 410286423278424 y^2 = 1
\]
What was Archimedes thinking? You are find information on the Archimedes Cattle Problem on the web.]
}


  \solutionlink{sol:semigroup-associativity-0}
  \qed
\end{ex} 
\begin{python0}
from solutions import *
add(label="ex:semigroup-associativity-0",
    srcfilename='exercises/semigroup-associativity-0/answer.tex') 
\end{python0}


\begin{ex} 
  \label{ex:semigroup-associativity-0}
  \tinysidebar{\debug{exercises/{exercise-12/question.tex}}}
\mbox{}
  \begin{myenum}
  \item
    Solve
    \[
      5x^2 + y^2 = 3
    \]
    % mod 5, squares = 0^2=0, 1^2=1, 2^2=4, 3^2=4, 4^2=1
    (HINT: You don't really need number theory for this one. Why?
    But if you want to, imitate the solution for the previous
    problem.)
  \item
    Solve
    \[
      11y^2 - 5x^2 = 3
    \]
    % mod 5, squares = 0,1,4
    (HINT: This is just a slight change from the
    previous problem. \textit{But} now you need number theory. Try mod 4.
    If it does not work, try mod 5. Repeat.)
    % 3x^2 + y^2 = 3
    % {0,3} + {0,1} = 3
    % 0, 1, 3, 0 = 3
    % So x = 1, y = 0 (4)
    %
    % y^2=3 (5)
  \item
    Solve
    \[
      y^2 - 5x^2 = 2
    \]
    % mod 5, squares = 0^2=0, 1^2=1, 2^2=4, 3^2=4, 4^2=1    

  \item
    What about this one:
    \[
      x^2 - 5y^2 = 1
    \]    
  \end{myenum}
  {\scriptsize
[\textsc{Aside.}
Integer solutions to $x^2 - dy^2 = 1$ has been studied since at least 400BC.
This equation appear the Cattle Problem of Archimedes:

\begin{itemize}
  \item[]
If thou art diligent and wise, O stranger, compute the number of
cattle of the Sun, who once upon a time grazed on the fields of the
Thrinacian isle of Sicily, divided into four herds of different colours,
one milk white, another a glossy black, the third yellow and the last
dappled. In each herd were bulls, mighty in number according to these
proportions: Understand, stranger, that the white bulls were equal to
a half and a third of the black together with the whole of the yellow,
while the black were equal to the fourth part of the dappled and
a fifth, together with, once more, the whole of the yellow. Observe
further that the remaining bulls, the dappled, were equal to a sixth
part of the white and a seventh, together with all the yellow. These
were the proportions of the cows: The white were precisely equal to the
third part and a fourth of the whole herd of the black; while the black
were equal to the fourth part once more of the dappled and with it a
fifth part, when all, including the bulls went to pasture together. Now
the dappled in four parts8 were equal in number to a fifth part and a
sixth of the yellow herd. Finally the yellow were in number equal to
a sixth part and a seventh of the white herd. If thou canst accurately
tell, O stranger, the number of cattle of the Sun, giving separately the
number of well-fed bulls and again the number of females according
to each colour, thou wouldst not be called unskilled or ignorant of
numbers, but not yet shall thou be numbered among the wise. But
come, understand also all these conditions regarding the cows of the
Sun. When the white bulls mingled their number with the black, they
stood firm, equal in depth and breadth, and the plains of Thrinacia,
stretching far in all ways, were filled with their multitude. Again,
when the yellow and the dappled bulls were gathered into one herd
they stood in such a manner that their number, beginning from one,
grew slowly greater till it completed a triangular figure, there being
no bulls of other colours in their midst nor none of them lacking.
If thou art able, O stranger, to find out all these things and gather
them together in your mind, giving all the relations, thou shalt depart
crowned with glory and knowing that thou hast been adjudged perfect
in this species of wisdom.
\end{itemize}

If $W,X,Y,Z$ represents the number of white, black, yellow,
dappled bulls, you will get 
a systems of 7 linear equations, the first two being 
\begin{align*}
  W &= (1/2 + 1/3)X + Z \\
  X &= (1/4 + 1/5)Y + Z
\end{align*}
together with some constraints such as $W + X$ must be a square.
After some manipulations, it can be shown that the equation to solve looks like
\[
  x^2 - 410286423278424 y^2 = 1
\]
What was Archimedes thinking? You are find information on the Archimedes Cattle Problem on the web.]
}


  \solutionlink{sol:semigroup-associativity-0}
  \qed
\end{ex} 
\begin{python0}
from solutions import *
add(label="ex:semigroup-associativity-0",
    srcfilename='exercises/semigroup-associativity-0/answer.tex') 
\end{python0}


\begin{ex} 
  \label{ex:semigroup-associativity-0}
  \tinysidebar{\debug{exercises/{exercise-12/question.tex}}}
\mbox{}
  \begin{myenum}
  \item
    Solve
    \[
      5x^2 + y^2 = 3
    \]
    % mod 5, squares = 0^2=0, 1^2=1, 2^2=4, 3^2=4, 4^2=1
    (HINT: You don't really need number theory for this one. Why?
    But if you want to, imitate the solution for the previous
    problem.)
  \item
    Solve
    \[
      11y^2 - 5x^2 = 3
    \]
    % mod 5, squares = 0,1,4
    (HINT: This is just a slight change from the
    previous problem. \textit{But} now you need number theory. Try mod 4.
    If it does not work, try mod 5. Repeat.)
    % 3x^2 + y^2 = 3
    % {0,3} + {0,1} = 3
    % 0, 1, 3, 0 = 3
    % So x = 1, y = 0 (4)
    %
    % y^2=3 (5)
  \item
    Solve
    \[
      y^2 - 5x^2 = 2
    \]
    % mod 5, squares = 0^2=0, 1^2=1, 2^2=4, 3^2=4, 4^2=1    

  \item
    What about this one:
    \[
      x^2 - 5y^2 = 1
    \]    
  \end{myenum}
  {\scriptsize
[\textsc{Aside.}
Integer solutions to $x^2 - dy^2 = 1$ has been studied since at least 400BC.
This equation appear the Cattle Problem of Archimedes:

\begin{itemize}
  \item[]
If thou art diligent and wise, O stranger, compute the number of
cattle of the Sun, who once upon a time grazed on the fields of the
Thrinacian isle of Sicily, divided into four herds of different colours,
one milk white, another a glossy black, the third yellow and the last
dappled. In each herd were bulls, mighty in number according to these
proportions: Understand, stranger, that the white bulls were equal to
a half and a third of the black together with the whole of the yellow,
while the black were equal to the fourth part of the dappled and
a fifth, together with, once more, the whole of the yellow. Observe
further that the remaining bulls, the dappled, were equal to a sixth
part of the white and a seventh, together with all the yellow. These
were the proportions of the cows: The white were precisely equal to the
third part and a fourth of the whole herd of the black; while the black
were equal to the fourth part once more of the dappled and with it a
fifth part, when all, including the bulls went to pasture together. Now
the dappled in four parts8 were equal in number to a fifth part and a
sixth of the yellow herd. Finally the yellow were in number equal to
a sixth part and a seventh of the white herd. If thou canst accurately
tell, O stranger, the number of cattle of the Sun, giving separately the
number of well-fed bulls and again the number of females according
to each colour, thou wouldst not be called unskilled or ignorant of
numbers, but not yet shall thou be numbered among the wise. But
come, understand also all these conditions regarding the cows of the
Sun. When the white bulls mingled their number with the black, they
stood firm, equal in depth and breadth, and the plains of Thrinacia,
stretching far in all ways, were filled with their multitude. Again,
when the yellow and the dappled bulls were gathered into one herd
they stood in such a manner that their number, beginning from one,
grew slowly greater till it completed a triangular figure, there being
no bulls of other colours in their midst nor none of them lacking.
If thou art able, O stranger, to find out all these things and gather
them together in your mind, giving all the relations, thou shalt depart
crowned with glory and knowing that thou hast been adjudged perfect
in this species of wisdom.
\end{itemize}

If $W,X,Y,Z$ represents the number of white, black, yellow,
dappled bulls, you will get 
a systems of 7 linear equations, the first two being 
\begin{align*}
  W &= (1/2 + 1/3)X + Z \\
  X &= (1/4 + 1/5)Y + Z
\end{align*}
together with some constraints such as $W + X$ must be a square.
After some manipulations, it can be shown that the equation to solve looks like
\[
  x^2 - 410286423278424 y^2 = 1
\]
What was Archimedes thinking? You are find information on the Archimedes Cattle Problem on the web.]
}


  \solutionlink{sol:semigroup-associativity-0}
  \qed
\end{ex} 
\begin{python0}
from solutions import *
add(label="ex:semigroup-associativity-0",
    srcfilename='exercises/semigroup-associativity-0/answer.tex') 
\end{python0}


\begin{ex} 
  \label{ex:semigroup-associativity-0}
  \tinysidebar{\debug{exercises/{exercise-12/question.tex}}}
\mbox{}
  \begin{myenum}
  \item
    Solve
    \[
      5x^2 + y^2 = 3
    \]
    % mod 5, squares = 0^2=0, 1^2=1, 2^2=4, 3^2=4, 4^2=1
    (HINT: You don't really need number theory for this one. Why?
    But if you want to, imitate the solution for the previous
    problem.)
  \item
    Solve
    \[
      11y^2 - 5x^2 = 3
    \]
    % mod 5, squares = 0,1,4
    (HINT: This is just a slight change from the
    previous problem. \textit{But} now you need number theory. Try mod 4.
    If it does not work, try mod 5. Repeat.)
    % 3x^2 + y^2 = 3
    % {0,3} + {0,1} = 3
    % 0, 1, 3, 0 = 3
    % So x = 1, y = 0 (4)
    %
    % y^2=3 (5)
  \item
    Solve
    \[
      y^2 - 5x^2 = 2
    \]
    % mod 5, squares = 0^2=0, 1^2=1, 2^2=4, 3^2=4, 4^2=1    

  \item
    What about this one:
    \[
      x^2 - 5y^2 = 1
    \]    
  \end{myenum}
  {\scriptsize
[\textsc{Aside.}
Integer solutions to $x^2 - dy^2 = 1$ has been studied since at least 400BC.
This equation appear the Cattle Problem of Archimedes:

\begin{itemize}
  \item[]
If thou art diligent and wise, O stranger, compute the number of
cattle of the Sun, who once upon a time grazed on the fields of the
Thrinacian isle of Sicily, divided into four herds of different colours,
one milk white, another a glossy black, the third yellow and the last
dappled. In each herd were bulls, mighty in number according to these
proportions: Understand, stranger, that the white bulls were equal to
a half and a third of the black together with the whole of the yellow,
while the black were equal to the fourth part of the dappled and
a fifth, together with, once more, the whole of the yellow. Observe
further that the remaining bulls, the dappled, were equal to a sixth
part of the white and a seventh, together with all the yellow. These
were the proportions of the cows: The white were precisely equal to the
third part and a fourth of the whole herd of the black; while the black
were equal to the fourth part once more of the dappled and with it a
fifth part, when all, including the bulls went to pasture together. Now
the dappled in four parts8 were equal in number to a fifth part and a
sixth of the yellow herd. Finally the yellow were in number equal to
a sixth part and a seventh of the white herd. If thou canst accurately
tell, O stranger, the number of cattle of the Sun, giving separately the
number of well-fed bulls and again the number of females according
to each colour, thou wouldst not be called unskilled or ignorant of
numbers, but not yet shall thou be numbered among the wise. But
come, understand also all these conditions regarding the cows of the
Sun. When the white bulls mingled their number with the black, they
stood firm, equal in depth and breadth, and the plains of Thrinacia,
stretching far in all ways, were filled with their multitude. Again,
when the yellow and the dappled bulls were gathered into one herd
they stood in such a manner that their number, beginning from one,
grew slowly greater till it completed a triangular figure, there being
no bulls of other colours in their midst nor none of them lacking.
If thou art able, O stranger, to find out all these things and gather
them together in your mind, giving all the relations, thou shalt depart
crowned with glory and knowing that thou hast been adjudged perfect
in this species of wisdom.
\end{itemize}

If $W,X,Y,Z$ represents the number of white, black, yellow,
dappled bulls, you will get 
a systems of 7 linear equations, the first two being 
\begin{align*}
  W &= (1/2 + 1/3)X + Z \\
  X &= (1/4 + 1/5)Y + Z
\end{align*}
together with some constraints such as $W + X$ must be a square.
After some manipulations, it can be shown that the equation to solve looks like
\[
  x^2 - 410286423278424 y^2 = 1
\]
What was Archimedes thinking? You are find information on the Archimedes Cattle Problem on the web.]
}


  \solutionlink{sol:semigroup-associativity-0}
  \qed
\end{ex} 
\begin{python0}
from solutions import *
add(label="ex:semigroup-associativity-0",
    srcfilename='exercises/semigroup-associativity-0/answer.tex') 
\end{python0}


\begin{ex} 
  \label{ex:semigroup-associativity-0}
  \tinysidebar{\debug{exercises/{exercise-12/question.tex}}}
\mbox{}
  \begin{myenum}
  \item
    Solve
    \[
      5x^2 + y^2 = 3
    \]
    % mod 5, squares = 0^2=0, 1^2=1, 2^2=4, 3^2=4, 4^2=1
    (HINT: You don't really need number theory for this one. Why?
    But if you want to, imitate the solution for the previous
    problem.)
  \item
    Solve
    \[
      11y^2 - 5x^2 = 3
    \]
    % mod 5, squares = 0,1,4
    (HINT: This is just a slight change from the
    previous problem. \textit{But} now you need number theory. Try mod 4.
    If it does not work, try mod 5. Repeat.)
    % 3x^2 + y^2 = 3
    % {0,3} + {0,1} = 3
    % 0, 1, 3, 0 = 3
    % So x = 1, y = 0 (4)
    %
    % y^2=3 (5)
  \item
    Solve
    \[
      y^2 - 5x^2 = 2
    \]
    % mod 5, squares = 0^2=0, 1^2=1, 2^2=4, 3^2=4, 4^2=1    

  \item
    What about this one:
    \[
      x^2 - 5y^2 = 1
    \]    
  \end{myenum}
  {\scriptsize
[\textsc{Aside.}
Integer solutions to $x^2 - dy^2 = 1$ has been studied since at least 400BC.
This equation appear the Cattle Problem of Archimedes:

\begin{itemize}
  \item[]
If thou art diligent and wise, O stranger, compute the number of
cattle of the Sun, who once upon a time grazed on the fields of the
Thrinacian isle of Sicily, divided into four herds of different colours,
one milk white, another a glossy black, the third yellow and the last
dappled. In each herd were bulls, mighty in number according to these
proportions: Understand, stranger, that the white bulls were equal to
a half and a third of the black together with the whole of the yellow,
while the black were equal to the fourth part of the dappled and
a fifth, together with, once more, the whole of the yellow. Observe
further that the remaining bulls, the dappled, were equal to a sixth
part of the white and a seventh, together with all the yellow. These
were the proportions of the cows: The white were precisely equal to the
third part and a fourth of the whole herd of the black; while the black
were equal to the fourth part once more of the dappled and with it a
fifth part, when all, including the bulls went to pasture together. Now
the dappled in four parts8 were equal in number to a fifth part and a
sixth of the yellow herd. Finally the yellow were in number equal to
a sixth part and a seventh of the white herd. If thou canst accurately
tell, O stranger, the number of cattle of the Sun, giving separately the
number of well-fed bulls and again the number of females according
to each colour, thou wouldst not be called unskilled or ignorant of
numbers, but not yet shall thou be numbered among the wise. But
come, understand also all these conditions regarding the cows of the
Sun. When the white bulls mingled their number with the black, they
stood firm, equal in depth and breadth, and the plains of Thrinacia,
stretching far in all ways, were filled with their multitude. Again,
when the yellow and the dappled bulls were gathered into one herd
they stood in such a manner that their number, beginning from one,
grew slowly greater till it completed a triangular figure, there being
no bulls of other colours in their midst nor none of them lacking.
If thou art able, O stranger, to find out all these things and gather
them together in your mind, giving all the relations, thou shalt depart
crowned with glory and knowing that thou hast been adjudged perfect
in this species of wisdom.
\end{itemize}

If $W,X,Y,Z$ represents the number of white, black, yellow,
dappled bulls, you will get 
a systems of 7 linear equations, the first two being 
\begin{align*}
  W &= (1/2 + 1/3)X + Z \\
  X &= (1/4 + 1/5)Y + Z
\end{align*}
together with some constraints such as $W + X$ must be a square.
After some manipulations, it can be shown that the equation to solve looks like
\[
  x^2 - 410286423278424 y^2 = 1
\]
What was Archimedes thinking? You are find information on the Archimedes Cattle Problem on the web.]
}


  \solutionlink{sol:semigroup-associativity-0}
  \qed
\end{ex} 
\begin{python0}
from solutions import *
add(label="ex:semigroup-associativity-0",
    srcfilename='exercises/semigroup-associativity-0/answer.tex') 
\end{python0}


\begin{ex} 
  \label{ex:semigroup-associativity-0}
  \tinysidebar{\debug{exercises/{exercise-12/question.tex}}}
\mbox{}
  \begin{myenum}
  \item
    Solve
    \[
      5x^2 + y^2 = 3
    \]
    % mod 5, squares = 0^2=0, 1^2=1, 2^2=4, 3^2=4, 4^2=1
    (HINT: You don't really need number theory for this one. Why?
    But if you want to, imitate the solution for the previous
    problem.)
  \item
    Solve
    \[
      11y^2 - 5x^2 = 3
    \]
    % mod 5, squares = 0,1,4
    (HINT: This is just a slight change from the
    previous problem. \textit{But} now you need number theory. Try mod 4.
    If it does not work, try mod 5. Repeat.)
    % 3x^2 + y^2 = 3
    % {0,3} + {0,1} = 3
    % 0, 1, 3, 0 = 3
    % So x = 1, y = 0 (4)
    %
    % y^2=3 (5)
  \item
    Solve
    \[
      y^2 - 5x^2 = 2
    \]
    % mod 5, squares = 0^2=0, 1^2=1, 2^2=4, 3^2=4, 4^2=1    

  \item
    What about this one:
    \[
      x^2 - 5y^2 = 1
    \]    
  \end{myenum}
  {\scriptsize
[\textsc{Aside.}
Integer solutions to $x^2 - dy^2 = 1$ has been studied since at least 400BC.
This equation appear the Cattle Problem of Archimedes:

\begin{itemize}
  \item[]
If thou art diligent and wise, O stranger, compute the number of
cattle of the Sun, who once upon a time grazed on the fields of the
Thrinacian isle of Sicily, divided into four herds of different colours,
one milk white, another a glossy black, the third yellow and the last
dappled. In each herd were bulls, mighty in number according to these
proportions: Understand, stranger, that the white bulls were equal to
a half and a third of the black together with the whole of the yellow,
while the black were equal to the fourth part of the dappled and
a fifth, together with, once more, the whole of the yellow. Observe
further that the remaining bulls, the dappled, were equal to a sixth
part of the white and a seventh, together with all the yellow. These
were the proportions of the cows: The white were precisely equal to the
third part and a fourth of the whole herd of the black; while the black
were equal to the fourth part once more of the dappled and with it a
fifth part, when all, including the bulls went to pasture together. Now
the dappled in four parts8 were equal in number to a fifth part and a
sixth of the yellow herd. Finally the yellow were in number equal to
a sixth part and a seventh of the white herd. If thou canst accurately
tell, O stranger, the number of cattle of the Sun, giving separately the
number of well-fed bulls and again the number of females according
to each colour, thou wouldst not be called unskilled or ignorant of
numbers, but not yet shall thou be numbered among the wise. But
come, understand also all these conditions regarding the cows of the
Sun. When the white bulls mingled their number with the black, they
stood firm, equal in depth and breadth, and the plains of Thrinacia,
stretching far in all ways, were filled with their multitude. Again,
when the yellow and the dappled bulls were gathered into one herd
they stood in such a manner that their number, beginning from one,
grew slowly greater till it completed a triangular figure, there being
no bulls of other colours in their midst nor none of them lacking.
If thou art able, O stranger, to find out all these things and gather
them together in your mind, giving all the relations, thou shalt depart
crowned with glory and knowing that thou hast been adjudged perfect
in this species of wisdom.
\end{itemize}

If $W,X,Y,Z$ represents the number of white, black, yellow,
dappled bulls, you will get 
a systems of 7 linear equations, the first two being 
\begin{align*}
  W &= (1/2 + 1/3)X + Z \\
  X &= (1/4 + 1/5)Y + Z
\end{align*}
together with some constraints such as $W + X$ must be a square.
After some manipulations, it can be shown that the equation to solve looks like
\[
  x^2 - 410286423278424 y^2 = 1
\]
What was Archimedes thinking? You are find information on the Archimedes Cattle Problem on the web.]
}


  \solutionlink{sol:semigroup-associativity-0}
  \qed
\end{ex} 
\begin{python0}
from solutions import *
add(label="ex:semigroup-associativity-0",
    srcfilename='exercises/semigroup-associativity-0/answer.tex') 
\end{python0}


\begin{ex} 
  \label{ex:semigroup-associativity-0}
  \tinysidebar{\debug{exercises/{exercise-12/question.tex}}}
\mbox{}
  \begin{myenum}
  \item
    Solve
    \[
      5x^2 + y^2 = 3
    \]
    % mod 5, squares = 0^2=0, 1^2=1, 2^2=4, 3^2=4, 4^2=1
    (HINT: You don't really need number theory for this one. Why?
    But if you want to, imitate the solution for the previous
    problem.)
  \item
    Solve
    \[
      11y^2 - 5x^2 = 3
    \]
    % mod 5, squares = 0,1,4
    (HINT: This is just a slight change from the
    previous problem. \textit{But} now you need number theory. Try mod 4.
    If it does not work, try mod 5. Repeat.)
    % 3x^2 + y^2 = 3
    % {0,3} + {0,1} = 3
    % 0, 1, 3, 0 = 3
    % So x = 1, y = 0 (4)
    %
    % y^2=3 (5)
  \item
    Solve
    \[
      y^2 - 5x^2 = 2
    \]
    % mod 5, squares = 0^2=0, 1^2=1, 2^2=4, 3^2=4, 4^2=1    

  \item
    What about this one:
    \[
      x^2 - 5y^2 = 1
    \]    
  \end{myenum}
  {\scriptsize
[\textsc{Aside.}
Integer solutions to $x^2 - dy^2 = 1$ has been studied since at least 400BC.
This equation appear the Cattle Problem of Archimedes:

\begin{itemize}
  \item[]
If thou art diligent and wise, O stranger, compute the number of
cattle of the Sun, who once upon a time grazed on the fields of the
Thrinacian isle of Sicily, divided into four herds of different colours,
one milk white, another a glossy black, the third yellow and the last
dappled. In each herd were bulls, mighty in number according to these
proportions: Understand, stranger, that the white bulls were equal to
a half and a third of the black together with the whole of the yellow,
while the black were equal to the fourth part of the dappled and
a fifth, together with, once more, the whole of the yellow. Observe
further that the remaining bulls, the dappled, were equal to a sixth
part of the white and a seventh, together with all the yellow. These
were the proportions of the cows: The white were precisely equal to the
third part and a fourth of the whole herd of the black; while the black
were equal to the fourth part once more of the dappled and with it a
fifth part, when all, including the bulls went to pasture together. Now
the dappled in four parts8 were equal in number to a fifth part and a
sixth of the yellow herd. Finally the yellow were in number equal to
a sixth part and a seventh of the white herd. If thou canst accurately
tell, O stranger, the number of cattle of the Sun, giving separately the
number of well-fed bulls and again the number of females according
to each colour, thou wouldst not be called unskilled or ignorant of
numbers, but not yet shall thou be numbered among the wise. But
come, understand also all these conditions regarding the cows of the
Sun. When the white bulls mingled their number with the black, they
stood firm, equal in depth and breadth, and the plains of Thrinacia,
stretching far in all ways, were filled with their multitude. Again,
when the yellow and the dappled bulls were gathered into one herd
they stood in such a manner that their number, beginning from one,
grew slowly greater till it completed a triangular figure, there being
no bulls of other colours in their midst nor none of them lacking.
If thou art able, O stranger, to find out all these things and gather
them together in your mind, giving all the relations, thou shalt depart
crowned with glory and knowing that thou hast been adjudged perfect
in this species of wisdom.
\end{itemize}

If $W,X,Y,Z$ represents the number of white, black, yellow,
dappled bulls, you will get 
a systems of 7 linear equations, the first two being 
\begin{align*}
  W &= (1/2 + 1/3)X + Z \\
  X &= (1/4 + 1/5)Y + Z
\end{align*}
together with some constraints such as $W + X$ must be a square.
After some manipulations, it can be shown that the equation to solve looks like
\[
  x^2 - 410286423278424 y^2 = 1
\]
What was Archimedes thinking? You are find information on the Archimedes Cattle Problem on the web.]
}


  \solutionlink{sol:semigroup-associativity-0}
  \qed
\end{ex} 
\begin{python0}
from solutions import *
add(label="ex:semigroup-associativity-0",
    srcfilename='exercises/semigroup-associativity-0/answer.tex') 
\end{python0}


\begin{ex} 
  \label{ex:semigroup-associativity-0}
  \tinysidebar{\debug{exercises/{exercise-12/question.tex}}}
\mbox{}
  \begin{myenum}
  \item
    Solve
    \[
      5x^2 + y^2 = 3
    \]
    % mod 5, squares = 0^2=0, 1^2=1, 2^2=4, 3^2=4, 4^2=1
    (HINT: You don't really need number theory for this one. Why?
    But if you want to, imitate the solution for the previous
    problem.)
  \item
    Solve
    \[
      11y^2 - 5x^2 = 3
    \]
    % mod 5, squares = 0,1,4
    (HINT: This is just a slight change from the
    previous problem. \textit{But} now you need number theory. Try mod 4.
    If it does not work, try mod 5. Repeat.)
    % 3x^2 + y^2 = 3
    % {0,3} + {0,1} = 3
    % 0, 1, 3, 0 = 3
    % So x = 1, y = 0 (4)
    %
    % y^2=3 (5)
  \item
    Solve
    \[
      y^2 - 5x^2 = 2
    \]
    % mod 5, squares = 0^2=0, 1^2=1, 2^2=4, 3^2=4, 4^2=1    

  \item
    What about this one:
    \[
      x^2 - 5y^2 = 1
    \]    
  \end{myenum}
  {\scriptsize
[\textsc{Aside.}
Integer solutions to $x^2 - dy^2 = 1$ has been studied since at least 400BC.
This equation appear the Cattle Problem of Archimedes:

\begin{itemize}
  \item[]
If thou art diligent and wise, O stranger, compute the number of
cattle of the Sun, who once upon a time grazed on the fields of the
Thrinacian isle of Sicily, divided into four herds of different colours,
one milk white, another a glossy black, the third yellow and the last
dappled. In each herd were bulls, mighty in number according to these
proportions: Understand, stranger, that the white bulls were equal to
a half and a third of the black together with the whole of the yellow,
while the black were equal to the fourth part of the dappled and
a fifth, together with, once more, the whole of the yellow. Observe
further that the remaining bulls, the dappled, were equal to a sixth
part of the white and a seventh, together with all the yellow. These
were the proportions of the cows: The white were precisely equal to the
third part and a fourth of the whole herd of the black; while the black
were equal to the fourth part once more of the dappled and with it a
fifth part, when all, including the bulls went to pasture together. Now
the dappled in four parts8 were equal in number to a fifth part and a
sixth of the yellow herd. Finally the yellow were in number equal to
a sixth part and a seventh of the white herd. If thou canst accurately
tell, O stranger, the number of cattle of the Sun, giving separately the
number of well-fed bulls and again the number of females according
to each colour, thou wouldst not be called unskilled or ignorant of
numbers, but not yet shall thou be numbered among the wise. But
come, understand also all these conditions regarding the cows of the
Sun. When the white bulls mingled their number with the black, they
stood firm, equal in depth and breadth, and the plains of Thrinacia,
stretching far in all ways, were filled with their multitude. Again,
when the yellow and the dappled bulls were gathered into one herd
they stood in such a manner that their number, beginning from one,
grew slowly greater till it completed a triangular figure, there being
no bulls of other colours in their midst nor none of them lacking.
If thou art able, O stranger, to find out all these things and gather
them together in your mind, giving all the relations, thou shalt depart
crowned with glory and knowing that thou hast been adjudged perfect
in this species of wisdom.
\end{itemize}

If $W,X,Y,Z$ represents the number of white, black, yellow,
dappled bulls, you will get 
a systems of 7 linear equations, the first two being 
\begin{align*}
  W &= (1/2 + 1/3)X + Z \\
  X &= (1/4 + 1/5)Y + Z
\end{align*}
together with some constraints such as $W + X$ must be a square.
After some manipulations, it can be shown that the equation to solve looks like
\[
  x^2 - 410286423278424 y^2 = 1
\]
What was Archimedes thinking? You are find information on the Archimedes Cattle Problem on the web.]
}


  \solutionlink{sol:semigroup-associativity-0}
  \qed
\end{ex} 
\begin{python0}
from solutions import *
add(label="ex:semigroup-associativity-0",
    srcfilename='exercises/semigroup-associativity-0/answer.tex') 
\end{python0}



