\sectionthree{Linear feedback shift register}
\begin{python0}
from solutions import *; clear()
\end{python0}

Recall that the
one time pad uses exclusive-or to operator on
bit sequences.
Note that the exclusive-or is the same as
addition mod 2!!!
In other words addition \textit{in mod 2}
\begin{align*}
  0 + 0 \equiv 0 \pmod{2} \\
  0 + 1 \equiv 1 \pmod{2} \\
  1 + 0 \equiv 1 \pmod{2} \\
  1 + 0 \equiv 0 \pmod{2} \\
\end{align*}
is the same as exclusive-or operation \textit{on bits}:
\begin{align*}
  0 \oplus 0 = 0  \\
  0 \oplus 1 = 1  \\
  1 \oplus 0 = 1  \\
  1 \oplus 0 = 0 
\end{align*}
(I'm using $\oplus$ for exclusive-or bit operator -- that's pretty standard.)

Now I'm going to this:
first I define the following bit sequence of length 5:
\[
  x_1x_2x_3x_4x_5 = 10110
\]
which is the same as defining integer $x_1, ..., x_5$ in $\Z/2$.
Then I define
\[
  x_{n + 6} \equiv x_{n+1} + x_{n+2} + x_{n+4} \pmod{2}
\]
for $n \geq 0$.
For instance
\[
  x_{6} \equiv x_{1} + x_{2} + x_{4} \equiv 1 + 0 + 1 \equiv 0 \pmod{2} 
\]
Here are next about 15: 
\begin{align*}
  x_{6} &\equiv x_{1} + x_{2} + x_{4} \equiv 1 + 0 + 1 \equiv 0  & \pmod{2} \\
  x_{7} &\equiv x_{2} + x_{3} + x_{5} \equiv 0 + 1 + 0 \equiv 1  & \pmod{2} \\
  x_{8} &\equiv x_{3} + x_{4} + x_{6} \equiv 1 + 1 + 0 \equiv 0  & \pmod{2} \\
  x_{9} &\equiv x_{4} + x_{5} + x_{7} \equiv 1 + 0 + 1 \equiv 0  & \pmod{2} \\
  x_{10} &\equiv x_{5} + x_{6} + x_{8} \equiv 0 + 0 + 0 \equiv 0 & \pmod{2} \\
  x_{11} &\equiv x_{6} + x_{7} + x_{9} \equiv 0 + 1 + 0 \equiv 1 & \pmod{2} \\
  x_{12} &\equiv x_{7} + x_{8} + x_{10} \equiv 1 + 0 + 1 \equiv 1 & \pmod{2} \\
  x_{13} &\equiv x_{8} + x_{9} + x_{11} \equiv 0 + 0 + 1 \equiv 1 & \pmod{2} \\
  x_{14} &\equiv x_{9} + x_{10} + x_{12} \equiv 0 + 0 + 1 \equiv 1 & \pmod{2} \\
  x_{15} &\equiv x_{10} + x_{11} + x_{13} \equiv 0 + 1 + 1 \equiv 0 & \pmod{2} \\
  x_{16} &\equiv x_{11} + x_{12} + x_{14} \equiv 1 + 1 + 1 \equiv 1 & \pmod{2} \\
  x_{17} &\equiv x_{12} + x_{13} + x_{15} \equiv 1 + 1 + 0 \equiv 0 & \pmod{2} \\
  x_{18} &\equiv x_{13} + x_{14} + x_{16} \equiv 1 + 1 + 1 \equiv 1 & \pmod{2} \\
  x_{19} &\equiv x_{14} + x_{15} + x_{17} \equiv 1 + 0 + 0 \equiv 1 & \pmod{2} \\
  x_{20} &\equiv x_{15} + x_{16} + x_{18} \equiv 0 + 1 + 1 \equiv 0 & \pmod{2} \\
\end{align*}
More generally, after defining $x_1, ..., x_5$ (the initial conditions) you can
generated the sequence $x_i$ using
\[
  x_{n + 6} \equiv c_1 x_{n+1} + c_2 x_{n+2} + c_3 x_{n+3} + c_4x_{n+4} + c_5x_{n+5} \pmod{2}
\]
for $n \geq 0$ for constants $c_1, ..., c_5$ in $\Z/2$.
I will say that this is linear relation has \defone{degree 5}.
Even more generally, you can have any number of bits for the initial condition.
Say you begin with $x_1, ..., x_k$ (the initial condition) and
the relation is
\[
  x_{n + k + 1} \equiv c_1 x_{n+1} + c_2 x_{n+2} + c_3 x_{n+3} + \cdots + c_kx_{n+k} \pmod{2}
\]

LSRFs are very easy to implement in both hardware and software
and they are extremely fast
(they just access bits and XOR them).
The LSRF generator itself need to remember the $k$ bits
$c_1, ..., c_k$ (which is fixed)
and the $k$ bits of the sequence so far
$x_{n + 1}, ..., x_{n + k}$ in order to generate the next
bit $x_{n + k + 1}$.

\begin{comment}
x = [1,0,1,1,0]
c = [1,1,0,1,0]
# x_{n + 6} = 1x_{n+1} + 1x_{n+2} + 0x_{n+3} + 1x_{n+4} + 0x_{n+5}
def f(c, x):
    y = x[-5:]
    s = sum([a*b for (a,b) in zip(c,y)]) % 2
    x.append(s)

#print x
for i in range(1000):
    f(c, x)

found = False
y = None   
for length in range(1, 100):
    y = x[:length]
    z = x[length:]
    numtimes = 0
    while 1:
        y0 = z[:length]
        if y0 != y:
            break
        numtimes += 1
        z = z[length:]
        if len(z) < length:
            found = True
    if found: break

if found:
    print y
    print length
    print numtimes
\end{comment}

In the above example, the sequence from $x_1$ to $x_{20}$ is
\[
10110010001111010110010001111010110...
\]
Notice that the pattern repeats itself:
\[
101100100011110\ \ 101100100011110\ \ 10110...
\]
The period is 15.

The problem with the one-time pad is that you need to
generate a random sequence of 0s and 1s.
You can see that with 5 bits
\[
  x_1 x_2 x_3 x_4 x_5 = 10110
\]
and the relation
\[
  x_{n + 6} \equiv x_{n+1} + x_{n+2} + x_{n+4} \pmod{2}
\]
which involves (1,1,0,1,0) (5 bits), a total of 10 bits)  we can generate
\[
  101100100011110
\]
which has length 15.
The 15 bits is somewhat random -- we say that the 15 bits are
\defone{pseudorandom}.
Therefore LFSR can be used to generate a pseudorandom bit sequence,
which can 
used, for instance, as a key for the one-time pad.
Of course you want to find a LSFR with extremely long periods.


\begin{ex} 
  \label{ex:semigroup-associativity-0}
  \tinysidebar{\debug{exercises/{exercise-12/question.tex}}}
\mbox{}
  \begin{myenum}
  \item
    Solve
    \[
      5x^2 + y^2 = 3
    \]
    % mod 5, squares = 0^2=0, 1^2=1, 2^2=4, 3^2=4, 4^2=1
    (HINT: You don't really need number theory for this one. Why?
    But if you want to, imitate the solution for the previous
    problem.)
  \item
    Solve
    \[
      11y^2 - 5x^2 = 3
    \]
    % mod 5, squares = 0,1,4
    (HINT: This is just a slight change from the
    previous problem. \textit{But} now you need number theory. Try mod 4.
    If it does not work, try mod 5. Repeat.)
    % 3x^2 + y^2 = 3
    % {0,3} + {0,1} = 3
    % 0, 1, 3, 0 = 3
    % So x = 1, y = 0 (4)
    %
    % y^2=3 (5)
  \item
    Solve
    \[
      y^2 - 5x^2 = 2
    \]
    % mod 5, squares = 0^2=0, 1^2=1, 2^2=4, 3^2=4, 4^2=1    

  \item
    What about this one:
    \[
      x^2 - 5y^2 = 1
    \]    
  \end{myenum}
  {\scriptsize
[\textsc{Aside.}
Integer solutions to $x^2 - dy^2 = 1$ has been studied since at least 400BC.
This equation appear the Cattle Problem of Archimedes:

\begin{itemize}
  \item[]
If thou art diligent and wise, O stranger, compute the number of
cattle of the Sun, who once upon a time grazed on the fields of the
Thrinacian isle of Sicily, divided into four herds of different colours,
one milk white, another a glossy black, the third yellow and the last
dappled. In each herd were bulls, mighty in number according to these
proportions: Understand, stranger, that the white bulls were equal to
a half and a third of the black together with the whole of the yellow,
while the black were equal to the fourth part of the dappled and
a fifth, together with, once more, the whole of the yellow. Observe
further that the remaining bulls, the dappled, were equal to a sixth
part of the white and a seventh, together with all the yellow. These
were the proportions of the cows: The white were precisely equal to the
third part and a fourth of the whole herd of the black; while the black
were equal to the fourth part once more of the dappled and with it a
fifth part, when all, including the bulls went to pasture together. Now
the dappled in four parts8 were equal in number to a fifth part and a
sixth of the yellow herd. Finally the yellow were in number equal to
a sixth part and a seventh of the white herd. If thou canst accurately
tell, O stranger, the number of cattle of the Sun, giving separately the
number of well-fed bulls and again the number of females according
to each colour, thou wouldst not be called unskilled or ignorant of
numbers, but not yet shall thou be numbered among the wise. But
come, understand also all these conditions regarding the cows of the
Sun. When the white bulls mingled their number with the black, they
stood firm, equal in depth and breadth, and the plains of Thrinacia,
stretching far in all ways, were filled with their multitude. Again,
when the yellow and the dappled bulls were gathered into one herd
they stood in such a manner that their number, beginning from one,
grew slowly greater till it completed a triangular figure, there being
no bulls of other colours in their midst nor none of them lacking.
If thou art able, O stranger, to find out all these things and gather
them together in your mind, giving all the relations, thou shalt depart
crowned with glory and knowing that thou hast been adjudged perfect
in this species of wisdom.
\end{itemize}

If $W,X,Y,Z$ represents the number of white, black, yellow,
dappled bulls, you will get 
a systems of 7 linear equations, the first two being 
\begin{align*}
  W &= (1/2 + 1/3)X + Z \\
  X &= (1/4 + 1/5)Y + Z
\end{align*}
together with some constraints such as $W + X$ must be a square.
After some manipulations, it can be shown that the equation to solve looks like
\[
  x^2 - 410286423278424 y^2 = 1
\]
What was Archimedes thinking? You are find information on the Archimedes Cattle Problem on the web.]
}


  \solutionlink{sol:semigroup-associativity-0}
  \qed
\end{ex} 
\begin{python0}
from solutions import *
add(label="ex:semigroup-associativity-0",
    srcfilename='exercises/semigroup-associativity-0/answer.tex') 
\end{python0}


\begin{ex} 
  \label{ex:semigroup-associativity-0}
  \tinysidebar{\debug{exercises/{exercise-12/question.tex}}}
\mbox{}
  \begin{myenum}
  \item
    Solve
    \[
      5x^2 + y^2 = 3
    \]
    % mod 5, squares = 0^2=0, 1^2=1, 2^2=4, 3^2=4, 4^2=1
    (HINT: You don't really need number theory for this one. Why?
    But if you want to, imitate the solution for the previous
    problem.)
  \item
    Solve
    \[
      11y^2 - 5x^2 = 3
    \]
    % mod 5, squares = 0,1,4
    (HINT: This is just a slight change from the
    previous problem. \textit{But} now you need number theory. Try mod 4.
    If it does not work, try mod 5. Repeat.)
    % 3x^2 + y^2 = 3
    % {0,3} + {0,1} = 3
    % 0, 1, 3, 0 = 3
    % So x = 1, y = 0 (4)
    %
    % y^2=3 (5)
  \item
    Solve
    \[
      y^2 - 5x^2 = 2
    \]
    % mod 5, squares = 0^2=0, 1^2=1, 2^2=4, 3^2=4, 4^2=1    

  \item
    What about this one:
    \[
      x^2 - 5y^2 = 1
    \]    
  \end{myenum}
  {\scriptsize
[\textsc{Aside.}
Integer solutions to $x^2 - dy^2 = 1$ has been studied since at least 400BC.
This equation appear the Cattle Problem of Archimedes:

\begin{itemize}
  \item[]
If thou art diligent and wise, O stranger, compute the number of
cattle of the Sun, who once upon a time grazed on the fields of the
Thrinacian isle of Sicily, divided into four herds of different colours,
one milk white, another a glossy black, the third yellow and the last
dappled. In each herd were bulls, mighty in number according to these
proportions: Understand, stranger, that the white bulls were equal to
a half and a third of the black together with the whole of the yellow,
while the black were equal to the fourth part of the dappled and
a fifth, together with, once more, the whole of the yellow. Observe
further that the remaining bulls, the dappled, were equal to a sixth
part of the white and a seventh, together with all the yellow. These
were the proportions of the cows: The white were precisely equal to the
third part and a fourth of the whole herd of the black; while the black
were equal to the fourth part once more of the dappled and with it a
fifth part, when all, including the bulls went to pasture together. Now
the dappled in four parts8 were equal in number to a fifth part and a
sixth of the yellow herd. Finally the yellow were in number equal to
a sixth part and a seventh of the white herd. If thou canst accurately
tell, O stranger, the number of cattle of the Sun, giving separately the
number of well-fed bulls and again the number of females according
to each colour, thou wouldst not be called unskilled or ignorant of
numbers, but not yet shall thou be numbered among the wise. But
come, understand also all these conditions regarding the cows of the
Sun. When the white bulls mingled their number with the black, they
stood firm, equal in depth and breadth, and the plains of Thrinacia,
stretching far in all ways, were filled with their multitude. Again,
when the yellow and the dappled bulls were gathered into one herd
they stood in such a manner that their number, beginning from one,
grew slowly greater till it completed a triangular figure, there being
no bulls of other colours in their midst nor none of them lacking.
If thou art able, O stranger, to find out all these things and gather
them together in your mind, giving all the relations, thou shalt depart
crowned with glory and knowing that thou hast been adjudged perfect
in this species of wisdom.
\end{itemize}

If $W,X,Y,Z$ represents the number of white, black, yellow,
dappled bulls, you will get 
a systems of 7 linear equations, the first two being 
\begin{align*}
  W &= (1/2 + 1/3)X + Z \\
  X &= (1/4 + 1/5)Y + Z
\end{align*}
together with some constraints such as $W + X$ must be a square.
After some manipulations, it can be shown that the equation to solve looks like
\[
  x^2 - 410286423278424 y^2 = 1
\]
What was Archimedes thinking? You are find information on the Archimedes Cattle Problem on the web.]
}


  \solutionlink{sol:semigroup-associativity-0}
  \qed
\end{ex} 
\begin{python0}
from solutions import *
add(label="ex:semigroup-associativity-0",
    srcfilename='exercises/semigroup-associativity-0/answer.tex') 
\end{python0}


\begin{ex} 
  \label{ex:semigroup-associativity-0}
  \tinysidebar{\debug{exercises/{exercise-12/question.tex}}}
\mbox{}
  \begin{myenum}
  \item
    Solve
    \[
      5x^2 + y^2 = 3
    \]
    % mod 5, squares = 0^2=0, 1^2=1, 2^2=4, 3^2=4, 4^2=1
    (HINT: You don't really need number theory for this one. Why?
    But if you want to, imitate the solution for the previous
    problem.)
  \item
    Solve
    \[
      11y^2 - 5x^2 = 3
    \]
    % mod 5, squares = 0,1,4
    (HINT: This is just a slight change from the
    previous problem. \textit{But} now you need number theory. Try mod 4.
    If it does not work, try mod 5. Repeat.)
    % 3x^2 + y^2 = 3
    % {0,3} + {0,1} = 3
    % 0, 1, 3, 0 = 3
    % So x = 1, y = 0 (4)
    %
    % y^2=3 (5)
  \item
    Solve
    \[
      y^2 - 5x^2 = 2
    \]
    % mod 5, squares = 0^2=0, 1^2=1, 2^2=4, 3^2=4, 4^2=1    

  \item
    What about this one:
    \[
      x^2 - 5y^2 = 1
    \]    
  \end{myenum}
  {\scriptsize
[\textsc{Aside.}
Integer solutions to $x^2 - dy^2 = 1$ has been studied since at least 400BC.
This equation appear the Cattle Problem of Archimedes:

\begin{itemize}
  \item[]
If thou art diligent and wise, O stranger, compute the number of
cattle of the Sun, who once upon a time grazed on the fields of the
Thrinacian isle of Sicily, divided into four herds of different colours,
one milk white, another a glossy black, the third yellow and the last
dappled. In each herd were bulls, mighty in number according to these
proportions: Understand, stranger, that the white bulls were equal to
a half and a third of the black together with the whole of the yellow,
while the black were equal to the fourth part of the dappled and
a fifth, together with, once more, the whole of the yellow. Observe
further that the remaining bulls, the dappled, were equal to a sixth
part of the white and a seventh, together with all the yellow. These
were the proportions of the cows: The white were precisely equal to the
third part and a fourth of the whole herd of the black; while the black
were equal to the fourth part once more of the dappled and with it a
fifth part, when all, including the bulls went to pasture together. Now
the dappled in four parts8 were equal in number to a fifth part and a
sixth of the yellow herd. Finally the yellow were in number equal to
a sixth part and a seventh of the white herd. If thou canst accurately
tell, O stranger, the number of cattle of the Sun, giving separately the
number of well-fed bulls and again the number of females according
to each colour, thou wouldst not be called unskilled or ignorant of
numbers, but not yet shall thou be numbered among the wise. But
come, understand also all these conditions regarding the cows of the
Sun. When the white bulls mingled their number with the black, they
stood firm, equal in depth and breadth, and the plains of Thrinacia,
stretching far in all ways, were filled with their multitude. Again,
when the yellow and the dappled bulls were gathered into one herd
they stood in such a manner that their number, beginning from one,
grew slowly greater till it completed a triangular figure, there being
no bulls of other colours in their midst nor none of them lacking.
If thou art able, O stranger, to find out all these things and gather
them together in your mind, giving all the relations, thou shalt depart
crowned with glory and knowing that thou hast been adjudged perfect
in this species of wisdom.
\end{itemize}

If $W,X,Y,Z$ represents the number of white, black, yellow,
dappled bulls, you will get 
a systems of 7 linear equations, the first two being 
\begin{align*}
  W &= (1/2 + 1/3)X + Z \\
  X &= (1/4 + 1/5)Y + Z
\end{align*}
together with some constraints such as $W + X$ must be a square.
After some manipulations, it can be shown that the equation to solve looks like
\[
  x^2 - 410286423278424 y^2 = 1
\]
What was Archimedes thinking? You are find information on the Archimedes Cattle Problem on the web.]
}


  \solutionlink{sol:semigroup-associativity-0}
  \qed
\end{ex} 
\begin{python0}
from solutions import *
add(label="ex:semigroup-associativity-0",
    srcfilename='exercises/semigroup-associativity-0/answer.tex') 
\end{python0}


\begin{ex} 
  \label{ex:semigroup-associativity-0}
  \tinysidebar{\debug{exercises/{exercise-12/question.tex}}}
\mbox{}
  \begin{myenum}
  \item
    Solve
    \[
      5x^2 + y^2 = 3
    \]
    % mod 5, squares = 0^2=0, 1^2=1, 2^2=4, 3^2=4, 4^2=1
    (HINT: You don't really need number theory for this one. Why?
    But if you want to, imitate the solution for the previous
    problem.)
  \item
    Solve
    \[
      11y^2 - 5x^2 = 3
    \]
    % mod 5, squares = 0,1,4
    (HINT: This is just a slight change from the
    previous problem. \textit{But} now you need number theory. Try mod 4.
    If it does not work, try mod 5. Repeat.)
    % 3x^2 + y^2 = 3
    % {0,3} + {0,1} = 3
    % 0, 1, 3, 0 = 3
    % So x = 1, y = 0 (4)
    %
    % y^2=3 (5)
  \item
    Solve
    \[
      y^2 - 5x^2 = 2
    \]
    % mod 5, squares = 0^2=0, 1^2=1, 2^2=4, 3^2=4, 4^2=1    

  \item
    What about this one:
    \[
      x^2 - 5y^2 = 1
    \]    
  \end{myenum}
  {\scriptsize
[\textsc{Aside.}
Integer solutions to $x^2 - dy^2 = 1$ has been studied since at least 400BC.
This equation appear the Cattle Problem of Archimedes:

\begin{itemize}
  \item[]
If thou art diligent and wise, O stranger, compute the number of
cattle of the Sun, who once upon a time grazed on the fields of the
Thrinacian isle of Sicily, divided into four herds of different colours,
one milk white, another a glossy black, the third yellow and the last
dappled. In each herd were bulls, mighty in number according to these
proportions: Understand, stranger, that the white bulls were equal to
a half and a third of the black together with the whole of the yellow,
while the black were equal to the fourth part of the dappled and
a fifth, together with, once more, the whole of the yellow. Observe
further that the remaining bulls, the dappled, were equal to a sixth
part of the white and a seventh, together with all the yellow. These
were the proportions of the cows: The white were precisely equal to the
third part and a fourth of the whole herd of the black; while the black
were equal to the fourth part once more of the dappled and with it a
fifth part, when all, including the bulls went to pasture together. Now
the dappled in four parts8 were equal in number to a fifth part and a
sixth of the yellow herd. Finally the yellow were in number equal to
a sixth part and a seventh of the white herd. If thou canst accurately
tell, O stranger, the number of cattle of the Sun, giving separately the
number of well-fed bulls and again the number of females according
to each colour, thou wouldst not be called unskilled or ignorant of
numbers, but not yet shall thou be numbered among the wise. But
come, understand also all these conditions regarding the cows of the
Sun. When the white bulls mingled their number with the black, they
stood firm, equal in depth and breadth, and the plains of Thrinacia,
stretching far in all ways, were filled with their multitude. Again,
when the yellow and the dappled bulls were gathered into one herd
they stood in such a manner that their number, beginning from one,
grew slowly greater till it completed a triangular figure, there being
no bulls of other colours in their midst nor none of them lacking.
If thou art able, O stranger, to find out all these things and gather
them together in your mind, giving all the relations, thou shalt depart
crowned with glory and knowing that thou hast been adjudged perfect
in this species of wisdom.
\end{itemize}

If $W,X,Y,Z$ represents the number of white, black, yellow,
dappled bulls, you will get 
a systems of 7 linear equations, the first two being 
\begin{align*}
  W &= (1/2 + 1/3)X + Z \\
  X &= (1/4 + 1/5)Y + Z
\end{align*}
together with some constraints such as $W + X$ must be a square.
After some manipulations, it can be shown that the equation to solve looks like
\[
  x^2 - 410286423278424 y^2 = 1
\]
What was Archimedes thinking? You are find information on the Archimedes Cattle Problem on the web.]
}


  \solutionlink{sol:semigroup-associativity-0}
  \qed
\end{ex} 
\begin{python0}
from solutions import *
add(label="ex:semigroup-associativity-0",
    srcfilename='exercises/semigroup-associativity-0/answer.tex') 
\end{python0}


\begin{ex} 
  \label{ex:semigroup-associativity-0}
  \tinysidebar{\debug{exercises/{exercise-12/question.tex}}}
\mbox{}
  \begin{myenum}
  \item
    Solve
    \[
      5x^2 + y^2 = 3
    \]
    % mod 5, squares = 0^2=0, 1^2=1, 2^2=4, 3^2=4, 4^2=1
    (HINT: You don't really need number theory for this one. Why?
    But if you want to, imitate the solution for the previous
    problem.)
  \item
    Solve
    \[
      11y^2 - 5x^2 = 3
    \]
    % mod 5, squares = 0,1,4
    (HINT: This is just a slight change from the
    previous problem. \textit{But} now you need number theory. Try mod 4.
    If it does not work, try mod 5. Repeat.)
    % 3x^2 + y^2 = 3
    % {0,3} + {0,1} = 3
    % 0, 1, 3, 0 = 3
    % So x = 1, y = 0 (4)
    %
    % y^2=3 (5)
  \item
    Solve
    \[
      y^2 - 5x^2 = 2
    \]
    % mod 5, squares = 0^2=0, 1^2=1, 2^2=4, 3^2=4, 4^2=1    

  \item
    What about this one:
    \[
      x^2 - 5y^2 = 1
    \]    
  \end{myenum}
  {\scriptsize
[\textsc{Aside.}
Integer solutions to $x^2 - dy^2 = 1$ has been studied since at least 400BC.
This equation appear the Cattle Problem of Archimedes:

\begin{itemize}
  \item[]
If thou art diligent and wise, O stranger, compute the number of
cattle of the Sun, who once upon a time grazed on the fields of the
Thrinacian isle of Sicily, divided into four herds of different colours,
one milk white, another a glossy black, the third yellow and the last
dappled. In each herd were bulls, mighty in number according to these
proportions: Understand, stranger, that the white bulls were equal to
a half and a third of the black together with the whole of the yellow,
while the black were equal to the fourth part of the dappled and
a fifth, together with, once more, the whole of the yellow. Observe
further that the remaining bulls, the dappled, were equal to a sixth
part of the white and a seventh, together with all the yellow. These
were the proportions of the cows: The white were precisely equal to the
third part and a fourth of the whole herd of the black; while the black
were equal to the fourth part once more of the dappled and with it a
fifth part, when all, including the bulls went to pasture together. Now
the dappled in four parts8 were equal in number to a fifth part and a
sixth of the yellow herd. Finally the yellow were in number equal to
a sixth part and a seventh of the white herd. If thou canst accurately
tell, O stranger, the number of cattle of the Sun, giving separately the
number of well-fed bulls and again the number of females according
to each colour, thou wouldst not be called unskilled or ignorant of
numbers, but not yet shall thou be numbered among the wise. But
come, understand also all these conditions regarding the cows of the
Sun. When the white bulls mingled their number with the black, they
stood firm, equal in depth and breadth, and the plains of Thrinacia,
stretching far in all ways, were filled with their multitude. Again,
when the yellow and the dappled bulls were gathered into one herd
they stood in such a manner that their number, beginning from one,
grew slowly greater till it completed a triangular figure, there being
no bulls of other colours in their midst nor none of them lacking.
If thou art able, O stranger, to find out all these things and gather
them together in your mind, giving all the relations, thou shalt depart
crowned with glory and knowing that thou hast been adjudged perfect
in this species of wisdom.
\end{itemize}

If $W,X,Y,Z$ represents the number of white, black, yellow,
dappled bulls, you will get 
a systems of 7 linear equations, the first two being 
\begin{align*}
  W &= (1/2 + 1/3)X + Z \\
  X &= (1/4 + 1/5)Y + Z
\end{align*}
together with some constraints such as $W + X$ must be a square.
After some manipulations, it can be shown that the equation to solve looks like
\[
  x^2 - 410286423278424 y^2 = 1
\]
What was Archimedes thinking? You are find information on the Archimedes Cattle Problem on the web.]
}


  \solutionlink{sol:semigroup-associativity-0}
  \qed
\end{ex} 
\begin{python0}
from solutions import *
add(label="ex:semigroup-associativity-0",
    srcfilename='exercises/semigroup-associativity-0/answer.tex') 
\end{python0}


\begin{ex} 
  \label{ex:semigroup-associativity-0}
  \tinysidebar{\debug{exercises/{exercise-12/question.tex}}}
\mbox{}
  \begin{myenum}
  \item
    Solve
    \[
      5x^2 + y^2 = 3
    \]
    % mod 5, squares = 0^2=0, 1^2=1, 2^2=4, 3^2=4, 4^2=1
    (HINT: You don't really need number theory for this one. Why?
    But if you want to, imitate the solution for the previous
    problem.)
  \item
    Solve
    \[
      11y^2 - 5x^2 = 3
    \]
    % mod 5, squares = 0,1,4
    (HINT: This is just a slight change from the
    previous problem. \textit{But} now you need number theory. Try mod 4.
    If it does not work, try mod 5. Repeat.)
    % 3x^2 + y^2 = 3
    % {0,3} + {0,1} = 3
    % 0, 1, 3, 0 = 3
    % So x = 1, y = 0 (4)
    %
    % y^2=3 (5)
  \item
    Solve
    \[
      y^2 - 5x^2 = 2
    \]
    % mod 5, squares = 0^2=0, 1^2=1, 2^2=4, 3^2=4, 4^2=1    

  \item
    What about this one:
    \[
      x^2 - 5y^2 = 1
    \]    
  \end{myenum}
  {\scriptsize
[\textsc{Aside.}
Integer solutions to $x^2 - dy^2 = 1$ has been studied since at least 400BC.
This equation appear the Cattle Problem of Archimedes:

\begin{itemize}
  \item[]
If thou art diligent and wise, O stranger, compute the number of
cattle of the Sun, who once upon a time grazed on the fields of the
Thrinacian isle of Sicily, divided into four herds of different colours,
one milk white, another a glossy black, the third yellow and the last
dappled. In each herd were bulls, mighty in number according to these
proportions: Understand, stranger, that the white bulls were equal to
a half and a third of the black together with the whole of the yellow,
while the black were equal to the fourth part of the dappled and
a fifth, together with, once more, the whole of the yellow. Observe
further that the remaining bulls, the dappled, were equal to a sixth
part of the white and a seventh, together with all the yellow. These
were the proportions of the cows: The white were precisely equal to the
third part and a fourth of the whole herd of the black; while the black
were equal to the fourth part once more of the dappled and with it a
fifth part, when all, including the bulls went to pasture together. Now
the dappled in four parts8 were equal in number to a fifth part and a
sixth of the yellow herd. Finally the yellow were in number equal to
a sixth part and a seventh of the white herd. If thou canst accurately
tell, O stranger, the number of cattle of the Sun, giving separately the
number of well-fed bulls and again the number of females according
to each colour, thou wouldst not be called unskilled or ignorant of
numbers, but not yet shall thou be numbered among the wise. But
come, understand also all these conditions regarding the cows of the
Sun. When the white bulls mingled their number with the black, they
stood firm, equal in depth and breadth, and the plains of Thrinacia,
stretching far in all ways, were filled with their multitude. Again,
when the yellow and the dappled bulls were gathered into one herd
they stood in such a manner that their number, beginning from one,
grew slowly greater till it completed a triangular figure, there being
no bulls of other colours in their midst nor none of them lacking.
If thou art able, O stranger, to find out all these things and gather
them together in your mind, giving all the relations, thou shalt depart
crowned with glory and knowing that thou hast been adjudged perfect
in this species of wisdom.
\end{itemize}

If $W,X,Y,Z$ represents the number of white, black, yellow,
dappled bulls, you will get 
a systems of 7 linear equations, the first two being 
\begin{align*}
  W &= (1/2 + 1/3)X + Z \\
  X &= (1/4 + 1/5)Y + Z
\end{align*}
together with some constraints such as $W + X$ must be a square.
After some manipulations, it can be shown that the equation to solve looks like
\[
  x^2 - 410286423278424 y^2 = 1
\]
What was Archimedes thinking? You are find information on the Archimedes Cattle Problem on the web.]
}


  \solutionlink{sol:semigroup-associativity-0}
  \qed
\end{ex} 
\begin{python0}
from solutions import *
add(label="ex:semigroup-associativity-0",
    srcfilename='exercises/semigroup-associativity-0/answer.tex') 
\end{python0}

