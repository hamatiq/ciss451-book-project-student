\chapter{Introduction}

\section{Cryptography}

Obviously cryptography is about scrambling data to make it
meaningless to an eavesdropper and yet it can be descrambled by
the person you are sending your data to.

Easy right?

Well yes and no.

(By the way, before I begin, cryptography is not cybersecurity.)

Cryptography is a huge area in computer science, mathematics,
and engineering.
If you define cryptography as \lq\lq scrambling and descrambling"
of data, then in of itself, cryptography is already an extremely huge
area, in the sense that there's a lot of information in this subject.
And it's not just an academic subject.
It has extremely important consequences in the real world.
If there's no cryptography, then
there's no e-commerce.
More generally, there won't be any electronic communication
that requires privacy.
The world we know today will be extremely different if
cryptography does not exist.
For sure millions of jobs will cease to exist if
cryptography suddenly disappears or if some very powerful
adversary (a power superpower rogue country?) makes
cryptography useless.

Therefore cryptography is the cornerstone of
information security and a cornerstone of our society.

However you will see that cryptography is even larger than you think.
Besides \lq\lq scrambling and descrambling data" for
information security,
cryptography is also used for instance
in digital signatures which allows Amazon to 
say \lq\lq You signed this document to buy the XYZ
Complete Kit for Underwater Basket Weaving. Pay up. Don't deny it."

Cryptography can also be used to
implement secure voting 
so that if you were voted to be
the next president, the voting system can prove that the votes
are counted correctly,
and yet ... Tom Smith, the other candidate who did not win the election,
cannot possibly find out who voted for you and not him
so that Tom can't use his political power (if he has any) for retaliation.
So you can ensure the correctness of the voting process
\textit{and}
you prevent voting fraud.
It can even prevent buying votes --
if Tom Smith paid Harry Jones to vote for him,
after the voting process,
Harry cannot prove to Tom that he voted for Tom!
If Tom cannot have a guarantee of Harry's vote, he might think twice about
paying Harry.

You can also use cryptography to implement digital cash,
the electronic equivalent of our physical dollar bills.
Except that there's the benefit that digital cash cannot be counterfeited.
There's the benefit of privacy too -- Amazon, where
you bought your
\lq\lq XYZ Complete Kit for Underwater Basket Weaving", does not
know that it was you who bought it.
Neither can the bank or credit card agency that acts as the
intermediate payment gateway.
Blockchain-based cryptocurrency such as bitcoin and ether
have exploded in value in recent years.
Blockchain has applications beyond cryptocurrency.
It can be used to protect the integrity of data.

And if you are a spy working in a team that you don't fully trust (is there
a double agent or mole in your group?),
you can use cryptography to implement secret sharing so that
the door to the vault
that holds gold bullion is opened only when all the members are
present with their part of the key combination to unlock the door to the vault.

You do want to be sure that
your self-driving car will not follow an unauthenticated
message from your enemy to instruct your car to ram into the
nearest police car right?
Do you know that Tesla's AI chip has AES cipher built into it at the
hardware level so that it can communicate on the
car's controller area network securely?

And suppose you have invited a new algorithm/device that you know will be
worth billions.
But it's at the prototype stage and you need funds from
a venture capitalist.
How can you convince the venture capitalize that your idea
works without revealing it?
 
And wouldn't it be nice if you send an encrypted search to google,
google returns the results -- encrypted -- which you decrypt.
And google does not know what your search was and does not know what
the search results were.

And the list of interesting cryptographic problems go on and on.

As the world becomes more and more connected (electronically),
there's no doubt that more and more problems like the above will be
proposed and solved using cryptography.

You know the \lq\lq scrambling and descrambling" part of cryptography
just from reading novels or watching movies.
A \lq\lq scrambling and descrambling" algorithm is usually called a
cipher.
Most people equate cryptography with ciphers.
Cryptography is more than just about ciphers 
to achieve message confidentiality.
Cryptography is the study of real world problems
related to information protection in general using cryptographic tools.
Besides ciphers, the other two very important cryptographic tools
are cryptographic hash functions and
pseudorandom number generators.
Cryptographic hash functions is similar to the concept of hash functions in
the study of data structures and algorithms (CISS350 and CISS358).
except that
they are \lq\lq stronger".
I'll get into that later.
In the recipes to solve information protection problems,
other mathematical and algorithmic tools are used as well.

Cryptography is a very huge area of study and
research in this area is extremely active.
Researchers come from computer science, math, and engineering.


\section{Tools}

There are other benefits to studying cryptography besides working
in the area of information security.
The following are subjects/areas that I hope to touch on, if not all,
hopefully most:
\begin{itemize}

  \li
  To begin with, cryptography involves algorithms.
  And you can never get enough of algorithms.

  \li
  You will study probability theory which is incredibly important in the
  real world.
  Probability theory should be as important as calculus and
  algebra in college, but for some reason many schools do not emphasize
  it or do not teach it well.
  I have no idea why.
  Taking cryptography and advanced algorithms (CISS451 and 358)
  hopefully will repair some of that problem.

  \li
  You will study information theory which is an area that builds
  on top of probability theory.
  Information theory is one of the newest areas of study in
  computer science,
  math, and engineering.
  The concept of information theory was discovered only recently by
  Claude Shannon in 1948.
  Information theory can for instance predict for you how much data
  (the maximum data rate) that can be sent through a noisy data channel.
  It can also tell you what is the maximum data compression date.
  Information theory can also tell you if an encryption-decryption
  scheme is \lq\lq absolutely" secure.
  Although Shannon proposed this concept to study
  problems in math and engineering,
  information theory is now also used in physics in an area called
  quantum information theory.

  \li
  You'll be studying number theory. 
  You have seen a bit of number theory in discrete math.
  Many of the algorithms in crytography are number theoretic in nature.
  For instance you have seen the 
  Extended Euclidean Algorithm.
  I bet you have not implemented it.
  Well ... you'll be studying that in cryptography and
  seeing it in action.
  Then you'll see why the Extended Euclidean Algorithm is so important.
  Therefore the study of cryptography is partially
  a review of number theoretic concepts from discrete math.
  Practically speaking, number theory is used in the implementation of the
  RSA cipher.
  And lo and behold ... RSA depends on prime factorization, something that
  you learned in middle school and learned to program in CISS240.
  
  \li
  You'll be studying finite fields.
  You have actually seen finite fields before.
  Well ... actually you have only seen one single finite field.
  In CISS360 the values of a bit, i.e., $0$ and $1$,
  forms a field $\{0, 1\}$ called a binary field.
  But there are many other fields.
  A field is just a collection of values where you can
  add, subtract, multiply, and divide (except you can't
  divide by 0).
  You have seen many fields before, fields which are not finite.
  For instance the set of real numbers $\R$ is a field which is not finite.
  So are $\Q$ (field of rational numbers) and $\C$ (field of
  complex numbers).
  A finite field is just a field with finitely many values.
  They are very important not just in
  computer architecture.
  They appear in cryptography, digital signal processing, data compression,
  error correction codes, etc.

  \li
  You'll be studying groups.
  You have also seen groups before.
  If you think about real numbers, you see that you have two
  operations: addition and multiplication.
  (Subtraction is related to addition and division is related to
  multiplication.)
  A group is basically a set of values together with one operation.
  For instance the set of real numbers with addition is a group
  (just forget about the multiplication).
  While RSA uses number theory,
  certain cryptographic ciphers use
  groups.
  These are called group-based cryptography or discrete log cryptography.
  The standard group used (right now) is based on
  elliptic curves.
  At this point, elliptic curve cryptography (ECC) seems to be more
  popular in browser and SSL/TLS host.
  (If your browser is hitting a website that begins with \verb!htts!,
  then you are using TLS.)
  %https://malware.news/t/everyone-loves-curves-but-which-elliptic-curve-is-the-most-popular/17657
  ECC is also more suitable for the constrained environments such as
  smart cards, smart phones, and IoT.
  By the way, bitcoin uses ECC too.
  Besides cryptography, the theory of groups is extremely important
  and is used in computer science, engineering, physics, and chemistry.
  For instance group theory is used in
  computer vision and robotics.
  In physics and chemistry, group theory appears in quantum mechanics.
  In general when you study spatial structures, you will frequently
  find matrices and groups.

  \li
  A ring is just a field except that for fields you have
  addition, subtraction, multiplication, and division (by nonzero),
  while for rings you have all the four operations except you
  usually can't divide.
  The triad of \lq\lq groups, rings, and fields"
  forms a very important collection of structures for higher
  math, computer science, and engineering.
  The study of \lq\lq abstract algebra" is the study
  of groups, rings, and fields (and more).
  They are becoming more and more important in for instance
  algorithms, automata theory,
  and machine learning.
  By the way don't freak out: the word \lq\lq abstract" in abstract algebra
  does not mean more complicated.
  It means \lq\lq abstract away the non-essentials so that the
  theory is more applicable".
  It's similar to the concept of abstract base class in \cpp\ which is
  a general class interface applicable to more subclasses.

  \li
  Besides the groups-rings-fields triad you will see
  the structures which are more familiar to you:  
  graphs, matrices etc.

  \li
  In the above, I mentioned that you will be studying elliptic curves.
  Elliptic curves belong to an area of study called algebraic geometry.
  This is the only class where you can study a tiny bit of algebraic geometry.
  Elliptic curves are example of abelian varieties which makes
  them even more special than most curves. And you won't
  find this concept in any other class here.
  Furthermore, elliptic curves is a crucial ingredient in the proof of
  probably the most famous math problem in the world,
  Fermat's Last Theorem (FLT),
  which says that it is impossible to find integers $x > 0, y > 0, z > 0$
  such that
  \[
  x^n + y^n = z^n
  \]
  if $n > 2$. FLT is so difficult that it remained unsolved
  for more than 350 years.
  
  \li
  You will be writing python and \cpp\ programs.
  If you have done \cpp\ (CISS240 and CISS245), you will have no problems
  picking up python.
  Python is a super simple programming language and yet
  it is now the dominant language for scientific computations
  and seriously heavy data/number crunching computations.
  Therefore knowing python is crucial.
  By the way, one of the programming tool for number theory
  research, SageMath, uses python as its language for
  accessing the libraries inside SageMath.
  The guts of SageMath include libraries and
  code written in C, \cpp, Fortran, Lisp,
  and of course ... python.
  Here's an example.
  Suppose you want to compute
  \[
  \int \frac{1}{x^2 + x + 42} \ dx
  \]
  All you need to do is to run SageMath and enter this command:
  \begin{Verbatim}[frame=single, fontsize=\small]
print(integrate(1/(x**2 + x + 42), x))
  \end{Verbatim}
  and SageMath will give this to you in a split second:
  \begin{Verbatim}[frame=single, fontsize=\small]
2/167*sqrt(167)*arctan(1/167*sqrt(167)*(2*x+1))
  \end{Verbatim}
  i.e.,
  \[
  \int \frac{1}{x^2 + x + 42} \ dx
  =
  \frac{2}{167} \sqrt{167} \arctan \left( \frac{1}{167}\sqrt{167}(2x + 1) \right)
  \]
  At some point,
  I hope to incorporate SageMath into this set of notes.

  \li You will learn some parallel/concurrent programming.
  This is the only course where you will learn some basic
  parallel programming.

  \li
  You will learn probability through massive data processing programming.
  Massive data crunching is used a lot in security type programming.
  This is the only class (for now) that does basic statistical
  data crunching.
  
\end{itemize}

\section{Pre-requisites}

In terms of background, I assume
\begin{itemize}

  \li You know how to program at the level of CISS240 and CISS245.

  \li You have some discrete math background at the level of MATH225.
  In other words, you have seen some proofs,
  you know a bit of elementary number theory,
  you have a bit of combinatorics (counting),
  you know recursion, and
  you have a basic understanding of algorthms, including big-O.
  
\end{itemize}

In terms of computational tools, I assume
\begin{itemize}
  
  \li You know how to use one of our fedora virtual machine.
  I'll be using our Fedora 31 virtual machine.
  if you did not take CISS245 with me, this is not a deal breaker
  because you can easily learn how to use a virtual machine in 1 hour.
  In CISS245 I go over virtual machines and linux commands
  in one class.
  I'll just tell you what you need to do.
  
  \li You know basic linux commands (from CISS245).
  See above.
  
  \li I will be using python and \cpp\ for programming. You do not need to know
  python. If you have CISS240 and CISS245, learning python will be easy.
  In fact python is taught in many high schools.
  Also, most cryptographic programs are computational in nature
  and so the code does not use a lot of python syntax.
  
\end{itemize}

%\section{References}
%
%\begin{enumerate}[nosep]
%
%  \li Boneh-Shoup: \url{http://toc.cryptobook.us/}
%
%  \li Menezes: \url{http://cacr.uwaterloo.ca/hac/}
%  
%\end{enumerate}

